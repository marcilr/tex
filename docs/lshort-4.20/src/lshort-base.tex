%%%%%%%%%%%%%%%%%%%%%%%%%%%%%%%%%%%%%%%%%%%%%%%%%%%%%%%%%%%%%%%%%
% Contents: Main Input File of the LaTeX2e Introduction
% $Id: lshort-base.tex 374 2006-06-07 23:22:03Z marcilr $
%%%%%%%%%%%%%%%%%%%%%%%%%%%%%%%%%%%%%%%%%%%%%%%%%%%%%%%%%%%%%%%%%
% lshort.tex - The not so short introduction to LaTeX   
%                                                      by Tobias Oetiker
%                                                     oetiker@ee.ethz.ch
%
%                           based on LKURTZ.TEX Uni Graz & TU Wien, 1987
%-----------------------------------------------------------------------
%
% To compile lshort, you need TeX 3.x, LaTeX and makeindex
%
% The sources files of the Intro are:
%      lshort.tex (this file),
%      titel.tex, contrib.tex, biblio.tex
%      things.tes, typeset.tex, math.tex, lssym.tex, spec.tex,
%      lshort.sty, fancyheadings.sty
%
% Further the  verbatim.sty and the layout.sty 
% from the LaTeX Tools distribution is
% required.
%
%
% To print the AMS symbols you need the AMS fonts and the packages
% amsfonts, eufrak and eucal from (AMS LaTeX 1.2)
%
% ---------------------------------------------------------------------


\usepackage{ifpdf}
\ifpdf
\usepackage{thumbpdf}
\pdfcompresslevel=9
\RequirePackage[colorlinks,hyperindex,plainpages=false]{hyperref}
\def\pdfBorderAttrs{/Border [0 0 0] } % No border arround Links
\else
\RequirePackage[plainpages=true]{hyperref}
\usepackage{color}
\fi

\usepackage{lshort}
\usepackage{makeidx,shortvrb,latexsym}

%
% This document is ``public domain''. It may be printed and
% distributed free of charge in its original form (including the
% list of authors). If it is changed or if parts of it are used
% within another document, then the author list must include
% all the original authors AND that author (those authors) who
% has (have) made the changes.
%
% Original Copyright H.Partl, E.Schlegl, and I.Hyna (1987).
% English Version Copyright by Tobias Oetiker (1994,1995),
% 
% ---------------------------------------------------------------------
%
%
% Formats also with\textt{letterpaper} option, but the pagebreaks might not
% fall as nicely.
%
% To produce a A5 booklet, use a tool like  pstops or dvitodvi
% to  past them together in the right order. Most dvi printer drivers
% can shrink the resulting output to fit on a A4 sheet.
%
\makeindex
\typeout{Copyright T.Oetiker, H.Partl, E.Schlegl, I.Hyna}
 
\begin{document}
\selectlanguage{english}
\frontmatter
%%%%%%%%%%%%%%%%%%%%%%%%%%%%%%%%%%%%%%%%%%%%%%%%%%%%%%%%%%%%%%%%%
% Contents: The title page
% $Id: title.tex 14 2002-05-26 03:44:42Z marcilr $
%%%%%%%%%%%%%%%%%%%%%%%%%%%%%%%%%%%%%%%%%%%%%%%%%%%%%%%%%%%%%%%%%

\ifx\pdfoutput\undefined % We're not running pdftex
\else
\pdfbookmark{Title Page}{title}
\fi
\newlength{\centeroffset}
\setlength{\centeroffset}{-0.5\oddsidemargin}
\addtolength{\centeroffset}{0.5\evensidemargin}
%\addtolength{\textwidth}{-\centeroffset}
\thispagestyle{empty}
\vspace*{\stretch{1}}
\noindent\hspace*{\centeroffset}\makebox[0pt][l]{\begin{minipage}{\textwidth}
\flushright
{\Huge\bfseries The Not So Short\\ 
Introduction to \LaTeXe

}
\noindent\rule[-1ex]{\textwidth}{5pt}\\[2.5ex]
\hfill\emph{\Large Or \LaTeXe{} in \pageref{verylast} minutes}
\end{minipage}}

\vspace{\stretch{1}}
\noindent\hspace*{\centeroffset}\makebox[0pt][l]{\begin{minipage}{\textwidth}
\flushright
{\bfseries 
by Tobias Oetiker\\[1.5ex]
Hubert Partl, Irene Hyna and  Elisabeth Schlegl\\[3ex]} 
Version~3.20, 09 August, 2001
\end{minipage}}

%\addtolength{\textwidth}{\centeroffset}
\vspace{\stretch{2}}


\pagebreak
\begin{small} 
  Copyright \copyright 2000 Tobias Oetiker and all the Contributers to
  LShort.  All rights reserved.
 
  This document is free; you can redistribute it and/or modify it
  under the terms of the GNU General Public License as published by
  the Free Software Foundation; either version 2 of the License, or
  (at your option) any later version.
  
  This document is distributed in the hope that it will be useful, but
  WITHOUT ANY WARRANTY; without even the implied warranty of
  MERCHANTABILITY or FITNESS FOR A PARTICULAR PURPOSE\@.  See the GNU
  General Public License for more details.
  
  You should have received a copy of the GNU General Public License
  along with this document; if not, write to the Free Software
  Foundation, Inc., 675 Mass Ave, Cambridge, MA 02139, USA.

\end{small}


\endinput

%%% Local Variables: 
%%% mode: latex
%%% TeX-master: "lshort2e"
%%% End: 


%%%%%%%%%%%%%%%%%%%%%%%%%%%%%%%%%%%%%%%%%%%%%%%%%%%%%%%%%%%%%%%%%
% Contents: Who contributed to this Document
% $Id: contrib.tex 14 2002-05-26 03:44:42Z marcilr $
%%%%%%%%%%%%%%%%%%%%%%%%%%%%%%%%%%%%%%%%%%%%%%%%%%%%%%%%%%%%%%%%%
\chapter{Thank you!}
\noindent Much of the material used in this introduction comes from an
Austrian introduction to \LaTeX\ 2.09 written in German by:
\begin{verse}
\contrib{Hubert Partl}{partl@mail.boku.ac.at}%
{Zentraler Informatikdienst der Universit\"at f\"ur Bodenkultur Wien}
\contrib{Irene Hyna}{Irene.Hyna@bmwf.ac.at}%
   {Bundesministerium f\"ur Wissenschaft und Forschung Wien}
\contrib{Elisabeth Schlegl}{no email}%
   {in Graz}
\end{verse}

If you are interested in the German document, you can find a version
updated for \LaTeXe{} by J\"org Knappen at\\
\texttt{CTAN:/tex-archive/info/lshort/german}

\vspace{\stretch{1}}
\noindent While preparing this document, I asked
for reviewers on \texttt{comp.text.tex}. I got a lot of response. The
following individuals helped with corrections, suggestions and
material to improve this paper. They put in a big effort to help me
get this document into its present shape. I would like to
sincerely thank all of them. Naturally, all the mistakes you'll find
in this book are mine. If you ever find a word which is spelled
correctly, it must have been one of the people below dropping me a
line.


\begin{quote}
\flushleft\small
Rosemary~Bailey,        %r.a.bailey@qmw.ac.uk 0.2
Friedemann~Brauer,      %fbrauer@is.dal.ca 3.4
Jan~Busa,               % <busaj@ccsun.tuke.sk>
Markus~Br\"uhwiler,     % <m.br@switzerland.org>
David~Carlisle,         %GONE carlisle@cs.man.ac.uk 1.0
Jos\'e~Carlos~Santos,   % <jcsantos@fc.up.pt>
Mike~Chapman,           %chapman@eeh.ee.ethz.ch 3.16
Christopher~Chin,       %chris.chin@rmit.edu.au 3.1
Carl~Cerecke,           %cdc@cosc.canterbury.ac.nz>
Chris~McCormack,        %GONE chrismc@eecs.umich.edu 0.1
Wim~van~Dam,            %GONE wimvdam@cs.kun.nl 2.2
Jan~Dittberner,         %jan@jan-dittberner.de 3.15
Michael~John~Downes,    %<mjd@ams.org> 14 Oct 1999
David~Dureisseix,       %dureisse@lmt.ens-cachan.fr 1.1
Elliot,                 %GONE enh-a@minster.york.ac.uk 1.1
David~Frey,             %david@eos.lugs.ch 2.2
Robin~Fairbairns,       %Robin.Fairbairns@cl.cam.ac.uk 0.2 1.0
J\"org|~Fischer,        %j.fischer@xpoint.at 3.16
Erik~Frisk,             %frisk@isy.liu.se 3.4
Frank,                  %frank@freezone.co.uk 11 Feb 2000
Kasper~B.~Graversen,    % <kbg@dkik.dk>
Alexandre~Guimond,      %guimond@IRO.UMontreal.CA 0.9
Cyril~Goutte,           %goutte@ei.dtu.dk 2.1 2.2
Greg~Gamble,            %gregg@maths.uwa.edu.au 2.2
Neil~Hammond,           %nfh@dmu.ac.uk 0.3
Rasmus~Borup~Hansen,    %GONE rbhfamos@math.ku.dk 0.2 0.9 0.91 0.92 1.9.9
Joseph~Hilferty,        % <hilferty@fil.ub.es>
Bj\"orn Hvittfeldt,     %bjorn@hvittfeldt.com 3.13
Martien~Hulsen,         %M.A.Hulsen@WbMt.TUDelft.NL 1.0 1.1
Werner~Icking,          %<Werner.Icking@gmd.de> 3.1
Jakob,                  %diness@get2net.dk
Eric~Jacoboni,          %GONE jacoboni@enseeiht.fr 0.1 0.9
Alan~Jeffrey,           %alanje@cogs.sussex.ac.uk 0.2
Byron~Jones,            %bj@dmu.ac.uk 1.1
David~Jones,            %GONE djones@CA.McMaster.dcss.insight 1.1
Johannes-Maria~Kaltenbach, %<kaltenbach@zeiss.de> 3.01
Michael~Koundouros,     % <mkoundouros@hotmail.com>
Andrzej~Kawalec,        %GONE akawalec@prz.rzeszow.pl 1.9.9
Alain~Kessi,            %ALAIN_KESSI@HOTMAIL.COM 2.2
Christian Kern,         %ck@unixen.hrz.uni-oldenburg.de 2.1
J\"org~Knappen,         %knappen@vkpmzd.kph.uni-mainz.de 0.1
Kjetil~Kjernsmo,        %<kjetil.kjernsmo@astro.uio.no> 3.2
Maik~Lehradt,           %greek@uni-paderborn.de 0.1
Alexander~Mai,          %Alexander.Mai@physik.tu-darmstadt.de 3.8
Martin~Maechler,        %<maechler@stat.math.ethz.ch> 2.2
Aleksandar~S~Milosevic, % <aleksandar.milosevic@yale.edu>
Claus~Malten,           %GONE <ASI138%BITNET.DJUKFA11@BITNET.CEARN> 1.1
Kevin~Van~Maren,        % <vanmaren@fast.cs.utah.edu>  24 Nov 1999
Lenimar~Nunes~de~Andrade, % <lenimar@mat.ufpb.br> Fri, 12 Nov 1999
Hubert~Partl,           %partl@mail.boku.ac.at 0.2 1.1
John~Refling,           %refling@sierra.lbl.gov 0.1 0.9
Mike~Ressler,           %ressler@cougar.jpl.nasa.gov 0.1 0.2 0.9 1.0 1.9.9
Brian~Ripley,           %ripley@stats.ox.ac.uk 2.1
Young~U.~Ryu,           %ryoung@utdallas.edu 2.1
Bernd~Rosenlecher,      %9rosenle@informatik.uni-hamburg.de 10 Feb 2000
Chris~Rowley,           %C.A.Rowley@open.ac.uk 0.91
Hanspeter~Schmid,       %schmid@isi.ee.ethz.ch
Craig~Schlenter,        %cschle@lucy.ee.und.ac.za 0.1 0.2 0.9
Christopher~Sawtell,    %<csawtell@xtra.co.nz> 1 Sep 1999
Geoffrey~Swindale,      % <geofftswin@ntlworld.com>
Josef~Tkadlec,          %tkadlec@math.feld.cvut.cz 2.0 2.2
Didier~Verna,           %verna@inf.enst.fr 2.2
Fabian~Wernli,          %wernli@iap.fr 3.2
Carl-Gustav~Werner,     % <Carl-Gustav.Werner@math.lu.se> 11 Oct 1999,3.16
David~Woodhouse,        % <dwmw2@infradead.org> 3.16
Chris~York,             % <c.s.york@Cummins.com>  21 Nov 1999
Fritz~Zaucker,          %zaucker@ee.ethz.ch 3.0
Rick~Zaccone,           %zaccone@bucknell.edu 2.2
and Mikhail~Zotov.      %zotov@eas.npi.msu.su 3.1
\end{quote}

\vspace*{\stretch{1}}



\pagebreak
\endinput
%%% Local Variables: 
%%% mode: latex
%%% TeX-master: "lshort"
%%% End: 

%%%%%%%%%%%%%%%%%%%%%%%%%%%%%%%%%%%%%%%%%%%%%%%%%%%%%%%%%%%%%%%%%
% Contents: Who contributed to this Document
% $Id: overview.tex 374 2006-06-07 23:22:03Z marcilr $
%%%%%%%%%%%%%%%%%%%%%%%%%%%%%%%%%%%%%%%%%%%%%%%%%%%%%%%%%%%%%%%%%

% Because this introduction is the reader's first impression, I have
% edited very heavily to try to clarify and economize the language.
% I hope you do not mind! I always try to ask "is this word needed?"
% in my own writing but I don't want to impose my style on you... 
% but here I think it may be more important than the rest of the book.
% --baron

\chapter{Preface}

\LaTeX{} \cite{manual} is a typesetting system that is very 
suitable for producing scientific and mathematical documents of high
typographical quality. It is also suitable for producing all
sorts of other documents, from simple letters to complete books.
\LaTeX{} uses \TeX{} \cite{texbook} as its formatting engine.

This short introduction describes \LaTeXe{} and should be sufficient
for most applications of \LaTeX. Refer to~\cite{manual,companion} for
a complete description of the \LaTeX{} system.

\bigskip
\noindent This introduction is split into 6 chapters:
\begin{description}
\item[Chapter 1] tells you about the basic structure of \LaTeXe{}
  documents. You will also learn a bit about the history of \LaTeX{}.
  After reading this chapter, you should have a rough understanding how
  \LaTeX{} works.
\item[Chapter 2] goes into the details of typesetting your
  documents. It explains most of the essential \LaTeX{} commands and
  environments. After reading this chapter, you will be able to write
  your first documents. 
\item[Chapter 3] explains how to typeset formulae with \LaTeX. Many
  examples demonstrate how to use one of \LaTeX{}'s
  main strengths. At the end of the chapter are tables listing
  all mathematical symbols available in \LaTeX{}.
\item[Chapter 4] explains indexes,  bibliography generation and
  inclusion of EPS graphics. It introduces creation of PDF documents with pdf\LaTeX{}
  and presents some handy extension packages.
\item[Chapter 5] shows how to use \LaTeX{} for creating graphics. Instead
  of drawing a picture with some graphics program, saving it to a file and
  then including it into \LaTeX{} you describe the picture and have \LaTeX{}
  draw it for you.
\item[Chapter 6] contains some potentially dangerous information about
  how to alter the
  standard document layout produced by \LaTeX{}. It will tell you how  to
  change things such that the beautiful output of \LaTeX{}
  turns ugly or stunning, depending on your abilities.
\end{description}
\bigskip
\noindent It is important to read the chapters in order---the book is
not that big, after all. Be sure to carefully read the examples,
because a lot of the information is in the
examples placed throughout the book.

\bigskip
\noindent \LaTeX{} is available for most computers, from the PC and Mac to large
UNIX and VMS systems. On many university computer clusters you will
find that a \LaTeX{} installation is available, ready to use.
Information on how to access
the local \LaTeX{} installation should be provided in the \guide. If
you have problems getting started, ask the person who gave you this
booklet. The scope of this document is \emph{not} to tell you how to
install and set up a \LaTeX{} system, but to teach you how to write
your documents so that they can be processed by~\LaTeX{}.

\bigskip
\noindent If you need to get hold of any \LaTeX{} related material, 
have a look at one of the Comprehensive \TeX{} Archive Network
(\texttt{CTAN}) sites. The homepage is at
\texttt{http://www.ctan.org}. All packages can also be retrieved from
the ftp archive \texttt{ftp://www.ctan.org} and its mirror
sites all over the world.

You will find other references to CTAN throughout the book, especially
pointers to software and documents you might want to download. Instead
of writing down complete urls, I just wrote \texttt{CTAN:} followed by
whatever location within the CTAN tree you should go to. 

If you want to run \LaTeX{} on your own computer, take a look at what
is available from \CTAN|systems|.

\vspace{\stretch{1}}
\noindent If you have ideas for something to be
added, removed or altered in this document, please let me know. I am
especially interested in feedback from \LaTeX{} novices about which
bits of this intro are easy to understand and which could be explained
better.

\bigskip
\begin{verse}
\contrib{Tobias Oetiker}{oetiker@ee.ethz.ch}%
\noindent{Department of Information Technology and\\ Electrical
Engineering,\\
Swiss Federal Institute of Technology}
\end{verse}
\vspace{\stretch{1}}
\noindent The current version of this document is available on\\
\CTAN|info/lshort|

\endinput



%

% Local Variables:
% TeX-master: "lshort2e"
% mode: latex
% mode: flyspell
% End:

\tableofcontents
\listoffigures
\listoftables
\enlargethispage{\baselineskip}
\mainmatter
%%%%%%%%%%%%%%%%%%%%%%%%%%%%%%%%%%%%%%%%%%%%%%%%%%%%%%%%%%%%%%%%%
% Contents: Things you need to know
% $Id: things.tex 14 2002-05-26 03:44:42Z marcilr $
%%%%%%%%%%%%%%%%%%%%%%%%%%%%%%%%%%%%%%%%%%%%%%%%%%%%%%%%%%%%%%%%%
 
\chapter{Things You Need to Know}
\begin{intro}
In the first part of this chapter, you will get a short 
overview about the philosophy and history of \LaTeXe. The second part
of the chapter focuses on the basic structures of a \LaTeX{} document. 
After reading this chapter, you should have a rough knowledge
of how \LaTeX{} works. When reading on, this will help you to integrate
all the new information into the big picture.  
\end{intro}

\section{The Name of the Game}
\subsection{\TeX}
 
\TeX{} is a computer program created by \index{Knuth, Donald E.}Donald
E. Knuth \cite{texbook}. It is aimed at typesetting text and
mathematical formulae. Knuth started writing the \TeX{} typesetting
engine in 1977 to explore the potential of the digital printing
equipment that was beginning to infiltrate the publishing industry at
that time, especially in the hope that he could reverse the trend of
deteriorating typographical quality that he saw affecting his own
books and articles. \TeX{} as we use it today was released in 1982,
with some slight enhancements added in 1989 to better support 8-bit
characters and multiple languages. \TeX{} is renowned for being
extremely stable, for running on many different kinds of computers,
and for being virtually bug free. The version number of \TeX{} is
converging to $\pi$ and is now at $3.14159$.
                                                                       
\TeX{} is pronounced ``Tech,'' with a ``ch'' as in the German word
``Ach'' or in the Scottish ``Loch.'' In an ASCII environment, \TeX{}
becomes \texttt{TeX}.

\subsection{\LaTeX}
 
\LaTeX{} is a macro package which enables authors to typeset and print
their work at the highest typographical quality, using a predefined,
professional layout. \LaTeX{} was originally written by
\index{Lamport, Leslie}Leslie Lamport~\cite{manual}. It uses the
\TeX{} formatter as its typesetting engine.

In 1994 the \LaTeX{} package was updated by the \index{LaTeX3@\LaTeX
  3}\LaTeX 3 team, led by \index{Mittelbach, Frank}Frank Mittelbach,
to include some long-requested improvements, and to re\-unify all the
patched versions which had cropped up since the release of
\index{LaTeX 2.09@\LaTeX{} 2.09}\LaTeX{} 2.09 some years earlier. To
distinguish the new version from the old, it is called \index{LaTeX
  2e@\LaTeXe}\LaTeXe. This documentation deals with \LaTeXe.

\LaTeX{} is pronounced ``Lay-tech'' or ``Lah-tech.'' If you refer to
\LaTeX{} in an \texttt{ASCII} environment, you type \texttt{LaTeX}.
\LaTeXe{} is pronounced ``Lay-tech two e'' and typed \texttt{LaTeX2e}.

Figure~\ref{components} above % on page \pageref{components}
shows how \TeX{} and \LaTeXe{} work together. This figure is taken from
\texttt{wots.tex} by Kees van der Laan.

\begin{figure}[btp]
\begin{lined}{0.8\textwidth}
\begin{center}
\input{kees.fig}
\end{center}
\end{lined}
\caption{Components of a \TeX{} System.} \label{components}
\end{figure}

\section{Basics}
 
\subsection{Author, Book Designer, and Typesetter}

To publish something, authors give their typed manuscript to a
publishing company. One of their book designers then
decides the layout of the document (column width, fonts, space before
and after headings,~\ldots). The book designer writes his instructions
into the manuscript and then gives it to a typesetter, who typesets the
book according to these instructions.

A human book designer tries to find out what the author had in mind
while writing the manuscript. He decides on chapter headings,
citations, examples, formulae, etc.\ based on his professional
knowledge and from the contents of the manuscript.

In a \LaTeX{} environment, \LaTeX{} takes the role of the book
designer and uses \TeX{} as its typesetter. But \LaTeX{} is ``only'' a
program and therefore needs more guidance. The author has to provide
additional information which describes the logical structure of his
work. This information is written into the text as ``\LaTeX{}
commands.''

This is quite different from the \wi{WYSIWYG}\footnote{What you see is
  what you get.} approach which most modern word processors such as
\emph{MS Word} or \emph{Corel WordPerfect} take. With these
applications, authors specify the document layout interactively while
typing text into the computer. All along the way, they can see on the
screen how the final work will look when it is printed.

When using \LaTeX{} it is normally not possible to see the final output
while typing the text. But the final output can be previewed on the
screen after processing the file with \LaTeX. Then corrections can be
made before actually sending the document to the printer.

\subsection{Layout Design}

Typographical design is a craft. Unskilled authors often commit
serious formatting errors by assuming that book design is mostly a
question of aesthetics---``If a document looks good artistically,
it is well designed.'' But as a document has to be read and not hung
up in a picture gallery, the readability and understandability is of
much greater importance than the beautiful look of it.
Examples: 
\begin{itemize}
\item The font size and the numbering of headings have to be chosen to make
  the structure of chapters and sections clear to the reader.
\item The line length has to be short enough to not strain
  the eyes of the reader, while long enough to fill the page
  beautifully.
\end{itemize}

With \wi{WYSIWYG} systems, authors often generate aesthetically
pleasing documents with very little or inconsistent structure.
\LaTeX{} prevents such formatting errors by forcing the author to
declare the \emph{logical} structure of his document. \LaTeX{} then
chooses the most suitable layout.

\subsection{Advantages and Disadvantages}

When People from the \wi{WYSIWYG} world meet people who use \LaTeX{},
they often discuss ``the \wi{advantages of \LaTeX{}} over a normal
word processor'' or the opposite.  The best thing you can do when such
a discussion starts is to keep a low profile, since such discussions
often get out of hand. But sometimes you cannot escape \ldots

\medskip\noindent So here is some ammunition. The main advantages
of \LaTeX{} over normal word processors are the following:

\begin{itemize}

\item Professionally crafted layouts are available, which make a
  document really look as if ``printed.''
\item The typesetting of mathematical formulae is supported in a
  convenient way.
\item The user only needs to learn a few easy-to-understand commands
  which specify the logical structure of a document. They almost never
  need to tinker with the actual layout of the document.
\item Even complex structures such as footnotes, references, table of
  contents, and bibliographies can be generated easily.
\item Free add-on packages exist for many typographical tasks not directly supported by basic
  \LaTeX. For example, packages are
  available to include \textsc{PostScript} graphics or to typeset
  bibliographies conforming to exact standards. Many of these add-on
  packages are described in \companion.
\item \LaTeX{} encourages authors to write well-structured texts,
  because this is how \LaTeX{} works---by specifying structure.
\item \TeX, the formatting engine of \LaTeXe, is highly portable and free.
  Therefore the system runs on almost any hardware platform
  available. 

%
% Add examples ...
%
\end{itemize}

\medskip

\noindent\LaTeX{} also has some disadvantages, and I guess it's a bit
difficult for me to find any sensible ones, though I am sure other people
can tell you hundreds \texttt{;-)}

\begin{itemize}
\item \LaTeX{} does not work well for people who have sold their
  souls \ldots
\item Although some parameters can be adjusted within a predefined
  document layout, the design of a whole new layout is difficult and
  takes a lot of time.\footnote{Rumour says that this is one of the
    key elements which will be addressed in the upcoming \LaTeX 3
    system.}\index{LaTeX3@\LaTeX 3}
\item It is very hard to write unstructured and disorganized documents.
\item Your hamster might, despite some encouraging first steps, never be
able to fully grasp the concept of Logical Markup.
\end{itemize}
 
\section{\LaTeX{} Input Files}

The input for \LaTeX{} is a plain \texttt{ASCII} text file. You can create it
with any text editor. It contains the text of the document as well as
the commands which tell \LaTeX{} how to typeset the text.

\subsection{Spaces}

``Whitespace'' characters such as blank or tab are
treated uniformly as ``\wi{space}'' by \LaTeX{}. \emph{Several
  consecutive} \wi{whitespace} characters are treated as \emph{one}
``space''.  Whitespace at the start of a line is generally ignored, and
a single linebreak is treated as ``whitespace''.
\index{whitespace!at the start of a line}

An empty line between two lines of text defines the end of a
paragraph. \emph{Several} empty lines are treated the same as
\emph{one} empty line. The text below is an example. On the left hand
side is the text from the input file, and on the right hand side is the
formatted output.

\begin{example}
It does not matter whether you
enter one or several     spaces
after a word.

An empty line starts a new 
paragraph.
\end{example}
 
\subsection{Special Characters}

The following symbols are \wi{reserved characters} that either have a
special meaning under \LaTeX{} or are not available in all the fonts.
If you enter them directly in your text, they will normally not print,
but rather coerce \LaTeX{} to do things you did not intend. 
\begin{code}
\verb.#  $  %  ^  &  _  {  }  ~  \ . %$
\end{code}

As you will see, these characters can be used in your documents all
the same by adding a prefix backslash:

\begin{example}
\# \$ \% \^{} \& \_ \{ \} \~{} 
\end{example}

The other symbols and many more can be printed with special commands
in mathematical formulae or as accents. The backslash character
$\backslash$ can \emph{not} be entered by adding another backslash
in front of it (\verb|\\|), this sequence is used for
linebreaking.\footnote{Try the \texttt{\$}\ci{backslash}\texttt{\$} command instead. It
  produces a `$\backslash$'.}

\subsection{\LaTeX{} Commands}

\LaTeX{} \wi{commands} are case sensitive and take one of the following
two formats:

\begin{itemize}
\item They start with a \wi{backslash} \verb|\| and then have a name
 consisting of letters only. Command names are terminated by a
 space, a number or any other `non-letter'.
\item They consist of a backslash and exactly one % numerical or
 special character.
\end{itemize}

%
% \\* doesn't comply !
%

%
% Can \3 be a valid command ? (jacoboni)
%
\label{whitespace}

\LaTeX{} ignores whitespace after commands. If you want to get a
\index{whitespace!after commands}space after a command, you have to
put either \verb|{}| and a blank or a special spacing command after the
command name. The \verb|{}| stops \LaTeX{} from eating up all the space after
the command name. 

\begin{example}
I read that Knuth divides the 
people working with \TeX{} into 
\TeX{}nicians and \TeX perts.\\
Today is \today.
\end{example}

Some commands need a \wi{parameter} which has to be given between
\wi{curly braces} \verb|{ }| after the command name. Some commands support
\wi{optional parameters} which are added after the command name in
\wi{square brackets}~\verb|[ ]|. The next examples use some \LaTeX{}
commands. Don't worry about them, they will be explained later.

\begin{example}
You can \textsl{lean} on me!
\end{example}
\begin{example}
Please, start a new line
right here!\newline
Thank you!
\end{example}

\subsection{Comments}
\index{comments}

When \LaTeX{} encounters a \verb|%| character while processing an input file,
it ignores the rest of the present line, the linebreak, and all
whitespace at the beginning of the next line.

This can be used to write notes into the input file, which will not show up
in the printed version.

\begin{example}
This is an % stupid
% Better: instructive <----
example: Supercal%
              ifragilist%
    icexpialidocious
\end{example}

The \texttt{\%} character can also be used to split long input lines where no
whitespace or linebreaks are allowed.

For longer comments you should use the \ei{comment} environment
provided by the \pai{verbatim} package. This means, to use the
\ei{comment} environment you have to add the commend
\verb|\usepackage{verbatim}| to the preamble of your document.

\begin{example}
This is another
\begin{comment}
rather stupid,
but helpful
\end{comment}
example for embedding
comments in your document.
\end{example}

Note that this won't work inside complex environments like math for example.

\section{Input File Structure}

When \LaTeXe{} processes an input file, it expects it to follow a
certain \wi{structure}. Thus every input file must start with the
command
\begin{code}
\verb|\documentclass{...}|
\end{code}
This specifies what sort of document you intend to write. After that,
you can include commands which influence the style of the whole
document, or you can load \wi{package}s which add new
features to the \LaTeX{} system. To load such a package you use the
command
\begin{code}
\verb|\usepackage{...}|
\end{code}

When all the setup work is done,\footnote{The area between \texttt{\bs
    documentclass} and \texttt{\bs
    begin$\mathtt{\{}$document$\mathtt{\}}$} is called
  \emph{\wi{preamble}}.} you start the body of the text with the
command

\begin{code}
\verb|\begin{document}|
\end{code}

Now you enter the text mixed with some useful \LaTeX{} commands.  At
the end of the document you add the
\begin{code}
\verb|\end{document}|
\end{code}
command, which tells \LaTeX{} to call it a day. Anything which
follows this command will be ignored by \LaTeX.

Figure~\ref{mini} shows the contents of a minimal \LaTeXe{} file. A
slightly more complicated \wi{input file} is given in
Figure~\ref{document}.

\begin{figure}[!bp]
\begin{lined}{6cm}
\begin{verbatim}
\documentclass{article}
\begin{document}
Small is beautiful.
\end{document}
\end{verbatim}
\end{lined}
\caption{A Minimal \LaTeX{} File.} \label{mini}
\end{figure}
 
\begin{figure}[!bp]
\begin{lined}{10cm}
\begin{verbatim}
\documentclass[a4paper,11pt]{article}
% define the title
\author{H.~Partl}
\title{Minimalism}
\begin{document}
% generates the title
\maketitle 
% insert the table of contents
\tableofcontents
\section{Start}
Well, and here begins my lovely article.
\section{End}
\ldots{} and here it ends.
\end{document}
\end{verbatim}
\end{lined}
\caption{Example of a Realistic Journal Article.} \label{document}
\end{figure}

\section{A Typical Commandline Session}

I bet you must be dying to try out the neat small \LaTeX{} input file
shown on page \pageref{mini}. Here is some help:
\LaTeX{} itself comes without a GUI or
fancy buttons to press. It is just a program which crunches away
on your input file. Some \LaTeX{} installations feature a graphical
front end where you can click \LaTeX{} into compiling your input file.
But Real Men don't Click, so here is how to coax \LaTeX{} into
compiling your input file on a text based system. Please note, this
description assumes that a working \LaTeX{} installation already sitts
on your computer.

\begin{enumerate}
\item 
  
  Edit/Create your \LaTeX{} input file. This file must be plain ASCII
  text.  On Unix all the editors will create just that. On windows you
  might want to make sure that you save the file in ASCII or
  \emph{Plain Text} format.  When picking a name for your file, make
  sure it bears the extention \texttt{.tex}.

\item 
Run \LaTeX{} on your input file. If successful you will end up
with a \texttt{.dvi} file.
\begin{verbatim}
latex foo.tex
\end{verbatim}

\item 
Now you may view the DVI file.
\begin{verbatim}
xdvi foo.dvi
\end{verbatim}
or
convert it to PS
\begin{verbatim}
dvips -Pcmz foo.dvi -o foo.ps
\end{verbatim}
\texttt{\wi{xdvi}} and \texttt{\wi{dvips}} are open-source
tools for handling \texttt{.dvi} files. The first displays them on
screen within the X11 environment and the other
creates a PostScript file for printing. If you are not working on
a Unix system, other means for handling the \texttt{.dvi} files may be
provided. 

\end{enumerate}

 
\section{The Layout of the Document}
 
\subsection {Document Classes}\label{sec:documentclass}

The first information \LaTeX{} needs to know when processing an
input file is the type of document the author wants to create. This
is specified with the \ci{documentclass} command.
\begin{lscommand}
\ci{documentclass}\verb|[|\emph{options}\verb|]{|\emph{class}\verb|}|
\end{lscommand}
\noindent Here \emph{class} specifies the type of document to be created.
Table~\ref{documentclasses} lists the document classes explained in
this introduction. The \LaTeXe{} distribution provides additional
classes for other documents, including letters and slides.  The
\emph{\wi{option}s} parameter customises the behaviour of the document
class. The options have to be separated by commas. The most common options for the standard document
classes are listed in
Table~\ref{options}.


\begin{table}[!bp]
\caption{Document Classes.} \label{documentclasses}
\begin{lined}{12cm}
\begin{description}
 
\item [\normalfont\texttt{article}] for articles in scientific journals, presentations,
  short reports, program documentation, invitations, \ldots
  \index{article class}
\item [\normalfont\texttt{report}] for longer reports containing several chapters, small
  books, PhD theses, \ldots \index{report class}
\item [\normalfont\texttt{book}] for real books \index{book class}
\item [\normalfont\texttt{slides}] for slides. The class uses big sans serif
  letters. You might want to consider using Foil\TeX{}\footnote{%
        \texttt{CTAN:/tex-archive/macros/latex/contrib/supported/foiltex}} instead.
        \index{slides class}\index{foiltex}
\end{description}
\end{lined}
\end{table}

\begin{table}[!bp]
\caption{Document Class Options.} \label{options}
\begin{lined}{12cm}
\begin{flushleft}
\begin{description}
\item[\normalfont\texttt{10pt}, \texttt{11pt}, \texttt{12pt}] \quad Sets the size
  of the main font in the document. If no option is specified,
  \texttt{10pt} is assumed.  \index{document font size}\index{base
    font size}
\item[\normalfont\texttt{a4paper}, \texttt{letterpaper}, \ldots] \quad Defines
  the paper size. The default size is \texttt{letterpaper}. Besides
  that, \texttt{a5paper}, \texttt{b5paper}, \texttt{executivepaper},
  and \texttt{legalpaper} can be specified.  \index{legal paper}
  \index{paper size}\index{A4 paper}\index{letter paper} \index{A5
    paper}\index{B5 paper}\index{executive paper}

\item[\normalfont\texttt{fleqn}] \quad Typesets displayed formulae left-aligned
  instead of centred.

\item[\normalfont\texttt{leqno}] \quad Places the numbering of formulae on the
  left hand side instead of the right.

\item[\normalfont\texttt{titlepage}, \texttt{notitlepage}] \quad Specifies
  whether a new page should be started after the \wi{document title}
  or not. The \texttt{article} class does not start a new page by
  default, while \texttt{report} and \texttt{book} do.  \index{title}

\item[\normalfont\texttt{onecolumn}, \texttt{twocolumn}] \quad Instructs \LaTeX{} to typeset the
  document in \wi{one column}\wi{two column}s.

\item[\normalfont\texttt{twoside, oneside}] \quad Specifies whether double or
  single sided output should be generated. The classes
  \texttt{article} and \texttt{report} are \wi{single sided} and the
  \texttt{book} class is \wi{double sided} by default. Note that this
  option concerns the style of the document only. The option
  \texttt{twoside} does \emph{not} tell the printer you use that it
  should actually make a two-sided printout.

\item[\normalfont\texttt{openright, openany}] \quad Makes chapters begin either
  only on right hand pages or on the next page available. This does
  not work with the \texttt{article} class, as it does not know about
  chapters. The \texttt{report} class by default starts chapters on
  the next page available and the \texttt{book} class starts them on
  right hand pages.

\end{description}
\end{flushleft}
\end{lined}
\end{table}

Example: An input file for a \LaTeX{} document could start with the
line
\begin{code}
\ci{documentclass}\verb|[11pt,twoside,a4paper]{article}|
\end{code}
which instructs \LaTeX{} to typeset the document as an \emph{article}
with a base font size of \emph{eleven points}, and to produce a
layout suitable for \emph{double sided} printing on \emph{A4 paper}.
\pagebreak[2]
\subsection{Packages}
\index{package} While writing your document, you will probably find
that there are some areas where basic \LaTeX{} cannot solve your
problem. If you want to include \wi{graphics}, \wi{coloured text} or
source code from a file into your document, you need to enhance the
capabilities of \LaTeX.  Such enhancements are called packages.
Packages are activated with the
\begin{lscommand}
\ci{usepackage}\verb|[|\emph{options}\verb|]{|\emph{package}\verb|}|
\end{lscommand}
\noindent command where \emph{package} is the name of the package and
\emph{options} is a list of keywords which trigger special features in
the package. Some packages come with the \LaTeXe{} base distribution
(See Table~\ref{packages}). Others are provided separately. You may
find more information on the packages installed at your site in your
\guide. The prime source for information about \LaTeX{} packages is \companion.
It contains descriptions of hundreds of packages along with
information of how to write your own extensions to \LaTeXe.

\begin{table}[!hbp]
\caption{Some of the Packages Distributed with \LaTeX.} \label{packages}
\begin{lined}{11cm}
\begin{description}
\item[\normalfont\pai{doc}] Allows the documentation of \LaTeX{} programs.\\
 Described in \texttt{doc.dtx}\footnote{This file should be installed
   on your system, and you should be able to get a \texttt{dvi} file
   by typing \texttt{latex doc.dtx} in any directory where you have
   write permission. The same is true for all the
   other files mentioned in this table.}  and in \companion.

\item[\normalfont\pai{exscale}] Provides scaled versions of the
  math extension  font.\\ 
  Described in \texttt{ltexscale.dtx}.

\item[\normalfont\pai{fontenc}] Specifies which \wi{font encoding}
  \LaTeX{} should use.\\
  Described in \texttt{ltoutenc.dtx}.

\item[\normalfont\pai{ifthen}] Provides commands of the form\\ 
  `if\ldots then do\ldots otherwise do\ldots.'\\ Described in
  \texttt{ifthen.dtx} and \companion.

\item[\normalfont\pai{latexsym}] To access the \LaTeX{} symbol
  font, you should use the \texttt{latexsym} package. Described in
  \texttt{latexsym.dtx} and in \companion.
 
\item[\normalfont\pai{makeidx}] Provides commands for producing
  indexes.  Described in section~\ref{sec:indexing} and in \companion.

\item[\normalfont\pai{syntonly}] Processes a document without
  typesetting it.
  
\item[\normalfont\pai{inputenc}] Allows the specification of an
  input encoding such as ASCII, ISO Latin-1, ISO Latin-2, 437/850 IBM
  code pages,  Apple Macintosh, Next, ANSI-Windows or user-defined one.
  Described in \texttt{inputenc.dtx}. 
\end{description}
\end{lined}
\end{table}

\section{Files you might encounter}

When you work with \LaTeX{} you will soon find yourself in a maze of
files with various \wi{extension}s and probably no clue. Below there is a
list telling about the various \wi{file types} you might encounter when
working with \TeX{}. Please note that this table does not claim to be
a complete list of extensions, but if you find one missing which you
think is important, please drop a line.

\begin{description}
  
\item[\wi{.tex}] \LaTeX{} or \TeX{} input file. Can be compiled with
  \texttt{latex}.
\item[\wi{.sty}] \LaTeX{} Macro package. This is a file you can load
  into your \LaTeX{} document using the \ci{usepackage} command.
\item[\wi{.dtx}] Documented \TeX{}. This is the main distribution
  format for \LaTeX{} style files. If you process a .dtx file you get
  documented macro code of the \LaTeX{} package contained in the .dtx
  file.
\item[\wi{.ins}] Is the installer for the files contained in the
  matching .dtx file. If you download a \LaTeX{} package from the net,
  you will normally get a .dtx and a .ins file. Run \LaTeX{} on the
  .ins file to unpack the .dtx file.
\item[\wi{.cls}] Class files define what your document looks
  like. They are selected with the \ci{documentclass} command.
\end{description}
The following files are generated when you run \LaTeX{} on your input
file:

\begin{description}
\item[\wi{.dvi}] Device Independent file. This is the main result of a \LaTeX{}
  compile run. You can look at its content with a DVI previewer
  program or you can send it to a printer with \texttt{dvips} or a
  similar application.
\item[\wi{.log}] Gives a detailed account of what happened during the
  last compiler run.
\item[\wi{.toc}] Stores all your section headers. It gets read in for the
  next compiler run and is used to produce the table of content.
\item[\wi{.lof}] This is like .toc but for the list of figures.
\item[\wi{.lot}] And again the same for the list of tables.
\item[\wi{.aux}] Another file which transports information from one
  compiler run to the next. Among other things, the .aux file is used
  to store information associated with crossreferences.
\item[\wi{.idx}] If your document contains an index. \LaTeX{} stores all
  the words which go into the index in this file. Process this file with
  \texttt{makeindex}. Refer to section \ref{sec:indexing} on
  page \pageref{sec:indexing} for more information on indexing.
\item[\wi{.ind}] Is the processed .idx file, ready for inclusion into your
  document on the next compile cycle.
\item[\wi{.ilg}] Logfile telling about what \texttt{makeindex} did.
\end{description}


% Package Info pointer
%
%

\subsection{Page Styles}
 
\LaTeX{} supports three predefined \wi{header}/\wi{footer}
combinations---so-called \wi{page style}s. The \emph{style} parameter
of the \index{page style!plain@\texttt{plain}}\index{plain@\texttt{plain}}
\index{page style!headings@\texttt{headings}}\index{headings@texttt{headings}}
\index{page style!empty@\texttt{empty}}\index{empty@\texttt{empty}}
\begin{lscommand}
\ci{pagestyle}\verb|{|\emph{style}\verb|}|
\end{lscommand}
\noindent command defines which one to use. 
Table~\ref{pagestyle}
lists the predefined page styles.

\begin{table}[!hbp]
\caption{The Predefined Page Styles of \LaTeX.} \label{pagestyle}
\begin{lined}{12cm}
\begin{description}

\item[\normalfont\texttt{plain}] prints the page numbers on the bottom
  of the page, in the middle of the footer. This is the default page
  style.

\item[\normalfont\texttt{headings}] prints the current chapter heading
  and the page number in the header on each page, while the footer
  remains empty.  (This is the style used in this document)
\item[\normalfont\texttt{empty}] sets both the header and the footer
  to be empty.

\end{description}
\end{lined}
\end{table}

It is possible to change the page style of the current page
with the command
\begin{lscommand}
\ci{thispagestyle}\verb|{|\emph{style}\verb|}|
\end{lscommand}
A description how to create your own
headers and footers can be found in \companion{} and in section~\ref{sec:fancy} on page~\pageref{sec:fancy}.
%
% Pointer to the Fancy headings Package description !
%


%
% Add Info on page-numbering, ...
% \pagenumbering

\section{Big Projects}
When working on big documents, you might want to split the input file
into several parts. \LaTeX{} has two commands which help you to do
that.

\begin{lscommand}
\ci{include}\verb|{|\emph{filename}\verb|}|
\end{lscommand}
\noindent you can use this command in the document body to insert the
contents of another file named \emph{filename.tex}. Note that \LaTeX{}
will start a new page
before processing the material input from \emph{filename.tex}.

The second command can be used in the preamble. It allows you to
instruct \LaTeX{} to only input some of the \verb|\include|d files.
\begin{lscommand}
\ci{includeonly}\verb|{|\emph{filename}\verb|,|\emph{filename}%
\verb|,|\ldots\verb|}|
\end{lscommand}
After this command is executed in the preamble of the document, only
\ci{include} commands for the filenames which are listed in the
argument of the \ci{includeonly} command will be executed. Note that
there must be no spaces between the filenames and the commas.

The \ci{include} command starts typesetting the included text on a new
page. This is helpful when you use \ci{includeonly}, because the
pagebreaks will not move, even when some included files are omitted.
Sometimes this might not be desirable. In this case, you can use the
\begin{lscommand}
\ci{input}\verb|{|\emph{filename}\verb|}|
\end{lscommand}
\noindent command. It simply includes the file specified. 
No flashy suits, no strings attached.


To make \LaTeX{} quickly check your document you can use the \pai{syntonly}
package. This makes \LaTeX{} skim through your document only checking for
proper syntax and usage of the commands, but doesn't produce any (DVI) output.
As \LaTeX{} runs faster in this mode you may save yourself valuable time.
Usage is very simple:

\begin{verbatim}
\usepackage{syntonly}
\syntaxonly
\end{verbatim}

When you want to produce pages, just comment out the second line
(by adding a percent sign).


%%% Local Variables: 
%%% mode: latex
%%% TeX-master: "lshort"
%%% End: 

%%%%%%%%%%%%%%%%%%%%%%%%%%%%%%%%%%%%%%%%%%%%%%%%%%%%%%%%%%%%%%%%%
% Contents: Typesetting Part of LaTeX2e Introduction
% $Id: typeset.tex 14 2002-05-26 03:44:42Z marcilr $
%%%%%%%%%%%%%%%%%%%%%%%%%%%%%%%%%%%%%%%%%%%%%%%%%%%%%%%%%%%%%%%%%
\chapter{Typesetting Text}

\begin{intro}
  After reading the previous chapter, you should know about the basic
  stuff of which a \LaTeXe{} document is made. In this chapter I
  will fill in the remaining structure you will need to know in order
  to produce real world material.
\end{intro}

\section{The Structure of Text and Language}

The main point of writing a text (some modern DAAC\footnote{Different
  At All Cost, a translation of the Swiss German UVA (Um's Verrecken
  Anders).} literature excluded), is to convey ideas, information, or
knowledge to the reader.  The reader will understand the text better
if these ideas are well-structured, and will see and feel this
structure much better if the typographical form reflects the logical
and semantical structure of the content.

\LaTeX{} is different from other typesetting systems in that you just
have to tell it the logical and semantical structure of a text.  It
then derives the typographical form of the text according to the
``rules'' given in the document class file and in various style files.

The most important text unit in \LaTeX{} (and in typography) is the
\wi{paragraph}.  We call it ``text unit'' because a paragraph is the
typographical form which should reflect one coherent thought, or one
idea.  You will learn in the following sections, how you can force
linebreaks with e.g.{} \texttt{\bs\bs} and paragraph breaks with e.g.{} 
leaving an empty line in the source code.  Therefore, if a new thought
begins, a new paragraph should begin, and if not, only linebreaks
should be used.  If in doubt about paragraph breaks, think about your
text as a conveyor of ideas and thoughts.  If you have a paragraph
break, but the old thought continues, it should be removed.  If some
totally new line of thought occurs in the same paragraph, then it
should be broken.

Most people completely underestimate the importance of well-placed
paragraph breaks.  Many people do not even know what the meaning of
a paragraph break is, or, especially in \LaTeX, introduce paragraph
breaks without knowing it.  The latter mistake is especially easy to
make if equations are used in the text.  Look at the following
examples, and figure out why sometimes empty lines (paragraph breaks)
are used before and after the equation, and sometimes not.  (If you
don't yet understand all commands well enough to understand these
examples, please read this and the following chapter, and then read
this section again.)

\begin{code}
\begin{verbatim}
% Example 1
\ldots when Einstein introduced his formula 
\begin{equation} 
  e = m \cdot c^2 \; , 
\end{equation} 
which is at the same time the most widely known 
and the least well understood physical formula. 


% Example 2
\ldots from which follows Kirchoff's current law:
\begin{equation} 
  \sum_{k=1}^{n} I_k = 0 \; .
\end{equation} 

Kirchhoff's voltage law can be derived \ldots


% Example 3
\ldots which has several advantages.

\begin{equation} 
  I_D = I_F - I_R
\end{equation} 
is the core of a very different transistor model. \ldots
\end{verbatim}
\end{code} 

The next smaller text unit is a sentence.  In English texts, there is
a larger space after a period which ends a sentence than after one
which ends an abbreviation.  \LaTeX{} tries to figure out which one
you wanted to have.  If \LaTeX{} gets it wrong, you must tell it what
you want.  This is explained later in this chapter.

The structuring of text even extends to parts of sentences.  Most
languages have very complicated punctuation rules, but in many
languages (including German and English), you will get almost every
comma right if you remember what it represents: a short stop in the
flow of language.  If you are not sure about where to put a comma,
read the sentence aloud, and take a short breath at every comma.  If
this feels awkward at some place, delete that comma, if you feel the
urge to breathe (or make a short stop) at some other place, insert a
comma.

Finally, the paragraphs of a text should also be structured logically
at a higher level, by putting them into chapters, sections,
subsections, and so on.  However, the typographical effect of writing
e.g.{} \verb|\section{The| \texttt{Structure of Text and Language}\verb|}| is
so obvious that it is almost self-evident how these high-level
structures should be used.

\section{Linebreaking and Pagebreaking}
 
\subsection{Justified Paragraphs}

Often books are typeset with each line having the same length.
\LaTeX{} inserts the necessary \wi{linebreak}s and spaces between words
by optimizing the contents of a whole paragraph. If necessary, it
also hyphenates words that would not fit comfortably on a line.
How the paragraphs are typeset depends on the document class.
Normally the first line of a paragraph is indented, and there is no
additional space between two paragraphs. Refer to section~\ref{parsp}
for more information.

In special cases it might be necessary to order \LaTeX{} to break a
line: 
\begin{lscommand}
\ci{\bs} or \ci{newline} 
\end{lscommand}
\noindent starts a new line without starting a new paragraph. 

\begin{lscommand}
\ci{\bs*}
\end{lscommand}
\noindent additionally prohibits a pagebreak after the forced
linebreak. 

\begin{lscommand}
\ci{newpage}
\end{lscommand}
\noindent starts a new page. 

\begin{lscommand}
\ci{linebreak}\verb|[|\emph{n}\verb|]|,
\ci{nolinebreak}\verb|[|\emph{n}\verb|]|, 
\ci{pagebreak}\verb|[|\emph{n}\verb|]| and
\ci{nopagebreak}\verb|[|\emph{n}\verb|]|
\end{lscommand}
\noindent do what their names say. They enable the author to influence their
actions with the optional argument \emph{n}. It can be set to a number
between zero to four. By setting \emph{n} to a value below 4 you leave
\LaTeX{} the option of ignoring your command if the result would look very
bad. Do not confuse these ``break'' commands with the ``new'' commands. Even
when you give a ``break'' command, \LaTeX{} still tries to even out the
right border of the page and the total length of the page as described in
the next section. If you really want to start a ``new line'', then use the
corresponding command. Guess its name!

\LaTeX{} always tries to produce the best linebreaks possible. If it
cannot find a way to break the lines in a manner which meets its high
standards, it lets one line stick out on the right of the paragraph.
\LaTeX{} then complains (``\wi{overfull hbox}'') while processing the
input file. This happens most often when \LaTeX{} cannot find a
suitable place to hyphenate a word.\footnote{Although \LaTeX{} gives
  you a warning when that happens (Overfull hbox) and displays the
  offending line, such lines are not always easy to find. If you use
  the option \texttt{draft} in the \ci{documentclass} command, these
  lines will be marked with a thick black line on the right margin.}
You can instruct \LaTeX{} to lower its standards a little by giving
the \ci{sloppy} command. It prevents such over-long lines by
increasing the inter-word spacing --- even if the final output is not
optimal.  In this case a warning (``\wi{underfull hbox}'') is given to
the user.  In most such cases the result doesn't look very good. The
command \ci{fussy} brings \LaTeX{} back to its default behaviour.

\subsection{Hyphenation} \label{hyph}

\LaTeX{} hyphenates words whenever necessary. If the hyphenation
algorithm does not find the correct hyphenation points, you can
remedy the situation by using the following commands to tell \TeX{}
about the exception.

The command
\begin{lscommand}
\ci{hyphenation}\verb|{|\emph{word list}\verb|}|
\end{lscommand}
\noindent causes the words listed in the argument to be hyphenated only at
the points marked by ``\verb|-|''.  The argument of the command should only
contain words built from normal letters or rather signes which are regarded
as normal letters in the active context. The hyphenation hints are
stored for the language which is active when the hyphenation command
occurs. This means that if you place a hyphenation command into the preamble
of your document it will influence the english language hyphenation. If you
place the command after the \verb|\begin{document}| and you are using some
package for national language support like \pai{babel}, then the hyphenation
hints will be active in the language activated through \pai{babel}.

The example below will allow ``hyphenation'' to be hyphenated as well as
``Hyphenation'', and it prevents ``FORTRAN'', ``Fortran'' and ``fortran''
from being hyphenated at all.  No special characters or symbols are allowed
in the argument.

Example:
\begin{code}
\verb|\hyphenation{FORTRAN Hy-phen-a-tion}|
\end{code}

The command \ci{-} inserts a discretionary hyphen into a word. This
also becomes the only point hyphenation is allowed in this word. This
command is especially useful for words containing special characters
(e.g.{} accented characters), because \LaTeX{} does not automatically
hyphenate words containing special characters.
%\footnote{Unless you are using the new
%\wi{DC fonts}.}.

\begin{example}
I think this is: su\-per\-cal\-%
i\-frag\-i\-lis\-tic\-ex\-pi\-%
al\-i\-do\-cious
\end{example}

Several words can be kept together on one line with the command
\begin{lscommand}
\ci{mbox}\verb|{|\emph{text}\verb|}|
\end{lscommand}
\noindent It causes its argument to be kept together under all circumstances.

\begin{example}
My phone number will change soon.
It will be \mbox{0116 291 2319}.

The parameter 
\mbox{\emph{filename}} should 
contain the name of the file.
\end{example}

\ci{fbox} is similar to mbox, but in addition there will
be a visible box drawn around the content.


\section{Ready made Strings}

In some of the examples on the previous pages you have seen
some very simple \LaTeX{} commands for typesetting special
text strings:

\vspace{2ex}

\noindent
\begin{tabular}{@{}lll@{}}
Command&Example&Description\\
\hline
\ci{today} & \today   &  Current date in the current language\\
\ci{TeX} & \TeX       & The name of your favorite typesetter\\
\ci{LaTeX} & \LaTeX   & The name of the Game\\
\ci{LaTeXe} & \LaTeXe & The current incarnation of \LaTeX\\
\end{tabular}

\section{Special Characters and Symbols}
 
\subsection{Quotation Marks}

You should \emph{not} use the \verb|"| for \wi{quotation marks}
\index{""@\texttt{""}} as you would on a typewriter.  In publishing
there are special opening and closing quotation marks.  In \LaTeX{},
use two~\verb|`|s (grace accent) for opening quotation marks and
two~\verb|'|s (apostrophe) for closing quotation marks. For single
quotes you use just one of each.
\begin{example}
``Please press the `x' key.''
\end{example}
 
\subsection{Dashes and Hyphens}

\LaTeX{} knows four kinds of \wi{dash}es. You can access three of
them with different numbers of consecutive dashes. The fourth sign
is actually not a dash at all: It is the mathematical minus sign: \index{-}
\index{--} \index{---} \index{-@$-$} \index{mathematical!minus}

\begin{example}
daughter-in-law, X-rated\\
pages 13--67\\
yes---or no? \\
$0$, $1$ and $-1$
\end{example}
The names for these dashes are: 
`-' \wi{hyphen}, `--' \wi{en-dash}, `---' \wi{em-dash} and
`$-$' \wi{minus sign}.

\subsection{Tilde ($\sim$)}
\index{www}\index{URL}\index{tilde}
A character, often seen with web addresses is the tilde. To generate
this in \LaTeX{} you can use \verb|\~| but the result: \~{} is not really
what you want. Try this instead:

\begin{example}
http://www.rich.edu/\~{}bush \\
http://www.clever.edu/$\sim$demo
\end{example}  
 
\subsection{Degree Symbol ($\circ$)}

How to print a \wi{degree symbol} in \LaTeX{}?

\begin{example}
Its $-30\,^{\circ}\mathrm{C}$,
I will soon start to
super-conduct.
\end{example}

\subsection{Ellipsis ( \ldots )}

On a typewriter a \wi{comma} or a \wi{period} takes the same amount of
space as any other letter. In book printing these characters occupy
only a little space and are set very close to the preceding letter.
Therefore you cannot enter `\wi{ellipsis}' by just typing three
dots, as the spacing would be wrong. Besides that there is a special
command for these dots. It is called

\begin{lscommand}
\ci{ldots}
\end{lscommand}
\index{...@\ldots}


\begin{example}
Not like this ... but like this:\\
New York, Tokyo, Budapest, \ldots
\end{example}
 
\subsection{Ligatures}

Some letter combinations are typeset not just by setting the
different letters one after the other, but by actually using special
symbols.
\begin{code}
{\large ff fi fl ffi\ldots}\quad
instead of\quad {\large f{}f f{}i f{}l f{}f{}i \ldots}
\end{code}
These so-called \wi{ligature}s can be prohibited by inserting an \ci{mbox}\verb|{}|
between the two letters in question. This might be necessary with
words built from two words.

\begin{example}
Not shelfful\\
but shelf\mbox{}ful
\end{example}
 
\subsection{Accents and Special Characters}
 
\LaTeX{} supports the use of \wi{accent}s and \wi{special character}s
from many languages. Table~\ref{accents} shows all sorts of accents
being applied to the letter o. Naturally other letters work too.

To place an accent on top of an i or a j, its dots have to be
removed. This is accomplished by typing \verb|\i| and \verb|\j|.

\begin{example}
H\^otel, na\"\i ve, \'el\`eve,\\ 
sm\o rrebr\o d, !`Se\~norita!,\\
Sch\"onbrunner Schlo\ss{} 
Stra\ss e
\end{example}

\begin{table}[!hbp]
\caption{Accents and Special Characters.} \label{accents}
\begin{lined}{10cm}
\begin{tabular}{*4{cl}}
\A{\`o} & \A{\'o} & \A{\^o} & \A{\~o} \\
\A{\=o} & \A{\.o} & \A{\"o} & \B{\c}{c}\\[6pt]
\B{\u}{o} & \B{\v}{o} & \B{\H}{o} & \B{\c}{o} \\
\B{\d}{o} & \B{\b}{o} & \B{\t}{oo} \\[6pt]
\A{\oe}  &  \A{\OE} & \A{\ae} & \A{\AE} \\
\A{\aa} &  \A{\AA} \\[6pt]
\A{\o}  & \A{\O} & \A{\l} & \A{\L} \\
\A{\i}  & \A{\j} & !` & \verb|!`| & ?` & \verb|?`| 
\end{tabular}
\index{dotless \i{} and \j}\index{Scandinavian letters}
\index{ae@\ae}\index{umlaut}\index{grave}\index{acute}
\index{oe@\oe}

\bigskip
\end{lined}
\end{table}

\section{International Language Support}
\index{international} If you need to write documents in \wi{language}s
other than English, there are two areas where \LaTeX{} has to be
configured appropriately:

\begin{enumerate}
\item All automatically generated text strings\footnote{Table of
    Contents, List of Figures, \ldots} have to be adapted to the new
  language.  For many languages, these changes can be accomplished by
  using the \pai{babel} package by Johannes Braams.
\item \LaTeX{} needs to know the hyphenation rules for the new
  language. Getting hyphenation rules into \LaTeX{} is a bit more
  tricky. It means rebuilding the format file with different
  hyphenation patterns enabled. Your \guide{} should give more
  information on this.
\end{enumerate}

If your system is already configured appropriately, you can activate
the \pai{babel} package by adding the command
\begin{lscommand}
\ci{usepackage}\verb|[|\emph{language}\verb|]{babel}| 
\end{lscommand}
\noindent after the \verb|\documentclass| command. The \emph{language}s your
system supports should also be listed in the Local Guide. Babel will
automatically activate the apropriate hyphenation rules for the
language you choose. If your \LaTeX{} format does not support
hyphenation in the language of your choice, babel will still work but
it will disable hyphenation which has quite a negative effect on the
visual appearance of the typeset document.

For some languages, \textsf{babel} also specifies new commands which
simplify the input of special characters. The \wi{German} language, for
example, contains a lot of umlauts (\"a\"o\"u).  With \textsf{babel},
you can enter an \"o by typing \verb|"o| instead of~\verb|\"o|.

Some computer systems allow you to input special characters directly
from the keyboard. \LaTeX{} can handle such characters. Since the
December 1994 release of \LaTeXe{}, support for several input
encodings is included in the basic distribution of \LaTeXe. Check the
\pai{inputenc} package:
\begin{lscommand}
\ci{usepackage}\verb|[|\emph{encoding}\verb|]{inputenc}| 
\end{lscommand}

When using this package, you should consider
that other people might not be able to display your input files on
their computer, because they use a different encoding. For example,
the German umlaut \"a on a PC is encoded as 132, but on some Unix
systems using ISO-LATIN~1 it is encoded as 228. Therefore you should
use this feature with care. The following encodings may come handy,
depending on the type of system you are working on: 

\begin{center}
\begin{tabular}{l | r}
Operating system & encoding\\
\hline
Mac     &  \texttt{applemac} \\
Unix    &  \texttt{latin1} \\ 
Windows &  \texttt{ansinew} \\
OS/2    &  \texttt{cp850}
\end{tabular}
\end{center}

Font encoding is a different matter. It defines at which position inside
a \TeX-font each letter is stored. The original Computer Modern
\TeX{} font does only contain the 128 characters of the old 7-bit ASCII
character set. When accented characters are required, \TeX{} creates
them by combining a normal character with an accent. While the
resulting output looks perfect, this approach stops the automatic
hyphenation from working inside words containing accented characters.

Fortunately, most modern \TeX{} distributions contain a copy of the EC
fonts. These fonts look like the Computer Modern fonts, but contain
special characters for most of the accented characters used in
European languages. By using these fonts you can improve hyphenation
in non-English documents. The EC fonts are activated by including the
\pai{fontenc} package in the preamble of your document.

\begin{lscommand}
\ci{usepackage}\verb|[T1]{fontenc}| 
\end{lscommand}
\newpage

\subsection{Support for German}

Some hints for those creating \wi{German}\index{Deutsch}
documents with \LaTeX{}. You can load German language support with the
command:

\begin{lscommand}
\verb|\usepackage[german]{babel}|
\end{lscommand}

This enables German hyphenation, if you have configured your
LaTeX system accordingly. It also changes all automatic text into
German. Eg. ``Chapter'' becomes ``Kapitel''. Further a set of new commands
becomes available which allows you to write German input files more
quickly. Check out table \ref{german} for inspiration. 

\begin{table}[!hbp]
\caption{German Special Characters.} \label{german}
\begin{lined}{5cm}
\begin{tabular}{*2{cl}}
\verb|"a| & \"a \hspace*{1ex} & \verb|"s| & \ss \\[1ex]
\verb|"`| & \glqq & \verb|"'| & \grqq \\[1ex]
\verb|"<| & \flqq  & \verb|">| & \frqq \\[1ex]
\ci{dq} & " \\
\end{tabular}
\bigskip
\end{lined}
\end{table}


\section{The Space between Words}

To get a straight right margin in the output, \LaTeX{} inserts varying
amounts of space between the words. It inserts slightly more space at
the end of a sentence, as this makes the text more readable.  \LaTeX{}
assumes that sentences end with periods, question marks or exclamation
marks. If a period follows an uppercase letter, this is not taken as a
sentence ending, since periods after uppercase letters normally occur in
abbreviations.

Any exception from these assumptions has to be specified by the
author. A backslash in front of a space generates a space which will
not be enlarged. A tilde~`\verb|~|' character generates a space which cannot be
enlarged and which additionally prohibits a linebreak. The command
\verb|\@| in front of a period specifies that this period terminates a
sentence even when it follows an uppercase letter.
\cih{"@} \index{~@ \verb.~.} \index{tilde@tilde ( \verb.~.)}
\index{., space after}

\begin{example}
Mr.~Smith was happy to see her\\
cf.~Fig.~5\\
I like BASIC\@. What about you?
\end{example}

The additional space after periods can be disabled with the command
\begin{lscommand}
\ci{frenchspacing}
\end{lscommand}
\noindent which tells \LaTeX{} \emph{not} to insert more space after a
period than after ordinary character. This is very common in
non-English languages, except bibliographies. If you use
\ci{frenchspacing}, the command \verb|\@| is not necessary.
    
\section{Titles, Chapters, and Sections}

To help the reader find his or her way through your work, you should
divide it into chapters, sections, and subsections.  \LaTeX{} supports
this with special commands which take the section title as their
argument.  It is up to you to use them in the correct order.

The following sectioning commands are available for the
\texttt{article} class: \nopagebreak
\begin{code}
\ci{section}\verb|{...}           |\ci{paragraph}\verb|{...}|\\
\ci{subsection}\verb|{...}        |\ci{subparagraph}\verb|{...}|\\
\ci{subsubsection}\verb|{...}|
\end{code}

You can use two additional sectioning commands for the \texttt{report}
and the \texttt{book} class:
\begin{code}
\ci{part}\verb|{...}              |\ci{chapter}\verb|{...}|
\end{code}

As the \texttt{article} class does not know about chapters, it is quite easy
to add articles as chapters to a book.
The spacing between sections, the numbering and the font size of the
titles will be set automatically by \LaTeX. 

\pagebreak[3]
Two of the sectioning commands are a bit special: 
\begin{itemize}
\item The \ci{part} command does
  not influence the numbering sequence of chapters.  
\item The \ci{appendix} command does not take an argument. It just
  changes the chapter numbering to letters.\footnote{For the article
    style it changes the section numbering.}
\end{itemize}



\LaTeX{} creates a table of contents by taking the section headings
and page numbers from the last compile cycle of the document. The command 
\begin{lscommand} 
\ci{tableofcontents}
\end{lscommand} 
\noindent expands to a table of contents at the place where it
is issued. A new
document has to be compiled (``\LaTeX ed'') twice to get a
correct \wi{table of contents}. Sometimes it might be
necessary to compile the document a third time. \LaTeX{} will tell you
when this is necessary.

All sectioning commands listed above also exist as ``starred''
versions.  A ``starred'' version of a command is built by adding a
star \verb|*| after the command name.  They generate section headings
which do not show up in the table of contents and which are not
numbered. The command \verb|\section{Help}|, for example, would become
\verb|\section*{Help}|.

Normally the section headings show up in the table of contents exactly
as they are entered in the text. Sometimes this is not possible,
because the heading is too long to fit into the table of contents. The
entry for the table of contents can then be specified as an
optional argument in front of the actual heading.

\begin{code}
\verb|\chapter[Title for the table of contents]{A long|\\
\verb|    and especially boring title, shown in the text}|
\end{code} 

The \wi{title} of the whole document is generated by issuing a 
\begin{lscommand}
\ci{maketitle}
\end{lscommand}
\noindent command. The contents of the title have to be defined by the commands
\begin{lscommand}
\ci{title}\verb|{...}|, \ci{author}\verb|{...}| 
and optionally \ci{date}\verb|{...}| 
\end{lscommand}
\noindent before calling \verb|\maketitle|. In the argument of \ci{author}, you can
supply several names separated by \ci{and} commands. 

An example of some of the commands mentioned above can be found in
Figure~\ref{document} on page~\pageref{document}.

Apart from the sectioning commands explained above, \LaTeXe{}
introduced three additional commands for use with the \verb|book| class. 
They are useful for dividing your publication. The commands alter
chapter headings and page numbering to work as you would expect it in
a book:
\begin{description}
\item[\ci{frontmatter}] should be the very first command after
  \verb|\begin{document}|. It will switch page numbering to Roman
    numerals. It is common to use the starred sectioning commands (eg
    \verb|\chapter*{Preface}|) for
    frontmatter as this stopps \LaTeX{} from
    enumerating them.
\item[\ci{mainmatter}] comes after right befor the first chapter of
  the book. It turns on Arabic page numbering and restarts the page
  counter.
\item[\ci{appendix}] marks the start of additional material in your
  book. After this command chapters will be numbered with letters.
\item[\ci{backmatter}] should be inserted before the very last items
  in your book like the bibliography and the index. In the standard
  document classes, this has no visual effect.
\end{description}


\section{Cross References}

In books, reports and articles, there are often 
\wi{cross-references} to figures, tables and special segments of text.
\LaTeX{} provides the following commands for cross referencing
\begin{lscommand}
\ci{label}\verb|{|\emph{marker}\verb|}|, \ci{ref}\verb|{|\emph{marker}\verb|}| 
and \ci{pageref}\verb|{|\emph{marker}\verb|}|
\end{lscommand}
\noindent where \emph{marker} is an identifier chosen by the user. \LaTeX{}
replaces \verb|\ref| by the number of the section, subsection, figure,
table, or theorem after which the corresponding \verb|\label| command
was issued. \verb|\pageref| prints the page number of the
page where the \verb|\label| command occurred.\footnote{Note that these commands
  are not aware of what they refer to. \ci{label} just saves the last
  automatically generated number.} Just as the section titles, the
numbers from the previous run are used.

\begin{example}
A reference to this subsection
\label{sec:this} looks like:
``see section~\ref{sec:this} on 
page~\pageref{sec:this}.''
\end{example}
 
\section{Footnotes}
With the command
\begin{lscommand}
\ci{footnote}\verb|{|\emph{footnote text}\verb|}|
\end{lscommand}
\noindent a footnote is printed at the foot of the current page.  Footnotes
should always be put\footnote{``put'' is one of the most common
  English words.} after the word or sentence they refer to. Footnotes
referring to a sentence or part of it should therefore be put after
the comma or period.\footnote{Note, that footnotes are
  distracting the reader from the main body of your document. After all
  everybody reads the footnotes, we are a curious species. So why not
  just integrate everything you want to say into the body of the
  document.\footnotemark}
\footnotetext{A guidepost doesn't necessarily go where it's pointing to :-).}

\begin{example}
Footnotes\footnote{This is 
  a footnote.} are often used 
by people using \LaTeX.
\end{example}
 
\section{Emphasized Words}

If a text is typed using a typewriter, \texttt{important words are
  emphasized by \underline{underlining} them.}
\begin{lscommand}
\ci{underline}\verb|{|\emph{text}\verb|}|
\end{lscommand}
In printed books,
however, words are emphasized by typesetting them in an \emph{italic}
font.  \LaTeX{} provides the command
\begin{lscommand}
\ci{emph}\verb|{|\emph{text}\verb|}|
\end{lscommand}
\noindent to emphasize text.  What the command actually does with 
its argument depends on the context:

\begin{example}
\emph{If you use 
  emphasizing inside a piece
  of emphasized text, then 
  \LaTeX{} uses the
  \emph{normal} font for 
  emphasizing.}
\end{example}

Please note the difference between telling \LaTeX{} to
\emph{emphasize} something and telling it to use a different
\emph{font}:

\begin{example}
\textit{You can also
  \emph{emphasize} text if 
  it is set in italics,} 
\textsf{in a 
  \emph{sans-serif} font,}
\texttt{or in 
  \emph{typewriter} style.}
\end{example}

\section{Environments} \label{env}

% To typeset special purpose text, \LaTeX{} defines many different
% \wi{environment}s for all sorts of formatting:
\begin{lscommand}
\ci{begin}\verb|{|\emph{environment}\verb|}|\quad
   \emph{text}\quad
\ci{end}\verb|{|\emph{environment}\verb|}|
\end{lscommand}
\noindent Where \emph{environment} is the name of the environment. Environments can be
called several times within each other as long as the calling order is
maintained.
\begin{code}
\verb|\begin{aaa}...\begin{bbb}...\end{bbb}...\end{aaa}|
\end{code}

\noindent In the following sections all important environments are explained.

\subsection{Itemize, Enumerate, and Description}

The \ei{itemize} environment is suitable for simple lists, the
\ei{enumerate} environment for enumerated lists, and the
\ei{description} environment for descriptions.
\cih{item}

\begin{example}
\flushleft
\begin{enumerate}
\item You can mix the list
environments to your taste:
\begin{itemize}
\item But it might start to
look silly. 
\item[-] With a dash.
\end{itemize}
\item Therefore remember:
\begin{description}
\item[Stupid] things will not
become smart because they are
in a list.
\item[Smart] things, though, can be
presented beautifully in a list.
\end{description}
\end{enumerate}
\end{example}
 
\subsection{Flushleft, Flushright, and Center}

The environments \ei{flushleft} and \ei{flushright} generate
paragraphs which are either left- or \wi{right-aligned}. \index{left
  aligned} The \ei{center} environment generates centred text. If you
do not issue \ci{\bs} to specify linebreaks, \LaTeX{} will
automatically determine linebreaks.

\begin{example}
\begin{flushleft}
This text is\\ left-aligned. 
\LaTeX{} is not trying to make 
each line the same length.
\end{flushleft}
\end{example}

\begin{example}
\begin{flushright}
This text is right-\\aligned. 
\LaTeX{} is not trying to make
each line the same length.
\end{flushright}
\end{example}

\begin{example}
\begin{center}
At the centre\\of the earth
\end{center}
\end{example}

\subsection{Quote, Quotation, and Verse}

The \ei{quote} environment is useful for quotes, important phrases and
examples.

\begin{example}
A typographical rule of thumb
for the line length is:
\begin{quote}
On average, no line should
be longer than 66 characters.
\end{quote}
This is why \LaTeX{} pages have 
such large borders by default and
also why multicolumn print is
used in newspapers.
\end{example}

There are two similar environments: the \ei{quotation} and the
\ei{verse} environments. The \texttt{quotation} environment is useful
for longer quotes going over several paragraphs, because it does
indent paragraphs. The \texttt{verse} environment is useful for poems
where the line breaks are important. The lines are separated by
issuing a \ci{\bs} at the end of a line and a empty line after each
verse.


\begin{example}
I know only one English poem by 
heart. It is about Humpty Dumpty.
\begin{flushleft}
\begin{verse}
Humpty Dumpty sat on a wall:\\
Humpty Dumpty had a great fall.\\ 
All the King's horses and all
the King's men\\
Couldn't put Humpty together
again.
\end{verse}
\end{flushleft}
\end{example}

\subsection{Printing Verbatim}

Text which is enclosed between \verb|\begin{|\ei{verbatim}\verb|}| and
\verb|\end{verbatim}| will be directly printed, as if it was typed on a
typewriter, with all linebreaks and spaces, without any \LaTeX{}
command being executed.

Within a paragraph, similar behavior can be accessed with
\begin{lscommand}
\ci{verb}\verb|+|\emph{text}\verb|+|
\end{lscommand}
\noindent The \verb|+| is just an example of a delimiter character. You can use any
character except letters, \verb|*| or space. Many \LaTeX{} examples in this
booklet are typeset with this command.

\begin{example}
The \verb|\ldots| command \ldots

\begin{verbatim}
10 PRINT "HELLO WORLD ";
20 GOTO 10
\end{verbatim}
\end{example}

\begin{example}
\begin{verbatim*}
the starred version of
the      verbatim   
environment emphasizes
the spaces   in the text
\end{verbatim*}
\end{example}

The \ci{verb} command can be used in a similar fashion with a star:

\begin{example}
\verb*|like   this :-) |
\end{example}

The \texttt{verbatim} environment and the \verb|\verb| command may not be used
within parameters of other commands.


\subsection{Tabular}

The \ei{tabular} environment can be used to typeset beautiful
\wi{table}s with optional horizontal and vertical lines. \LaTeX{}
determines the width of the columns automatically.

The \emph{table spec} argument of the 
\begin{lscommand}
\verb|\begin{tabular}{|\emph{table spec}\verb|}|
\end{lscommand} 
\noindent command defines the format of the table. Use an \texttt{l} for a column of
left-aligned text, \texttt{r} for right-aligned text, and \texttt{c} for
centred text; \verb|p{|\emph{width}\verb|}| for a column containing justified
text with linebreaks, and \verb.|. for a vertical line.
 
Within a \texttt{tabular} environment, \verb|&| jumps to the next column, \ci{\bs}
starts a new line and \ci{hline} inserts a horizontal line.
You can add partial Lines by using the \ci{cline}\verb|{j-i}| whereby
j and i are the column numbers the line should extend over.

\index{"|@ \verb."|.}

\begin{example}
\begin{tabular}{|r|l|}
\hline
7C0 & hexadecimal \\
3700 & octal \\ \cline{2-2}
11111000000 & binary \\
\hline \hline
1984 & decimal \\
\hline
\end{tabular}
\end{example}

\begin{example}
\begin{tabular}{|p{4.7cm}|}
\hline
Welcome to Boxy's paragraph.
We sincerely hope you'll 
all enjoy the show.\\
\hline 
\end{tabular}
\end{example}

The column separator can be specified with the \verb|@{...}|
construct. This command kills the inter-column space and replaces it
with whatever is between the curly braces.  One common use for
this command is explained below in the decimal alignment problem.
Another possible application is to suppress leading space in a table with
\verb|@{}|.

\begin{example}
\begin{tabular}{@{} l @{}}
\hline 
no leading space\\
\hline
\end{tabular}
\end{example}

\begin{example}
\begin{tabular}{l}
\hline
leading space left and right\\
\hline
\end{tabular}
\end{example}

%
% This part by Mike Ressler
%

\index{decimal alignment} Since there is no built-in way to align
numeric columns to a decimal point,\footnote{If the `tools' bundle is
  installed on your system, have a look at the \pai{dcolumn} package.}
we can ``cheat'' and do it by using two columns: a right-aligned
integer and a left-aligned fraction. The \verb|@{.}| command in the
\verb|\begin{tabular}| line replaces the normal inter-column spacing with
just a ``.'', giving the appearance of a single,
decimal-point-justified column.  Don't forget to replace the decimal
point in your numbers with a column separator (\verb|&|)! A column label
can be placed above our numeric ``column'' by using the
\ci{multicolumn} command.
 
\begin{example}
\begin{tabular}{c r @{.} l}
Pi expression       &
\multicolumn{2}{c}{Value} \\
\hline
$\pi$               & 3&1416  \\
$\pi^{\pi}$         & 36&46   \\
$(\pi^{\pi})^{\pi}$ & 80662&7 \\
\end{tabular}
\end{example}

\begin{example}
\begin{tabular}{|c|c|}
\hline
\multicolumn{2}{|c|}{Ene} \\
\hline
Mene & Muh! \\
\hline
\end{tabular}
\end{example}

Material typeset with the tabular environment always stays together on
one page. If you want to typeset long tables you might want to have a
look at the \pai{supertabular} and the \pai{longtabular} environments.

\section{Floating Bodies}
Today most publications contain a lot of figures and tables. These
elements need special treatment, because they cannot be broken across
pages.  One method would be to start a new page every time a figure or
a table is too large to fit on the present page. This approach would
leave pages partially empty, which looks very bad.

The solution to this problem is to `float' any figure or table which
does not fit on the current page to a later page, while filling the
current page with body text. \LaTeX{} offers two environments for
\wi{floating bodies}; one for tables and  one for figures.  To
take full advantage of these two environments it is important to
understand approximately how \LaTeX{} handles floats internally.
Otherwise floats may become a major source of frustration, because
\LaTeX{} never puts them where you want them to be.

\bigskip
Let's first have a look at the commands \LaTeX{} supplies
for floats:

Any material enclosed in a \ei{figure} or \ei{table} environment will
be treated as floating matter. Both float environments support an optional
parameter
\begin{lscommand}
\verb|\begin{figure}[|\emph{placement specifier}\verb|]| or
\verb|\begin{table}[|\emph{placement specifier}\verb|]|
\end{lscommand}
\noindent called the \emph{placement specifier}. This parameter
is used to tell \LaTeX{} about the locations to which the float
is allowed to be moved.  A \emph{placement specifier} is constructed by building a string
of \emph{float-placing permissions}. See Table~\ref{tab:permiss}.

\begin{table}[!bp]
\caption{Float Placing Permissions.}\label{tab:permiss}
\noindent \begin{minipage}{\textwidth}
\medskip
\begin{center}
\begin{tabular}{@{}cp{10cm}@{}}
Spec&Permission to place the float \ldots\\
\hline
\rule{0pt}{1.05em}\texttt{h} & \emph{here} at the very place in the text
  where it occurred.  This is useful mainly for small floats.\\[0.3ex]
\texttt{t} & at the \emph{top} of a page\\[0.3ex]
\texttt{b} & at the \emph{bottom} of a page\\[0.3ex]
\texttt{p} & on a special \emph{page} containing only floats.\\[0.3ex]
\texttt{!} & without considering most of the  internal parameters\footnote{Such as the
    maximum number of floats allowed  on one page.} which could stop this
  float from being placed.
\end{tabular}
\end{center}
\end{minipage}
\end{table}
Note: The \texttt{0pt} and \texttt{1.05em} are \TeX{} units. Read more
on this in table \ref{units} on page \pageref{units}.

\pagebreak[3]
A table could be started with the following line e.g.{}
\begin{code}
\verb|\begin{table}[!hbp]|
\end{code}
\noindent The \wi{placement specifier} \verb|[!hbp]| allows \LaTeX{} to 
place the table right here (\texttt{h}) or at the bottom (\texttt{b}) 
of some page
or on a special floats page (\texttt{p}), and all this even if it does not
look that good (\texttt{!}). If no placement specifier is given, the standard
classes assume \verb|[tbp]|.

\LaTeX{} will place every float it encounters, according to the
placement specifier supplied by the author. If a float cannot be
placed on the current page it is deferred either to the
\emph{figures} or the \emph{tables} queue\footnote{These are fifo -
  `first in first out' queues!}.  When a new page is started,
\LaTeX{} first checks if it is possible to fill a special `float'
page with floats from the queues. If this is not possible, the first
float on each queue is treated as if it had just occurred in the
text: \LaTeX{} tries again to place it according to its
respective placement specifiers (except `h' which is no longer
possible).  Any new floats occurring in the text get placed into the
appropriate queues. \LaTeX{} strictly maintains the original order of
appearance for each type of float. That's why a figure which cannot
be placed pushes all further figures to the end of the document.
Therefore:

\begin{quote}
If \LaTeX{} is not placing the floats as you expected,
it is often only one float jamming one of the two float queues.
\end{quote}                 

While it is possible to give LaTeX single-location placement
specifiers, this causes problems.  If the float does not fit in the
location specified, then it becomes stuck, blocking subsequent floats.
In particular, you should never ever use the [h] option, it is so bad
that in more recent versions of LaTeX, it is automatically replaced by
[ht].

\bigskip
\noindent Having explained the difficult bit, there are some more things to
mention about the \ei{table} and \ei{figure} environments.
With the 

\begin{lscommand}
\ci{caption}\verb|{|\emph{caption text}\verb|}|
\end{lscommand}

\noindent command, you can define a caption for the float. A running number and
the string ``Figure'' or ``Table'' will be added by \LaTeX.

The two commands

\begin{lscommand}
\ci{listoffigures} and \ci{listoftables} 
\end{lscommand}

\noindent operate analogously to the \verb|\tableofcontents| command,
printing a list of figures or tables, respectively.  In these lists,
the whole caption will be repeated. If you tend to use long captions,
you must have a shorter version of the caption going into the lists.
This is accomplished by entering the short version in brackets after
the \verb|\caption| command.
\begin{code}
\verb|\caption[Short]{LLLLLoooooonnnnnggggg}| 
\end{code}

With \verb|\label| and \verb|\ref|, you can create a reference to a float 
within your text.

The following example draws a square and inserts it into the
document. You could use this if you wanted to reserve space for images
you are going to paste into the finished document.

\begin{code}
\begin{verbatim}
Figure~\ref{white} is an example of Pop-Art.
\begin{figure}[!hbp]
\makebox[\textwidth]{\framebox[5cm]{\rule{0pt}{5cm}}}
\caption{Five by Five in Centimetres.} \label{white}
\end{figure}
\end{verbatim}
\end{code}

\noindent In the example above, 
\LaTeX{} will try \emph{really hard}~(\texttt{!}) to place the figure
right \emph{here}~(\texttt{h}).\footnote{assuming the figure queue is
  empty.} If this is not possible, it tries to place the figure at the
\emph{bottom}~(\texttt{b}) of the page.  Failing to place the figure
on the current page, it determines whether it is possible to create a float
page containing this figure and maybe some tables from the tables
queue. If there is not enough material for a special float page,
\LaTeX{} starts a new page, and once more treats the figure as if it
had just occurred in the text.

Under certain circumstances it might be necessary to use the 

\begin{lscommand}
\ci{clearpage} or even the \ci{cleardoublepage} 
\end{lscommand}

\noindent command. It orders \LaTeX{} to immediately place all 
floats remaining in the queues and then start a new
page. \ci{cleardoublepage} even goes to a new righthand page.

You will learn how to include PostScript
drawings into your \LaTeXe{} documents later in this introduction.

\section{Protecting fragile commands}

Text given as arguments of commands like \ci{caption} or \ci{section} may
show up more than once in the document (e.g. in the table of contents as
well as in the body of the document). Some commands fail when used in the
argument of \ci{section}-like commands. These are called \wi{fragile commands}.
Fragile commands are for example \ci{footnote} or \ci{phantom}. What these
fragile commands need to work, is protection (don't we all?). You can
protect them by putting the \ci{protect} command in front of them.

\ci{protect} only refers to the command which follows right behind, not even
to its arguments. In most cases a superfluous \ci{protect} won't hurt.

\begin{code}
\verb|\section{I am considerate|\\
\verb|      \protect\footnote{and protect my footnotes}}|
\end{code}


%%% Local Variables: 
%%% mode: latex
%%% TeX-master: "lshort"
%%% En


%%%%%%%%%%%%%%%%%%%%%%%%%%%%%%%%%%%%%%%%%%%%%%%%%%%%%%%%%%%%%%%%
% Contents: Math typesetting with LaTeX
% $Id: math.tex 14 2002-05-26 03:44:42Z marcilr $
%%%%%%%%%%%%%%%%%%%%%%%%%%%%%%%%%%%%%%%%%%%%%%%%%%%%%%%%%%%%%%%%%
 
\chapter{Typesetting Mathematical Formulae}

\begin{intro}
  Now you are ready! In this chapter, we will attack the main strength
  of \TeX{}: mathematical typesetting. But be warned, this chapter
  only scratches the surface. While the things explained here are
  sufficient for many people, don't despair if you can't find a
  solution to your mathematical typesetting needs here. It is highly likely
  that your problem is addressed in \AmS-\LaTeX{}%
  \footnote{\texttt{CTAN:/tex-archive/macros/latex/required/amslatex}}
  or some other package.
\end{intro}
  
\section{General}

\LaTeX{} has a special mode for typesetting \wi{mathematics}.
Mathematical text within a paragraph is entered between \ci{(}
and \ci{)}, \index{$@\texttt{\$}} %$
between \texttt{\$} and \texttt{\$} or between %}
\verb|\begin{|\ei{math}\verb|}| and \verb|\end{math}|.\index{formulae}
\begin{example}
Add $a$ squared and $b$ squared 
to get $c$ squared. Or, using 
a more mathematical approach:
$c^{2}=a^{2}+b^{2}$
\end{example}
\begin{example}
\TeX{} is pronounced as 
$\tau\epsilon\chi$.\\[6pt]
100~m$^{3}$ of water\\[6pt]
This comes from my $\heartsuit$
\end{example}

It is preferable to \emph{display} larger mathematical equations or formulae,
rather than to typeset them on separate lines. This means you enclose them
in \ci{[} and \ci{]} or between
\verb|\begin{|\ei{displaymath}\verb|}| and
  \verb|\end{displaymath}|.  This produces formulae which are not
numbered. If you want \LaTeX{} to number them, you can use the
\ei{equation} environment.
\begin{example}
Add $a$ squared and $b$ squared 
to get $c$ squared. Or, using 
a more mathematical approach:
\begin{displaymath}
c^{2}=a^{2}+b^{2}
\end{displaymath}
And just one more line.
\end{example}

You can reference an equation with \ci{label} and \ci{ref}
\begin{example}
\begin{equation} \label{eq:eps}
\epsilon > 0
\end{equation}
From (\ref{eq:eps}), we gather 
\ldots
\end{example}

Note that expressions will be typeset in a different style if displayed:
\begin{example}
$\lim_{n \to \infty} 
\sum_{k=1}^n \frac{1}{k^2} 
= \frac{\pi^2}{6}$
\end{example}
\begin{example}
\begin{displaymath}
\lim_{n \to \infty} 
\sum_{k=1}^n \frac{1}{k^2} 
= \frac{\pi^2}{6}
\end{displaymath}
\end{example}



There are differences between \emph{math mode} and \emph{text mode}. For
example in \emph{math mode}: 

\begin{enumerate}

\item Most spaces and linebreaks do not have any significance, as all spaces
either are derived logically from the mathematical expressions or
have to be specified using special commands such as \ci{,}, \ci{quad} or
\ci{qquad}.
 
\item Empty lines are not allowed. Only one paragraph per formula.

\item Each letter is considered to be the name of a variable and will be
typeset as such. If you want to typeset normal text within a formula
(normal upright font and normal spacing) then you have to enter the
text using the \verb|\textrm{...}| commands.
\end{enumerate}
\begin{example}
\begin{equation}
\forall x \in \mathbf{R}:
\qquad x^{2} \geq 0
\end{equation}
\end{example}
\begin{example}
\begin{equation}
x^{2} \geq 0\qquad
\textrm{for all }x\in\mathbf{R}
\end{equation}
\end{example}
 

%
% Add AMSSYB Package ... Blackboard bold .... R for realnumbers
%
Mathematicians can be very fussy about which symbols are used:
it would be conventional here to use `\wi{blackboard bold}',
\index{bold symbols} which is obtained using \ci{mathbb} from the
package \pai{amsfonts} or \pai{amssymb}.
\ifx\mathbb\undefined\else
The last example becomes
\begin{example}
\begin{displaymath}
x^{2} \geq 0\qquad
\textrm{for all }x\in\mathbb{R}
\end{displaymath}
\end{example}
\fi

\section{Grouping in Math Mode}

Most math mode commands act only on the next character. So if you
want a command to affect several characters, you have to group them
together using curly braces: \verb|{...}|.
\begin{example}
\begin{equation}
a^x+y \neq a^{x+y}
\end{equation}
\end{example}
 
\section{Building Blocks of a Mathematical Formula}

In this section, the most important commands used in mathematical
typesetting will be described. Take a look at section~\ref{symbols} on
page~\pageref{symbols} for a detailed list of commands for typesetting
mathematical symbols.

\textbf{Lowercase \wi{Greek letters}} are entered as \verb|\alpha|,
 \verb|\beta|, \verb|\gamma|, \ldots, uppercase letters
are entered as \verb|\Gamma|, \verb|\Delta|, \ldots\footnote{There is no
  uppercase Alpha defined in \LaTeXe{} because it looks the same as a
  normal roman A. Once the new math coding is done, things will
  change.} 
\begin{example}
$\lambda,\xi,\pi,\mu,\Phi,\Omega$
\end{example}
\enlargethispage{\baselineskip}
\pagebreak[4]

\textbf{Exponents and Subscripts} can be specified using\index{exponent}\index{subscript}
the \verb|^|\index{^@\verb"|^"|} and the \verb|_|\index{_@\verb"|_"|} character.
\begin{example}
$a_{1}$ \qquad $x^{2}$ \qquad
$e^{-\alpha t}$ \qquad
$a^{3}_{ij}$\\
$e^{x^2} \neq {e^x}^2$
\end{example}

The \textbf{\wi{square root}} is entered as \ci{sqrt}, the
$n^\mathrm{th}$ root is generated with \verb|\sqrt[|$n$\verb|]|. The size of
the root sign is determined automatically by \LaTeX. If just the sign
is needed, use \verb|\surd|.
\begin{example}
$\sqrt{x}$ \qquad 
$\sqrt{ x^{2}+\sqrt{y} }$ 
\qquad $\sqrt[3]{2}$\\[3pt]
$\surd[x^2 + y^2]$
\end{example}

The commands \ci{overline} and \ci{underline} create
\textbf{horizontal lines} directly over or under an expression.
\index{horizontal!line}
\begin{example}
$\overline{m+n}$
\end{example}

The commands \ci{overbrace} and \ci{underbrace} create
long \textbf{horizontal braces} over or under an expression.
\index{horizontal!brace}
\begin{example}
$\underbrace{ a+b+\cdots+z }_{26}$
\end{example}

\index{mathematical!accents} To add mathematical accents such as small
arrows or \wi{tilde} signs to variables, you can use the commands
given in Table~\ref{mathacc} on page \pageref{mathacc}.  Wide hats and
tildes covering several characters are generated with \ci{widetilde}
and \ci{widehat}.  The \verb|'|\index{'@\verb"|'"|} symbol gives a
\wi{prime}.
% a dash is --
\begin{example}
\begin{displaymath}
y=x^{2}\qquad y'=2x\qquad y''=2
\end{displaymath}
\end{example}

\textbf{Vectors}\index{vectors} often are specified by adding small
\wi{arrow symbols} on top of a variable. This is done with the
\ci{vec} command. The two commands \ci{overrightarrow} and
\ci{overleftarrow} are useful to denote the vector from $A$ to $B$.
\begin{example}
\begin{displaymath}
\vec a\quad\overrightarrow{AB}
\end{displaymath}
\end{example}

Usually you don't typeset an explicit dot sign to indicate
the multiplication operation. However sometimes it is written
to help the reader's eyes in grouping a formula.
Then you should use \ci{cdot}
\begin{example}
\begin{displaymath}
v = {\sigma}_1 \cdot {\sigma}_2
    {\tau}_1 \cdot {\tau}_2
\end{displaymath}
\end{example}


Names of log-like functions are often typeset in an upright
font and not in italic like variables. Therefore \LaTeX{} supplies the
following commands to typeset the most important function names:
\index{mathematical!functions}

\begin{tabular}{lllllll}
\ci{arccos} &  \ci{cos}  &  \ci{csc} &  \ci{exp} &  \ci{ker}    & \ci{limsup} & \ci{min} \\
\ci{arcsin} &  \ci{cosh} &  \ci{deg} &  \ci{gcd} &  \ci{lg}     & \ci{ln}     & \ci{Pr}  \\
\ci{arctan} &  \ci{cot}  &  \ci{det} &  \ci{hom} &  \ci{lim}    & \ci{log}    & \ci{sec} \\
\ci{arg}    &  \ci{coth} &  \ci{dim} &  \ci{inf} &  \ci{liminf} & \ci{max}    & \ci{sin} \\
\ci{sinh} & \ci{sup} & \ci{tan} & \ci{tanh}\\
\end{tabular}

\begin{example}
\[\lim_{x \rightarrow 0}
\frac{\sin x}{x}=1\]
\end{example}

For the \wi{modulo function}, there are two commands: \ci{bmod} for the
binary operator ``$a \bmod b$'' and \ci{pmod}
for expressions
such as ``$x\equiv a \pmod{b}$.''

A built-up \textbf{\wi{fraction}} is typeset with the
\ci{frac}\verb|{...}{...}| command.
Often the slashed form $1/2$ is preferable, because it looks better
for small amounts of `fraction material.'
\begin{example}
$1\frac{1}{2}$~hours
\begin{displaymath}
\frac{ x^{2} }{ k+1 }\qquad
x^{ \frac{2}{k+1} }\qquad
x^{ 1/2 }
\end{displaymath}
\end{example}

To typeset binomial coefficients or similar structures, you can use
either the command \verb|{... |\ci{choose}\verb| ...}| or 
\verb|{... |\ci{atop}\verb| ...}|. The second command produces the
same output as the first one, but without braces.
\footnote{Note that the usage of these old-style commands is expressly forbidden
by the \pai{amsmath} package. They are replaced by
\ci{binom} and \ci{genfrac}. The latter is a superset of all related
construct, e.g. you may get a similar construct to \ci{atop}
by \texttt{$\backslash$newcommand\{$\backslash$newatop\}[2]\{$\backslash$genfrac\{\}\{\}\{0pt\}\{1\}\{\#1\}\{\#2\}\}}.}

\begin{example}
\begin{displaymath}
{n \choose k}\qquad {x \atop y+2}
\end{displaymath}
\end{example}

For binary relations it may be useful to stack symbols over each other.
\ci{stackrel} puts the symbol given
in the first argument in superscript-like size over the second which
is set in its usual position.
\begin{example}
\begin{displaymath}
\int f_N(x) \stackrel{!}{=} 1
\end{displaymath}
\end{example}

The \textbf{\wi{integral operator}} is generated with \ci{int}, the
\textbf{\wi{sum operator}} with \ci{sum} and the \textbf{\wi{product operator}}
with \ci{prod}. The upper and lower limits are specified with~\verb|^|
and~\verb|_| like subscripts and superscripts.
\footnote{\AmS-\LaTeX{} in addition has multiline super-/subscripts}
\begin{example}
\begin{displaymath}
\sum_{i=1}^{n} \qquad
\int_{0}^{\frac{\pi}{2}} \qquad
\prod_\epsilon
\end{displaymath}
\end{example}

For \textbf{\wi{braces}} and other \wi{delimiters}, there exist all
types of symbols in \TeX{} (e.g.~$[\;\langle\;\|\;\updownarrow$).
Round and square braces can be entered with the corresponding keys,
curly braces with \verb|\{|, all other delimiters are generated with
special commands (e.g.~\verb|\updownarrow|). For a list of all
delimiters available, check table~\ref{tab:delimiters} on page
\pageref{tab:delimiters}.
\begin{example}
\begin{displaymath}
{a,b,c}\neq\{a,b,c\}
\end{displaymath}
\end{example}

If you put the command \ci{left} in front of an opening delimiter or
\ci{right} in front of a closing delimiter, \TeX{} will automatically
determine the correct size of the delimiter. Note that you must close
every \ci{left} with a corresponding \ci{right}, and that the size is
determined correctly only if both are typeset on the same line. If you
don't want anything on the right, use the invisible `\ci{right.}'!
\begin{example}
\begin{displaymath}
1 + \left( \frac{1}{ 1-x^{2} }
    \right) ^3
\end{displaymath}
\end{example}

\pagebreak[4]
In some cases it is necessary to specify the correct size of a
mathematical delimiter\index{mathematical!delimiter} by hand,
which can be done using the commands \ci{big}, \ci{Big}, \ci{bigg} and
\ci{Bigg} as prefixes to most delimiter commands.\footnote{These
  commands do not work as expected if a size changing command has been
  used, or the \texttt{11pt} or \texttt{12pt} option has been
  specified.  Use the \pai{exscale} or \pai{amsmath} packages to
  correct this behaviour.}
\begin{example}
$\Big( (x+1) (x-1) \Big) ^{2}$\\
$\big(\Big(\bigg(\Bigg($\quad
$\big\}\Big\}\bigg\}\Bigg\}$\quad
$\big\|\Big\|\bigg\|\Bigg\|$
\end{example}

To enter \textbf{\wi{three dots}} into a formula, you can use several
commands. \ci{ldots} typesets the dots on the baseline, \ci{cdots}
sets them centred. Besides that, there are the commands \ci{vdots} for
vertical and \ci{ddots} for \wi{diagonal dots}.\index{vertical
  dots}\index{horizontal!dots} You can find another example in section~\ref{sec:vert}.
\begin{example}
\begin{displaymath}
x_{1},\ldots,x_{n} \qquad
x_{1}+\cdots+x_{n}
\end{displaymath}
\end{example}
 
\section{Math Spacing}

\index{math spacing} If the spaces within formulae chosen by \TeX{}
are not satisfactory, they can be adjusted by inserting special
spacing commands. There are some commands for small spaces: \ci{,} for
$\frac{3}{18}\:\textrm{quad}$ (\demowidth{0.166em}), \ci{:} for $\frac{4}{18}\:
\textrm{quad}$ (\demowidth{0.222em}) and \ci{;} for $\frac{5}{18}\:
\textrm{quad}$ (\demowidth{0.277em}).  The escaped space character
\verb*.\ . generates a medium sized space and \ci{quad}
(\demowidth{1em}) and \ci{qquad} (\demowidth{2em}) produce large
spaces. The size of a \ci{quad} corresponds to the width of the
character `M' of the current font.  The \verb|\!|\cih{"!} command produces a
negative space of $-\frac{3}{18}\:\textrm{quad}$ (\demowidth{0.166em}).
\begin{example}
\newcommand{\ud}{\mathrm{d}}
\begin{displaymath}
\int\!\!\!\int_{D} g(x,y)
  \, \ud x\, \ud y 
\end{displaymath}
instead of 
\begin{displaymath}
\int\int_{D} g(x,y)\ud x \ud y
\end{displaymath}
\end{example}
Note that `d' in the differential is conventionally set in roman.

\AmS-\LaTeX{} provides another way for finetuning
the spacing between multiple integral signs,
namely the \ci{iint}, \ci{iiint}, \ci{iiiint}, and \ci{idotsint} commands.
With the \pai{amsmath} package loaded, the above example can be
typeset this way:
\begin{example}
\newcommand{\ud}{\mathrm{d}}
\begin{displaymath}
\iint_{D} \, \ud x \, \ud y
\end{displaymath}
\end{example}

See the electronic document testmath.tex (distributed with
\AmS-\LaTeX) or Chapter 8 of ``The LaTeX Companion'' for further details.

\section{Vertically Aligned Material}
\label{sec:vert}

To typeset \textbf{arrays}, use the \ei{array} environment. It works
somewhat similar to the \texttt{tabular} environment. The \verb|\\| command is
used to break the lines.
\begin{example}
\begin{displaymath}
\mathbf{X} =
\left( \begin{array}{ccc}
x_{11} & x_{12} & \ldots \\
x_{21} & x_{22} & \ldots \\
\vdots & \vdots & \ddots
\end{array} \right)
\end{displaymath}
\end{example}

The \ei{array} environment can also be used to typeset expressions which have one
big delimiter by using a ``\verb|.|'' as an invisible \ci{right} 
delimiter:
\begin{example}
\begin{displaymath}
y = \left\{ \begin{array}{ll}
 a & \textrm{if $d>c$}\\
 b+x & \textrm{in the morning}\\
 l & \textrm{all day long}
  \end{array} \right.
\end{displaymath}
\end{example}

As within the \verb|tabular| environment you can also
draw lines in the \ei{array} environent, e.g. separating the entries of
a matrix:
\begin{example}
\begin{displaymath}
\left(\begin{array}{c|c}
 1 & 2 \\
\hline
3 & 4
\end{array}\right)
\end{displaymath}
\end{example}



For formulae running over several lines or for \wi{equation system}s,
you can use the environments \ei{eqnarray}, and \verb|eqnarray*|
instead of \texttt{equation}. In \texttt{eqnarray} each line gets an
equation number. The \verb|eqnarray*| does not number anything.

The \texttt{eqnarray} and the \verb|eqnarray*| environments work like
a 3-column table of the form \verb|{rcl}|, where the middle column can
be used for the equal sign or the not-equal sign. Or any other sign
you see fit. The \verb|\\| command breaks the lines.
\begin{example}
\begin{eqnarray}
f(x) & = & \cos x     \\
f'(x) & = & -\sin x   \\
\int_{0}^{x} f(y)dy &
 = & \sin x
\end{eqnarray}
\end{example}
Notice that the space on either side of the 
the equal signs is rather large. It can be reduced by setting
\verb|\setlength\arraycolsep{2pt}|, as in the next example.

\index{long equations} \textbf{Long equations} will not be
automatically divided into neat bits.  The author has to specify
where to break them and how much to indent. The following two methods
are the most common ones used to achieve this.
\begin{example}
{\setlength\arraycolsep{2pt}
\begin{eqnarray}
\sin x & = & x -\frac{x^{3}}{3!}
     +\frac{x^{5}}{5!}-{}
                    \nonumber\\
 & & {}-\frac{x^{7}}{7!}+{}\cdots
\end{eqnarray}}
\end{example}
\begin{example}
\begin{eqnarray}
\lefteqn{ \cos x = 1
     -\frac{x^{2}}{2!} +{} }
                    \nonumber\\
 & & {}+\frac{x^{4}}{4!}
     -\frac{x^{6}}{6!}+{}\cdots
\end{eqnarray}
\end{example}

%\enlargethispage{\baselineskip}
\noindent The \ci{nonumber} command causes \LaTeX{} to not generate a number for
this equation.

It can be difficult to get vertically aligned equations to look right
with these methods; the package \pai{amsmath} provides a more
powerful set of alternatives. (see \verb|split| and \verb|align| environments).

\section{Phantom}

We can't see phantoms, but they still occupy some space in the minds of a
lot of people. \LaTeX{} is no different. We can use this for
some interesting spacing tricks.

When vertically aligning text using \verb|^| and \verb|_| \LaTeX{} sometimes
is just a little bit too helpful. Using the \ci{phantom} command you can
reserve space for characters which do not show up in the final output. Best
is to look at the following examples.
\begin{example}
\begin{displaymath}
{}^{12}_{\phantom{1}6}\textrm{C}
\qquad \textrm{versus} \qquad
{}^{12}_{6}\textrm{C}
\end{displaymath}
\end{example}
\begin{example}
\begin{displaymath} 
\Gamma_{ij}^{\phantom{ij}k}
\qquad \textrm{versus} \qquad
\Gamma_{ij}^{k}
\end{displaymath}  
\end{example}

\section{Math Font Size}

\index{math font size} In math mode, \TeX{} selects the font size
according to the context. Superscripts, for example, get typeset in a
smaller font. If you want to typeset part of an equation in roman,
don't use the \verb|\textrm| command, because the font size switching
mechanism will not work, as \verb|\textrm| temporarily escapes to text
mode. Use \verb|\mathrm| instead to keep the size switching mechanism
active. But pay attention, \ci{mathrm} will only work well on short
items. Spaces are still not active and accented characters do not
work.\footnote{The \AmS-\LaTeX{} package makes the \ci{textrm} command
  work with size changing.}
\begin{example}
\begin{equation}
2^{\textrm{nd}} \quad 
2^{\mathrm{nd}}
\end{equation}
\end{example}

Nevertheless, sometimes you need to tell \LaTeX{} the correct font
size. In math mode, the fontsize is set with the four commands:
\begin{flushleft}
\ci{displaystyle}~($\displaystyle 123$),
 \ci{textstyle}~($\textstyle 123$), 
\ci{scriptstyle}~($\scriptstyle 123$) and
\ci{scriptscriptstyle}~($\scriptscriptstyle 123$).
\end{flushleft}

Changing styles also affects the way limits are displayed.
\begin{example}
\begin{displaymath}
\mathop{\mathrm{corr}}(X,Y)= 
 \frac{\displaystyle 
   \sum_{i=1}^n(x_i-\overline x)
   (y_i-\overline y)} 
  {\displaystyle\biggl[
 \sum_{i=1}^n(x_i-\overline x)^2
\sum_{i=1}^n(y_i-\overline y)^2
\biggr]^{1/2}}
\end{displaymath}    
\end{example}
% This is not a math accent, and no maths book would be set this way.
% mathop gets the spacing right.

\noindent This is one of those examples in which we need larger
brackets than the standard \verb|\left[  \right]| provides.


\section{Theorems, Laws, \ldots}

When writing mathematical documents, you probably need a way to
typeset ``Lemmas'', ``Definitions'', ``Axioms'' and similar
structures. \LaTeX{} supports this with the command
\begin{lscommand}
\ci{newtheorem}\verb|{|\emph{name}\verb|}[|\emph{counter}\verb|]{|%
         \emph{text}\verb|}[|\emph{section}\verb|]|
\end{lscommand}
The \emph{name} argument, is a short keyword used to identify the
``theorem''. With the \emph{text} argument, you define the actual name
of the ``theorem'' which will be printed in the final document.

The arguments in square brackets are optional. They are both used to
specify the numbering used on the ``theorem''. With the \emph{counter}
argument you can specify the \emph{name} of a previously declared
``theorem''. The new ``theorem'' will then be numbered in the same
sequence.  The \emph{section} argument allows you to specify the
sectional unit within which you want your ``theorem'' to be numbered.

After executing the \ci{newtheorem} command in the preamble of your
document, you can use the following command within the document.
\begin{code}
\verb|\begin{|\emph{name}\verb|}[|\emph{text}\verb|]|\\
This is my interesting theorem\\
\verb|\end{|\emph{name}\verb|}|     
\end{code}

This should be enough theory. The following examples will hopefully
remove the final remains of doubt and make it clear that the
\verb|\newtheorem| environment is way too complex to understand.
\begin{example}
% definitions for the document
% preamble
\newtheorem{law}{Law}
\newtheorem{jury}[law]{Jury}
%in the document
\begin{law} \label{law:box}
Don't hide in the witness box
\end{law}
\begin{jury}[The Twelve]
It could be you! So beware and
see law~\ref{law:box}\end{jury}
\begin{law}No, No, No\end{law}
\end{example}

The ``Jury'' theorem uses the same counter as the ``Law''
theorem. Therefore it gets a number which is in sequence with
the other ``Laws''. The argument in square brackets is used to specify 
a title or something similar for the theorem.
\begin{example}
\flushleft
\newtheorem{mur}{Murphy}[section]
\begin{mur}
If there are two or more 
ways to do something, and 
one of those ways can result 
in a catastrophe, then 
someone will do it.\end{mur}
\end{example}

The ``Murphy'' theorem gets a number which is linked to the number of
the current section. You could also use another unit, for example chapter or
subsection.

\section{Bold symbols}
\index{bold symbols}

It is quite difficult to get bold symbols in \LaTeX{}; this is
probably intentional as amateur typesetters tend to overuse them.  The
font change command \verb|\mathbf| gives bold letters, but these are
roman (upright) whereas mathematical symbols are normally italic.
There is a \ci{boldmath} command, but \emph{this can only be used
outside mathematics mode}. It works for symbols too.
\begin{example}
\begin{displaymath}
\mu, M \qquad \mathbf{M} \qquad
\mbox{\boldmath $\mu, M$}
\end{displaymath}
\end{example}

\noindent
Notice that the comma is bold too, which may not be what is required.

The package \pai{amsbsy} (included by \pai{amsmath}) as well as the
\pai{bm} from the tools bundle make this much easier as they include
a \ci{boldsymbol} command.
\ifx\boldsymbol\undefined\else
\begin{example}
\begin{displaymath}
\mu, M \qquad
\boldsymbol{\mu}, \boldsymbol{M}
\end{displaymath}
\end{example}
\fi


%%% Local Variables: 
%%% mode: latex
%%% TeX-master: "lshort"
%%% End: 


 
%%%%%%%%%%%%%%%%%%%%%%%%%%%%%%%%%%%%%%%%%%%%%%%%%%%%%%%%%%%%%%%%%
% Contents: TeX and LaTeX and AMS symbols for Maths
% $Id: lssym.tex 374 2006-06-07 23:22:03Z marcilr $
%%%%%%%%%%%%%%%%%%%%%%%%%%%%%%%%%%%%%%%%%%%%%%%%%%%%%%%%%%%%%%%%%


\section{List of Mathematical Symbols}  \label{symbols}
 
The following tables demonstrate all the symbols normally accessible
from \emph{math mode}.  

%
% Conditional Text in case the AMS Fonts are installed
%
To use the symbols listed in
Tables~\ref{AMSD}--\ref{AMSNBR},\footnote{These tables were derived
  from \texttt{symbols.tex} by David~Carlisle and subsequently changed
extensively as suggested by Josef~Tkadlec.} the package
\pai{amssymb} must be loaded in the preamble of the document and the
AMS math fonts must be installed on the system. If the AMS package and
fonts are not installed on your system, have a look at\\ 
\CTANref|macros/latex/required/amslatex|. An even more comprehensive list of
symbols can be found at  \CTANref|info/symbols/comprehensive|.
 
\begin{table}[!h]
\caption{Math Mode Accents.}  \label{mathacc}
\begin{symbols}{*3{cl}}
\W{\hat}{a}     & \W{\check}{a} & \W{\tilde}{a} \\
\W{\grave}{a} & \W{\dot}{a} & \W{\ddot}{a}     \\
\W{\bar}{a} &\W{\vec}{a} &\W{\widehat}{A}  \\  
\W{\acute}{a}  & \W{\breve}{a} &\W{\widetilde}{A}
\end{symbols}
\end{table}
 
\begin{table}[!h]
\caption{Greek Letters.}
\begin{symbols}{*4{cl}}
 \X{\alpha}     & \X{\theta}     & \X{o}          & \X{\upsilon}  \\
 \X{\beta}      & \X{\vartheta}  & \X{\pi}        & \X{\phi}      \\
 \X{\gamma}     & \X{\iota}      & \X{\varpi}     & \X{\varphi}   \\
 \X{\delta}     & \X{\kappa}     & \X{\rho}       & \X{\chi}      \\
 \X{\epsilon}   & \X{\lambda}    & \X{\varrho}    & \X{\psi}      \\
 \X{\varepsilon}& \X{\mu}        & \X{\sigma}     & \X{\omega}    \\
 \X{\zeta}      & \X{\nu}        & \X{\varsigma}  &               \\
 \X{\eta}       & \X{\xi}        & \X{\tau} & \\
 \X{\Gamma}     & \X{\Lambda}    & \X{\Sigma}     & \X{\Psi}      \\
 \X{\Delta}     & \X{\Xi}        & \X{\Upsilon}   & \X{\Omega}    \\
 \X{\Theta}     & \X{\Pi}        & \X{\Phi} 
\end{symbols}
\end{table}



\begin{table}[!tbp]
\caption{Binary Relations.}
\bigskip
You can negate the following symbols by prefixing them with a \ci{not} command.
\begin{symbols}{*3{cl}}
 \X{<}           & \X{>}           & \X{=}          \\
 \X{\leq}or \verb|\le|   & \X{\geq}or \verb|\ge|   & \X{\equiv}     \\
 \X{\ll}         & \X{\gg}         & \X{\doteq}     \\
 \X{\prec}       & \X{\succ}       & \X{\sim}       \\
 \X{\preceq}     & \X{\succeq}     & \X{\simeq}     \\
 \X{\subset}     & \X{\supset}     & \X{\approx}    \\
 \X{\subseteq}   & \X{\supseteq}   & \X{\cong}      \\
 \X{\sqsubset}$^a$ & \X{\sqsupset}$^a$ & \X{\Join}$^a$    \\
 \X{\sqsubseteq} & \X{\sqsupseteq} & \X{\bowtie}    \\
 \X{\in}         & \X{\ni}, \verb|\owns|  & \X{\propto}    \\
 \X{\vdash}      & \X{\dashv}      & \X{\models}    \\
 \X{\mid}        & \X{\parallel}   & \X{\perp}      \\
 \X{\smile}      & \X{\frown}      & \X{\asymp}     \\
 \X{:}           & \X{\notin}      & \X{\neq}or \verb|\ne|
\end{symbols}
\centerline{\footnotesize $^a$Use the \textsf{latexsym} package to access this symbol}
\end{table}

\begin{table}[!tbp]
\caption{Binary Operators.}
\begin{symbols}{*3{cl}}
 \X{+}              & \X{-}              & &                 \\
 \X{\pm}            & \X{\mp}            & \X{\triangleleft} \\
 \X{\cdot}          & \X{\div}           & \X{\triangleright}\\
 \X{\times}         & \X{\setminus}      & \X{\star}         \\
 \X{\cup}           & \X{\cap}           & \X{\ast}          \\
 \X{\sqcup}         & \X{\sqcap}         & \X{\circ}         \\
 \X{\vee}, \verb|\lor|     & \X{\wedge}, \verb|\land|  & \X{\bullet}       \\
 \X{\oplus}         & \X{\ominus}        & \X{\diamond}      \\
 \X{\odot}          & \X{\oslash}        & \X{\uplus}        \\
 \X{\otimes}        & \X{\bigcirc}       & \X{\amalg}        \\
 \X{\bigtriangleup} &\X{\bigtriangledown}& \X{\dagger}       \\
 \X{\lhd}$^a$         & \X{\rhd}$^a$         & \X{\ddagger}      \\
 \X{\unlhd}$^a$       & \X{\unrhd}$^a$       & \X{\wr}
\end{symbols}
 
\end{table}

\begin{table}[!tbp]
\caption{BIG Operators.}
\begin{symbols}{*4{cl}}
 \X{\sum}      & \X{\bigcup}   & \X{\bigvee}  \\
 \X{\prod}     & \X{\bigcap}   & \X{\bigwedge} \\
 \X{\coprod}   & \X{\bigsqcup} & \X{\biguplus} \\
 \X{\int}      & \X{\oint}     & \X{\bigodot} \\
 \X{\bigoplus} & &\X{\bigotimes} & \\
\end{symbols}
 
\end{table}


\begin{table}[!tbp]
\caption{Arrows.}
\begin{symbols}{*2{cl}}
 \X{\leftarrow}or \verb|\gets|& \X{\longleftarrow} \\
 \X{\rightarrow}or \verb|\to|& \X{\longrightarrow} \\
 \X{\leftrightarrow}    & \X{\longleftrightarrow} \\
 \X{\Leftarrow}         & \X{\Longleftarrow}     \\
 \X{\Rightarrow}        & \X{\Longrightarrow}    \\
 \X{\Leftrightarrow}    & \X{\Longleftrightarrow}\\
 \X{\mapsto}            & \X{\longmapsto}        \\
 \X{\hookleftarrow}     & \X{\hookrightarrow}    \\
 \X{\leftharpoonup}     & \X{\rightharpoonup}    \\
 \X{\leftharpoondown}   & \X{\rightharpoondown}  \\
 \X{\rightleftharpoons} & \X{\iff}(bigger spaces) \\
 \X{\uparrow}   & \X{\downarrow} \\
 \X{\updownarrow} & \X{\Uparrow} \\
 \X{\Downarrow} &  \X{\Updownarrow} \\
 \X{\nearrow} &  \X{\searrow} \\
  \X{\swarrow} & \X{\nwarrow} \\
 \X{\leadsto}$^a$
\end{symbols}
\centerline{\footnotesize $^a$Use the \textsf{latexsym} package to access this symbol}
\end{table}

\begin{table}[!tbp]
\caption{Delimiters.}\label{tab:delimiters}
\begin{symbols}{*3{cl}}
 \X{(}            & \X{)}            & \X{\uparrow} \\
 \X{[}or \verb|\lbrack|   & \X{]}or \verb|\rbrack|  & \X{\downarrow}   \\
 \X{\{}or \verb|\lbrace|  & \X{\}}or \verb|\rbrace|  & \X{\updownarrow} \\
 \X{\langle}      & \X{\rangle}  & \X{|}or \verb|\vert| \\
 \X{\lfloor}      & \X{\rfloor}      & \X{\lceil}       \\
 \X{/}            & \X{\backslash}   & \X{\Updownarrow}\\ 
 \X{\Uparrow}     &  \X{\Downarrow}  & \X{\|}or \verb|\Vert| \\
  \X{\rceil}    
\end{symbols}
\end{table}

\begin{table}[!tbp]
\caption{Large Delimiters.}
\begin{symbols}{*3{cl}}
 \Y{\lgroup}      & \Y{\rgroup}      & \Y{\lmoustache}  \\
 \Y{\arrowvert}   & \Y{\Arrowvert}   & \Y{\bracevert} \\
 \Y{\rmoustache} \\
\end{symbols}
\end{table}


\begin{table}[!tbp]
\caption{Miscellaneous Symbols.}
\begin{symbols}{*4{cl}}
 \X{\dots}       & \X{\cdots}      & \X{\vdots}      & \X{\ddots}     \\
 \X{\hbar}       & \X{\imath}      & \X{\jmath}      & \X{\ell}       \\
 \X{\Re}         & \X{\Im}         & \X{\aleph}      & \X{\wp}        \\
 \X{\forall}     & \X{\exists}     & \X{\mho}$^a$      & \X{\partial}   \\
 \X{'}           & \X{\prime}      & \X{\emptyset}   & \X{\infty}     \\
 \X{\nabla}      & \X{\triangle}   & \X{\Box}$^a$     & \X{\Diamond}$^a$ \\
 \X{\bot}        & \X{\top}        & \X{\angle}      & \X{\surd}      \\
\X{\diamondsuit} & \X{\heartsuit}  & \X{\clubsuit}   & \X{\spadesuit} \\
 \X{\neg}or \verb|\lnot| & \X{\flat}       & \X{\natural}    & \X{\sharp}

\end{symbols}
\centerline{\footnotesize $^a$Use the \textsf{latexsym} package to access this symbol}
\end{table}


\begin{table}[!tbp]
\caption{Non-Mathematical Symbols.}
\bigskip
These symbols can also be used in text mode.
\begin{symbols}{*4{cl}}
 \SC{\dag}  &  \SC{\S}  &  \SC{\copyright} &  \SC{\textregistered}  \\
 \SC{\ddag} &  \SC{\P}  &  \SC{\pounds}    &  \SC{\%}               \\
\end{symbols}
\end{table}

%
%
% If the AMS Stuff is not available, we drop out right here :-)
%

\begin{table}[!tbp]
\caption{AMS Delimiters.}\label{AMSD}
\bigskip
\begin{symbols}{*4{cl}}
\X{\ulcorner}&\X{\urcorner}&\X{\llcorner}&\X{\lrcorner}\\
\X{\lvert}&\X{\rvert}&\X{\lVert}&\X{\rVert}
\end{symbols}
\end{table}

\begin{table}[!tbp]
\caption{AMS Greek and Hebrew.}
\begin{symbols}{*5{cl}}
\X{\digamma}     &\X{\varkappa} & \X{\beth} &\X{\gimel} & \X{\daleth}    
\end{symbols}
\end{table}

\begin{table}[!tbp]
\caption{AMS Binary Relations.}
\begin{symbols}{*3{cl}}
 \X{\lessdot}           & \X{\gtrdot}            & \X{\doteqdot} \\
 \X{\leqslant}          & \X{\geqslant}          & \X{\risingdotseq}     \\
 \X{\eqslantless}       & \X{\eqslantgtr}        & \X{\fallingdotseq}    \\
 \X{\leqq}              & \X{\geqq}              & \X{\eqcirc}           \\
 \X{\lll}or \verb|\llless| & \X{\ggg}            & \X{\circeq}  \\
 \X{\lesssim}           & \X{\gtrsim}            & \X{\triangleq}        \\
 \X{\lessapprox}        & \X{\gtrapprox}         & \X{\bumpeq}           \\
 \X{\lessgtr}           & \X{\gtrless}           & \X{\Bumpeq}           \\
 \X{\lesseqgtr}         & \X{\gtreqless}         & \X{\thicksim}         \\
 \X{\lesseqqgtr}        & \X{\gtreqqless}        & \X{\thickapprox}      \\
 \X{\preccurlyeq}       & \X{\succcurlyeq}       & \X{\approxeq}         \\
 \X{\curlyeqprec}       & \X{\curlyeqsucc}       & \X{\backsim}          \\
 \X{\precsim}           & \X{\succsim}           & \X{\backsimeq}        \\
 \X{\precapprox}        & \X{\succapprox}        & \X{\vDash}            \\
 \X{\subseteqq}         & \X{\supseteqq}         & \X{\Vdash}            \\
 \X{\shortparallel}     & \X{\Supset}            & \X{\Vvdash}           \\
 \X{\blacktriangleleft} & \X{\sqsupset}          & \X{\backepsilon}      \\
 \X{\vartriangleright}  & \X{\because}           & \X{\varpropto}        \\
 \X{\blacktriangleright}& \X{\Subset}            & \X{\between}          \\
 \X{\trianglerighteq}   & \X{\smallfrown}        & \X{\pitchfork}        \\
 \X{\vartriangleleft}   & \X{\shortmid} 	 & \X{\smallsmile} 	\\
 \X{\trianglelefteq}    & \X{\therefore} 	 & \X{\sqsubset}  
\end{symbols}
\end{table}

\begin{table}[!tbp]
\caption{AMS Arrows.}
\begin{symbols}{*2{cl}}
 \X{\dashleftarrow}      & \X{\dashrightarrow}     \\
 \X{\leftleftarrows}     & \X{\rightrightarrows}   \\
 \X{\leftrightarrows}    & \X{\rightleftarrows}    \\
 \X{\Lleftarrow}         & \X{\Rrightarrow}        \\
 \X{\twoheadleftarrow}   & \X{\twoheadrightarrow}  \\
 \X{\leftarrowtail}      & \X{\rightarrowtail}     \\
 \X{\leftrightharpoons}  & \X{\rightleftharpoons}  \\
 \X{\Lsh}                & \X{\Rsh}                \\
 \X{\looparrowleft}      & \X{\looparrowright}     \\
 \X{\curvearrowleft}     & \X{\curvearrowright}    \\
 \X{\circlearrowleft}    & \X{\circlearrowright}   \\
 \X{\multimap}  &  \X{\upuparrows}  \\
 \X{\downdownarrows} & \X{\upharpoonleft} \\
 \X{\upharpoonright} & \X{\downharpoonright} \\
 \X{\rightsquigarrow} & \X{\leftrightsquigarrow} \\
\end{symbols}
\end{table}

\begin{table}[!tbp]
\caption{AMS Negated Binary Relations and Arrows.}\label{AMSNBR}
\begin{symbols}{*3{cl}}
 \X{\nless}           & \X{\ngtr}            & \X{\varsubsetneqq}  \\
 \X{\lneq}            & \X{\gneq}            & \X{\varsupsetneqq}  \\
 \X{\nleq}            & \X{\ngeq}            & \X{\nsubseteqq}     \\
 \X{\nleqslant}       & \X{\ngeqslant}       & \X{\nsupseteqq}     \\
 \X{\lneqq}           & \X{\gneqq}           & \X{\nmid}           \\
 \X{\lvertneqq}       & \X{\gvertneqq}       & \X{\nparallel}      \\
 \X{\nleqq}           & \X{\ngeqq}           & \X{\nshortmid}      \\
 \X{\lnsim}           & \X{\gnsim}           & \X{\nshortparallel} \\
 \X{\lnapprox}        & \X{\gnapprox}        & \X{\nsim}           \\
 \X{\nprec}           & \X{\nsucc}           & \X{\ncong}          \\
 \X{\npreceq}         & \X{\nsucceq}         & \X{\nvdash}         \\
 \X{\precneqq}        & \X{\succneqq}        & \X{\nvDash}         \\
 \X{\precnsim}        & \X{\succnsim}        & \X{\nVdash}         \\
 \X{\precnapprox}     & \X{\succnapprox}     & \X{\nVDash}         \\
 \X{\subsetneq}       & \X{\supsetneq}       & \X{\ntriangleleft}  \\
 \X{\varsubsetneq}    & \X{\varsupsetneq}    & \X{\ntriangleright} \\
 \X{\nsubseteq}       & \X{\nsupseteq}       & \X{\ntrianglelefteq}\\
 \X{\subsetneqq}      & \X{\supsetneqq}      &\X{\ntrianglerighteq}\\[0.5ex]
 \X{\nleftarrow}      & \X{\nrightarrow}     & \X{\nleftrightarrow}\\
 \X{\nLeftarrow}      & \X{\nRightarrow}     & \X{\nLeftrightarrow}

\end{symbols}
\end{table}

\begin{table}[!tbp]
\caption{AMS Binary Operators.}
\begin{symbols}{*3{cl}}
 \X{\dotplus}        & \X{\centerdot}      &       \\
 \X{\ltimes}         & \X{\rtimes}         & \X{\divideontimes} \\
 \X{\doublecup}      & \X{\doublecap}	   & \X{\smallsetminus} \\
 \X{\veebar}         & \X{\barwedge}       & \X{\doublebarwedge}\\
 \X{\boxplus}        & \X{\boxminus}       & \X{\circleddash}   \\
 \X{\boxtimes}       & \X{\boxdot}         & \X{\circledcirc}   \\
 \X{\intercal}       & \X{\circledast}     & \X{\rightthreetimes} \\
 \X{\curlyvee}       & \X{\curlywedge}     & \X{\leftthreetimes}
\end{symbols}
\end{table}

\begin{table}[!tbp]
\caption{AMS Miscellaneous.}
\begin{symbols}{*3{cl}}
 \X{\hbar}             & \X{\hslash}           & \X{\Bbbk}            \\
 \X{\square}           & \X{\blacksquare}      & \X{\circledS}        \\
 \X{\vartriangle}      & \X{\blacktriangle}    & \X{\complement}      \\
 \X{\triangledown}     &\X{\blacktriangledown} & \X{\Game}            \\
 \X{\lozenge}          & \X{\blacklozenge}     & \X{\bigstar}         \\
 \X{\angle}            & \X{\measuredangle}    & \\
 \X{\diagup}           & \X{\diagdown}         & \X{\backprime}       \\
 \X{\nexists}          & \X{\Finv}             & \X{\varnothing}      \\
 \X{\eth}              & \X{\sphericalangle}   & \X{\mho}              
\end{symbols}
\end{table}



\begin{table}[!tbp]
\caption{Math Alphabets.}
\begin{symbols}{@{}*3l@{}}
Example& Command &Required package\\
\hline
\rule{0pt}{1.05em}$\mathrm{ABCDE abcde 1234}$
        & \verb|\mathrm{ABCDE abcde 1234}|
        &       \\
$\mathit{ABCDE abcde 1234}$
        & \verb|\mathit{ABCDE abcde 1234}|
        &       \\
$\mathnormal{ABCDE abcde 1234}$
        & \verb|\mathnormal{ABCDE abcde 1234}|
        &  \\
$\mathcal{ABCDE abcde 1234}$
        & \verb|\mathcal{ABCDE abcde 1234}|
        &  \\
$\mathscr{ABCDE abcde 1234}$
        &\verb|\mathscr{ABCDE abcde 1234}|
        &\pai{mathrsfs}\\
$\mathfrak{ABCDE abcde 1234}$
        & \verb|\mathfrak{ABCDE abcde 1234}|
        &\pai{amsfonts}  or \textsf{amssymb}  \\
$\mathbb{ABCDE abcde 1234}$
        & \verb|\mathbb{ABCDE abcde 1234}|
        &\pai{amsfonts}  or \textsf{amssymb} \\
\end{symbols}
\end{table}


\endinput

%

% Local Variables:
% TeX-master: "lshort2e"
% mode: latex
% mode: flyspell
% End:

%%%%%%%%%%%%%%%%%%%%%%%%%%%%%%%%%%%%%%%%%%%%%%%%%%%%%%%%%%%%%%%%%
% Contents: Specialities of the LaTeX system
% $Id: spec.tex 14 2002-05-26 03:44:42Z marcilr $
%%%%%%%%%%%%%%%%%%%%%%%%%%%%%%%%%%%%%%%%%%%%%%%%%%%%%%%%%%%%%%%%%
 
\chapter{Specialities}
\begin{intro}
  When putting together a large document, \LaTeX{} will help you
  with some special features like index generation,
  bibliography management, and other things.
  A much more complete description of specialities and
  enhancements possible with \LaTeX{} can be found in the
  {\normalfont\manual{}} and {\normalfont \companion}.
\end{intro}

\section{Including EPS Graphics}
\LaTeX{} provides the basic facilities to work with floating bodies
such as images or graphics, with the \texttt{figure} and the
\texttt{table} environment.

There are also several possibilities to generate the actual
\wi{graphics} with basic \LaTeX{} or a \LaTeX{} extension package.
Unfortunately, most users find them quite difficult to understand.
Therefore this will not be explained any further in this manual.
Please refer to \companion{} and the \manual{} for more information on
that subject.

A much easier way to get graphics into a document, is to generate them
with a specialised software package\footnote{Such as XFig, CorelDraw!,
  Freehand, Gnuplot, \ldots} and then include the finished graphics
into the document. Here again, \LaTeX{} packages offer many ways to do
that. In this introduction, only the use of \wi{Encapsulated
  PostScript} (EPS) graphics will be discussed, because it is quite
easy to do and widely used.  In order to use pictures in the EPS
format, you must have a \wi{PostScript} printer\footnote{Another
  possibility to output PostScript is the \textsc{\wi{GhostScript}}
  program available from
  \texttt{CTAN:/tex-archive/support/ghostscript}. Windows and OS/2 users might
  want to look for \textsc{GSview}.} available for output.

A good set of commands for inclusion of graphics is provided in the
\pai{graphicx} package by D.~P.~Carlisle. It is part of a whole family
of packages called the ``graphics''
bundle\footnote{\texttt{CTAN:/tex-archive/macros/latex/required/graphics}}.

Assuming you are working on a system with a
PostScript printer available for output and with the \textsf{graphicx}
package installed, you can use the following step by step guide to
include a picture into your document:

\begin{enumerate}
\item Export the picture from your graphics program in EPS
  format.\footnote{If your software can not export into EPS format, you
    can try to install a PostScript printer driver (some Apple
    LaserWriter for example) and then print to a file with this
    driver. With some luck this file will be in EPS format. Note that
    an EPS must not contain more than one page. Some printer drivers
    can be explicitly configured to produce EPS format.}
\item Load the \textsf{graphicx} package in the preamble of the input
  file with
\begin{lscommand}
\verb|\usepackage[|\emph{driver}\verb|]{graphicx}|
\end{lscommand}
where \emph{driver} is the name of your ``dvi to postscript''
converter program. The most
widely used program is called \texttt{dvips}. The name of the driver is
required, because there is no standard on how graphics are included in
\TeX{}. Knowing the name of the \emph{driver}, the
\textsf{graphicx} package can choose the correct method to insert
information about the graphics into the \texttt{.dvi}~file, so that the
printer understands it and can correctly include the \texttt{.eps} file.
\item Use the command 
\begin{lscommand}
\ci{includegraphics}\verb|[|\emph{key}=\emph{value}, \ldots\verb|]{|\emph{file}\verb|}|
\end{lscommand}
to include \emph{file} into your document. The optional parameter
accepts a comma separated list of \emph{keys} and associated
\emph{values}. The \emph{keys} can be used to alter the width, height
and rotation of the included graphic. Table~\ref{keyvals} lists the
most important keys.
\end{enumerate}

\begin{table}[htb]
\caption{Key Names for \textsf{graphicx} Package.}
\label{keyvals}
\begin{lined}{9cm}
\begin{tabular}{@{}ll}
\texttt{width}& scale graphic to the specified width\\
\texttt{height}& scale graphic to the specified height\\
\texttt{angle}& rotate graphic counterclockwise\\
\texttt{scale}& scale graphic \\
\end{tabular}

\bigskip
\end{lined}
\end{table}

\pagebreak
The following example code will hopefully make things clear:
\begin{code}
\begin{verbatim}
\begin{figure}
\begin{center}
\includegraphics[angle=90, width=0.5\textwidth]{test}
\end{center}
\end{figure}
\end{verbatim}
\end{code}
It includes the graphic stored in the file \texttt{test.eps}. The
graphic is \emph{first} rotated by an angle of 90 degrees and
\emph{then} scaled to the final width of 0.5 times the width of a
standard paragraph.  The aspect ratio is $1.0$, because no special
height is specified.  The width and height parameters can also be
specified in absolute dimensions. Refer to Table~\ref{units} on
page~\pageref{units} for more information. If you want to know more
about this topic, make sure to read \cite{graphics} and \cite{eps}.

\section{Bibliography}
 
You can produce a \wi{bibliography} with the \ei{thebibliography}
environment.  Each entry starts with
\begin{lscommand}
\ci{bibitem}\verb|{|\emph{marker}\verb|}|
\end{lscommand}
The \emph{marker} is then used to cite the book, article or paper
within the document.
\begin{lscommand}
\ci{cite}\verb|{|\emph{marker}\verb|}|
\end{lscommand}
The numbering of the entries is generated automatically.  The
parameter after the \verb|\begin{thebibliography}| command sets the maximum
width of these numbers. In the example below, \verb|{99}| tells
\LaTeX{} to expect that none of the bibliography item numbers will be
wider than the number 99.
\enlargethispage{2cm}
\begin{example}
Partl~\cite{pa} has 
proposed that \ldots 
\begin{thebibliography}{99}
\bibitem{pa} H.~Partl: 
\emph{German \TeX},
TUGboat Volume~9, Issue~1 (1988)
\end{thebibliography}
\end{example}

\chaptermark{Specialities} % w need to fix the damage done by the
                           %bibliography example.
\thispagestyle{fancyplain}

\newpage

For larger projects, you might want to check out the Bib\TeX{}
program. Bib\TeX{} is included with most \TeX{} distributions. It
allows you to maintain a bibliographic database and then extract the
references relevant to things you cited in your paper. The visual
presentation of Bib\TeX{} generated bibliographies is based on a style
sheets concept which allows you to create bibliographies following
a wide range of established designs.

\section{Indexing} \label{sec:indexing}
A very useful feature of many books is their \wi{index}. With \LaTeX{}
and the support program \texttt{makeindex}\footnote{On systems not
  necessarily supporting
  filenames longer than 8~characters, the program may be called
  \texttt{makeidx}.}, an index can be generated quite easily.  In this
introduction, only the basic index generation commands will be explained.
For a more in-depth view, please refer to \companion.  \index{makeindex
  program} \index{makeidx package}

To enable the indexing feature of \LaTeX{}, the \pai{makeidx} package
must be loaded in the preamble with:
\begin{lscommand}
\verb|\usepackage{makeidx}|
\end{lscommand}
\noindent and the special indexing commands must be enabled by putting 
the
\begin{lscommand}
  \ci{makeindex}
\end{lscommand}
\noindent command into the input file preamble.

The content of the index is specified with
\begin{lscommand}
  \ci{index}\verb|{|\emph{key}\verb|}|
\end{lscommand}
\noindent commands, where \emph{key} is the index entry. You enter the
index commands at the points in the text where you want the final index
entries to point to.  Table~\ref{index} explains the syntax of the
\emph{key} argument with several examples.

\begin{table}[!tp]
\caption{Index Key Syntax Examples.}
\label{index}
\begin{center}
\begin{tabular}{@{}lll@{}}
  \textbf{Example} &\textbf{Index Entry} &\textbf{Comment}\\\hline
  \rule{0pt}{1.05em}\verb|\index{hello}| &hello, 1 &Plain entry\\ 
\verb|\index{hello!Peter}|   &\hspace*{2ex}Peter, 3 &Subentry under `hello'\\ 
\verb|\index{Sam@\textsl{Sam}}|     &\textsl{Sam}, 2& Formatted entry\\ 
\verb|\index{Lin@\textbf{Lin}}|     &\textbf{Lin}, 7& Same as above\\ 
\verb.\index{Jenny|textbf}.     &Jenny, \textbf{3}& Formatted page number\\
\verb.\index{Joe|textit}.     &Joe, \textit{5}& Same as above
\end{tabular}
\end{center}
\end{table}

When the input file is processed with \LaTeX{}, each \verb|\index|
command writes an appropriate index entry together with the current
page number to a special file. The file has the same name as the
\LaTeX{} input file, but a different extension (\verb|.idx|). This
\texttt{.idx} file can then be processed with the \texttt{makeindex}
program.

\begin{lscommand}
  \texttt{makeindex} \emph{filename}
\end{lscommand}
The \texttt{makeindex} program generates a sorted index with the same base
file name, but this time with the extension \texttt{.ind}. If now the
\LaTeX{} input file is processed again, this sorted index gets
included into the document at the point where \LaTeX{} finds
\begin{lscommand}
  \ci{printindex}
\end{lscommand}

The \pai{showidx} package which comes with \LaTeXe{} prints out all
index entries in the left margin of the text. This is quite useful for
proofreading a document and verifying the index.
   
\section{Fancy Headers}
\label{sec:fancy}

The \pai{fancyhdr} package,\footnote{Available from
  \texttt{CTAN:/tex-archive/macros/latex/contrib/supported/fancyhdr}.} written by
Piet van Oostrum, provides a few simple commands which allow you to
customize the header and footer lines of your document.  If you look
at the top of this page, you can see a possible application of this
package.

\begin{figure}[!htbp]
\begin{lined}{\textwidth}
\begin{verbatim}
\documentclass{book}
\usepackage{fancyhdr}
\pagestyle{fancy}
% with this we ensure that the chapter and section
% headings are in lowercase.
\renewcommand{\chaptermark}[1]{\markboth{#1}{}}
\renewcommand{\sectionmark}[1]{\markright{\thesection\ #1}}
\fancyhf{}  % delete current setting for header and footer
\fancyhead[LE,RO]{\bfseries\thepage}
\fancyhead[LO]{\bfseries\rightmark}
\fancyhead[RE]{\bfseries\leftmark}
\renewcommand{\headrulewidth}{0.5pt}
\renewcommand{\footrulewidth}{0pt}
\addtolength{\headheight}{0.5pt} % make space for the rule
\fancypagestyle{plain}{%
   \fancyhead{} % get rid of headers on plain pages
   \renewcommand{\headrulewidth}{0pt} % and the line
}
\end{verbatim}
\end{lined}
\caption{Example \pai{fancyhdr} Setup.} \label{fancyhdr}
\end{figure}

The tricky problem when customising headers and footers is to get
things like running section and chapter names in there. \LaTeX{}
accomplishes this with a two-stage approach. In the header and footer
definition, you use the commands \ci{rightmark} and \ci{leftmark} to
represent the current section and chapter heading, respectively.
The values of these two commands are overwritten whenever a chapter or
section command is processed. 

For ultimate flexibility, the \verb|\chapter| command and its friends
do not redefine \ci{rightmark} and \ci{leftmark} themselves, they call
yet another command called \ci{chaptermark}, \ci{sectionmark} or
\ci{subsectionmark} which is responsible for redefining \ci{rightmark}
and \ci{leftmark}.

So, if you wanted to change the look of the chapter
name in the header line, you simply have to ``renew'' the \ci{chaptermark}
command. \cih{sectionmark}\cih{subsectionmark}

 
Figure~\ref{fancyhdr} shows a possible setup for the \pai{fancyhdr}
package which makes the headers look about the same as they look in
this booklet. In any case I suggest you fetch the documentation for
the package at the address mentioned in the footnote. 

\section{The Verbatim Package}

Earlier in this book, you got to know the \ei{verbatim}
\emph{environment}.  In this section, you are going to learn about the
\pai{verbatim} \emph{package}. The \pai{verbatim} package is basically
a re-implementation of the \ei{verbatim} environment, which works around
some of the limitations of the original \ei{verbatim} environment.
This by itself is not spectacular, but with the implementation of the
\pai{verbatim} package, there was also new functionality added, and
this is the reason I am mentioning the package here. The \pai{verbatim}
package provides the

\begin{lscommand}
\ci{verbatiminput}\verb|{|\emph{filename}\verb|}|
\end{lscommand}

\noindent command which allows you to include raw ASCII text into your
document as if it was inside a \ei{verbatim} environment.

As the \pai{verbatim} package is part of the `tools' bundle, you
should find it preinstalled on most systems. If you want to know more
about this package, make sure to read \cite{verbatim} 


\section{Downloading and Installing \LaTeX{} Packages}

Most \LaTeX{} installations come with a large set of pre-installed
style packages, but there are many more available on the net. The main
place to look for style package on the Internet is CTAN (\verb|http://www.ctan.org/|).

Packages, such as, \pai{geometry}, \pai{hyphenat}, and many
others, are typically made up of two files: a file with the extension
\texttt{.ins} and another with the extension \texttt{.dtx}. Often
there will be a \texttt{readme.txt} with a brief description of the
package. You should of course read this file first.

In any event, once you have copied the package files onto your
machine, you still have to process them in a way that (a) your
\TeX\ distribution knows about the new style package and (b) you get
the documentation.  Here's how you do the first part:

\begin{enumerate}
\item Run \LaTeX{} on the \texttt{.ins} file. This will
  extract a \texttt{.sty} file.
\item Move the \texttt{.sty} file to a place where your distribution
  can find it. Usually this is in your \texttt{\ldots/\emph{localtexmf}/tex/latex}
  subdirectory (Windows or OS/2 users should feel free to change the
  direction of the slashes).
\item Refresh your distribution's file-name database. The command
  depends on the \LaTeX-Distribution you use:
  teTeX, fpTeX -- \texttt{texhash}; web2c -- \texttt{maktexlsr};
  MikTeX -- \texttt{initexmf -update-fndb} or use the GUI.
\end{enumerate}

\noindent Now you can extract the documentation from the
\texttt{.dtx} file:

\begin{enumerate}
\item Run \LaTeX\ on the \texttt{.dtx} file.  This will generate a
  \texttt{.dvi} file. Note that you may have to run \LaTeX\
  several times before it gets the cross-references right.
\item Check to see if \LaTeX\ has produced a \texttt{.idx} file
  among the various files you now have.
  If you do not see this file, then you may proceed to
  step~\ref{step:final}.
\item In order to generate the index, type the following:\\
        \fbox{\texttt{makeindex -s gind.ist \textit{name}}}\\
        (where \textit{name} stands for the main-file name without any
    extension).
 \item Run \LaTeX\ on the \texttt{.dtx} file once again. \label{step:next}
    
\item Last but not least, make a \texttt{.ps} or \texttt{.pdf}
  file to increase your reading pleasure.\label{step:final}
  
\end{enumerate}

Sometimes you will see that a \texttt{.glo}
(glossary) file has been produced. Run the following
command between
step~\ref{step:next} and~\ref{step:final}:

\noindent\texttt{makeindex -s gglo.ist -o \textit{name}.gls \textit{name}.glo}

\noindent Be sure to run \LaTeX\ on the \texttt{.dtx} one last
time before moving on to step~\ref{step:final}.


%%% Local Variables: 
%%% mode: latex 
%%% TeX-master: "lshort" 
%%% End:  










%%%%%%%%%%%%%%%%%%%%%%%%%%%%%%%%%%%%%%%%%%%%%%%%%%%%%%%%%%%%%%%%%
%%%%%%%%%%%%%%%%%%%%%%%%%%%%%%%%%%%%%%%%%%%%%%%%%%%%%%%%%%%%%%%%%
\setcounter{chapter}{4}
\newcommand{\graphicscompanion}{\emph{The \LaTeX{} Graphics Companion}~\cite{graphicscompanion}} 
\newcommand{\hobby}{\emph{A User's Manual for MetaPost}~\cite{metapost}}
\newcommand{\hoenig}{\emph{\TeX{} Unbound}~\cite{unbound}}
\newcommand{\graphicsinlatex}{\emph{Graphics in \LaTeXe{}}~\cite{ursoswald}}

\chapter{Producing Mathematical Graphics}
\label{chap:graphics}

\begin{intro}
Most people use \LaTeX\ for typesetting their text. But as the non content and
structure oriented approach to authoring is so convenient, \LaTeX\ also offers a,
if somewhat restricted, possibility for producing graphical output from textual 
descriptions. Furthermore, quite a number of \LaTeX\ extensions have been created 
in order to overcome these restrictions. In this section, you will learn about a 
few of them.
\end{intro}

\section{Overview}

The \ei{picture} environment allows programming pictures directly in
\LaTeX. A detailed
description can be found in the \manual. On the one hand, there are rather
severe constraints, as the slopes of line segments as well as the radii of
circles are restricted to a narrow choice of values.  On the other hand, the
\ei{picture} environment of \LaTeXe\ brings with it the \ci{qbezier}
command, ``\texttt{q}'' meaning ``quadratic''.  Many frequently used curves
such as circles, ellipses, or catenaries can be satisfactorily approximated
by quadratic B\'ezier curves, although this may require some mathematical
toil. If, in addition, a programming language like Java is used to generate
\ci{qbezier} blocks of \LaTeX\ input files, the \ei{picture} environment
becomes quite powerful.

Although programming pictures directly in \LaTeX\ is severely restricted,
and often rather tiresome, there are still reasons for doing so. The documents
thus produced are ``small'' with respect to bytes, and there are no additional
graphics files to be dragged along.

Packages like \pai{epic} and \pai{eepic} (described, for instance, in \companion), or
\pai{pstricks} help to eliminate the restrictions hampering the original \ei{picture} 
environment, and greatly strengthen the graphical power of \LaTeX.

While the former two packages just enhance the \ei{picture} environment, the \pai{pstricks}
package has its own drawing environment, \ei{pspicture}. The power of \pai{pstricks} stems
from the fact that this package makes extensive use of \PSi{} possibilities.
In addition, numerous packages have been written for specific purposes. One of them is
\texorpdfstring{\Xy}{Xy}-pic, described at the end of this chapter. A wide variety of these
packages is described in detail in \graphicscompanion{} (not to be confused with \companion).

Perhaps the most powerful graphical tool related with \LaTeX\ is \texttt{MetaPost}, the twin of
Donald E. Knuth's \texttt{METAFONT}. \texttt{MetaPost} has the very powerful and 
mathematically sophisticated programming language of \texttt{METAFONT}. Contrary to \texttt{METAFONT},
which generates bitmaps, \texttt{MetaPost} generates encapsulated \PSi{} files, 
which can be imported in \LaTeX. For an introduction, see \hobby, or the tutorial on \cite{ursoswald}.

A very thorough discussion of \LaTeX{} and \TeX{} strategies for graphics (and fonts) can 
be found in \hoenig.

\section{The \texttt{picture} Environment}
\secby{Urs Oswald}{osurs@bluewin.ch}

\subsection{Basic Commands}

A \ei{picture} environment\footnote{Believe it or not, the picture environment works out of the
box, with standard \LaTeXe{} no package loading necessary.} is created with one of the two commands
\begin{lscommand}
\ci{begin}\verb|{picture}(|$x,y$\verb|)|\ldots\ci{end}\verb|{picture}|
\end{lscommand}
\noindent or
\begin{lscommand}
\ci{begin}\verb|{picture}(|$x,y$\verb|)(|$x_0,y_0$\verb|)|\ldots\ci{end}\verb|{picture}|
\end{lscommand}
The numbers $x,\,y,\,x_0,\,y_0$ refer to \ci{unitlength}, which can be reset any time
(but not within a \ei{picture} environment) with a command such as
\begin{lscommand}
\ci{setlength}\verb|{|\ci{unitlength}\verb|}{1.2cm}|
\end{lscommand}
The default value of \ci{unitlength} is \texttt{1pt}. The first pair, $(x,y)$, effects
the reservation, within the document, of rectangular space for the picture. The optional
second pair, $(x_0,y_0)$, assigns arbitrary coordinates to the bottom left corner of the
reserved rectangle. 

Most drawing commands have one of the two forms
\begin{lscommand}
\ci{put}\verb|(|$x,y$\verb|){|\emph{object}\verb|}|
\end{lscommand}
\noindent or
\begin{lscommand}
\ci{multiput}\verb|(|$x,y$\verb|)(|$\Delta x,\Delta y$\verb|){|$n$\verb|}{|\emph{object}\verb|}|\end{lscommand}
B\'ezier curves are an exception. They are drawn with the command
\begin{lscommand}
\ci{qbezier}\verb|(|$x_1,y_1$\verb|)(|$x_2,y_2$\verb|)(|$x_3,y_3$\verb|)|
\end{lscommand}
\newpage

\subsection{Line Segments}
\begin{example}
\setlength{\unitlength}{5cm}
\begin{picture}(1,1)
  \put(0,0){\line(0,1){1}}
  \put(0,0){\line(1,0){1}}  
  \put(0,0){\line(1,1){1}}  
  \put(0,0){\line(1,2){.5}}
  \put(0,0){\line(1,3){.3333}}
  \put(0,0){\line(1,4){.25}}  
  \put(0,0){\line(1,5){.2}}
  \put(0,0){\line(1,6){.1667}}
  \put(0,0){\line(2,1){1}}
  \put(0,0){\line(2,3){.6667}}
  \put(0,0){\line(2,5){.4}}
  \put(0,0){\line(3,1){1}}  
  \put(0,0){\line(3,2){1}}
  \put(0,0){\line(3,4){.75}}
  \put(0,0){\line(3,5){.6}}
  \put(0,0){\line(4,1){1}}
  \put(0,0){\line(4,3){1}}  
  \put(0,0){\line(4,5){.8}}
  \put(0,0){\line(5,1){1}}
  \put(0,0){\line(5,2){1}}
  \put(0,0){\line(5,3){1}}
  \put(0,0){\line(5,4){1}}
  \put(0,0){\line(5,6){.8333}}
  \put(0,0){\line(6,1){1}}
  \put(0,0){\line(6,5){1}}
\end{picture}
\end{example}
Line segments are drawn with the command
\begin{lscommand}
\ci{put}\verb|(|$x,y$\verb|){|\ci{line}\verb|(|$x_1,y_1$\verb|){|$length$\verb|}}|
\end{lscommand}
The \ci{line} command has two arguments:
\begin{enumerate}
  \item a direction vector,
  \item a length.
\end{enumerate}
The components of the direction vector are restricted to the integers
\[
  -6,\,-5,\,\ldots,\,5,\,6,
\]
and they have to be coprime (no common divisor except 1). The figure illustrates all
25 possible slope values in the first quadrant. The length is relative to \ci{unitlength}.
The length argument is the vertical coordinate in the case of a vertical line segment, the
horizontal coordinate in all other cases.

\subsection{Arrows}

\begin{example}
\setlength{\unitlength}{0.75mm}
\begin{picture}(60,40)
  \put(30,20){\vector(1,0){30}}
  \put(30,20){\vector(4,1){20}}
  \put(30,20){\vector(3,1){25}}
  \put(30,20){\vector(2,1){30}}
  \put(30,20){\vector(1,2){10}}
  \thicklines
  \put(30,20){\vector(-4,1){30}}
  \put(30,20){\vector(-1,4){5}}
  \thinlines
  \put(30,20){\vector(-1,-1){5}}
  \put(30,20){\vector(-1,-4){5}}
\end{picture}
\end{example}
Arrows are drawn with the command
\begin{lscommand}
\ci{put}\verb|(|$x,y$\verb|){|\ci{vector}\verb|(|$x_1,y_1$\verb|){|$length$\verb|}}|
\end{lscommand}
For arrows, the components of the direction vector are even more narrowly restricted than
for line segments, namely to the integers
\[
  -4,\,-3,\,\ldots,\,3,\,4.
\]
Components also have to be coprime (no common divisor except 1). Notice the effect  of the
\ci{thicklines} command on the two arrows pointing to the upper left.

\subsection{Circles}

\begin{example}
\setlength{\unitlength}{1mm}
\begin{picture}(60, 40)
  \put(20,30){\circle{1}}
  \put(20,30){\circle{2}}
  \put(20,30){\circle{4}}
  \put(20,30){\circle{8}}
  \put(20,30){\circle{16}}
  \put(20,30){\circle{32}}
  
  \put(40,30){\circle{1}}
  \put(40,30){\circle{2}}
  \put(40,30){\circle{3}}
  \put(40,30){\circle{4}}
  \put(40,30){\circle{5}}
  \put(40,30){\circle{6}}
  \put(40,30){\circle{7}}
  \put(40,30){\circle{8}}
  \put(40,30){\circle{9}}
  \put(40,30){\circle{10}}
  \put(40,30){\circle{11}}
  \put(40,30){\circle{12}}
  \put(40,30){\circle{13}}
  \put(40,30){\circle{14}}
  
  \put(15,10){\circle*{1}}
  \put(20,10){\circle*{2}}
  \put(25,10){\circle*{3}}
  \put(30,10){\circle*{4}}
  \put(35,10){\circle*{5}}
\end{picture}
\end{example}
The command
\begin{lscommand}
  \ci{put}\verb|(|$x,y$\verb|){|\ci{circle}\verb|{|\emph{diameter}\verb|}}|
\end{lscommand}
\noindent draws a circle with center $(x,y)$ and diameter (not radius) \emph{diameter}.
The \ei{picture} environment only admits diameters up to approximately 14\,mm,
and even below this limit, not all diameters are possible. The \ci{circle*}
command produces disks (filled circles).

As in the case of line segments, one may have to resort to additional packages, 
such as \pai{eepic} or \pai{pstricks}. 
For a thorough description of these packages, see \graphicscompanion.

There is also a possibility within the
\ei{picture} environment. If one is not afraid of doing the necessary calculations
(or leaving them to a program), arbitrary circles and ellipses can be patched
together from quadratic B\'ezier curves. 
See \graphicsinlatex\ for examples and Java source files.

\subsection{Text and Formulas}

\begin{example}
\setlength{\unitlength}{0.8cm}
\begin{picture}(6,5)
  \thicklines
  \put(1,0.5){\line(2,1){3}}
  \put(4,2){\line(-2,1){2}}
  \put(2,3){\line(-2,-5){1}}
  \put(0.7,0.3){$A$}
  \put(4.05,1.9){$B$}
  \put(1.7,2.95){$C$}
  \put(3.1,2.5){$a$}
  \put(1.3,1.7){$b$}
  \put(2.5,1.05){$c$}
  \put(0.3,4){$F=
    \sqrt{s(s-a)(s-b)(s-c)}$}  
  \put(3.5,0.4){$\displaystyle
    s:=\frac{a+b+c}{2}$}
\end{picture}
\end{example}
As this example shows, text and formulas can be written into a \ei{picture} environment with
the \ci{put} command in the usual way.

\subsection{\ci{multiput} and \ci{linethickness}}

\begin{example}
\setlength{\unitlength}{2mm}
\begin{picture}(30,20)
  \linethickness{0.075mm}
  \multiput(0,0)(1,0){26}%
    {\line(0,1){20}}
  \multiput(0,0)(0,1){21}%
    {\line(1,0){25}}
  \linethickness{0.15mm}    
  \multiput(0,0)(5,0){6}%
    {\line(0,1){20}}
  \multiput(0,0)(0,5){5}%
    {\line(1,0){25}}
  \linethickness{0.3mm}    
  \multiput(5,0)(10,0){2}%
    {\line(0,1){20}}
  \multiput(0,5)(0,10){2}%
    {\line(1,0){25}}
\end{picture}
\end{example}
The command
\begin{lscommand}
  \ci{multiput}\verb|(|$x,y$\verb|)(|$\Delta x,\Delta y$\verb|){|$n$\verb|}{|\emph{object}\verb|}|
\end{lscommand}
\noindent has 4 arguments: the starting point, the translation vector from one object to the next, 
the number of objects, and the object to be drawn. The \ci{linethickness} command applies to 
horizontal and vertical line segments, but neither to oblique line segments, nor to circles. 
It does, however, apply to quadratic B\'ezier curves!

\subsection{Ovals}

\begin{example}
\setlength{\unitlength}{0.75cm}
\begin{picture}(6,4)
  \linethickness{0.075mm}
  \multiput(0,0)(1,0){7}%
    {\line(0,1){4}}
  \multiput(0,0)(0,1){5}%
    {\line(1,0){6}}
  \thicklines
  \put(2,3){\oval(3,1.8)} 
  \thinlines
  \put(3,2){\oval(3,1.8)} 
  \thicklines
  \put(2,1){\oval(3,1.8)[tl]} 
  \put(4,1){\oval(3,1.8)[b]} 
  \put(4,3){\oval(3,1.8)[r]} 
  \put(3,1.5){\oval(1.8,0.4)}     
\end{picture}
\end{example}
The command
\begin{lscommand}
  \ci{put}\verb|(|$x,y$\verb|){|\ci{oval}\verb|(|$w,h$\verb|)}|
\end{lscommand}
\noindent or
\begin{lscommand}
  \ci{put}\verb|(|$x,y$\verb|){|\ci{oval}\verb|(|$w,h$\verb|)[|\emph{position}\verb|]}|
\end{lscommand}
\noindent produces an oval centered at $(x,y)$ and having width $w$ and height $h$. The optional 
\emph{position} arguments \texttt{b}, \texttt{t}, \texttt{l}, \texttt{r} refer to 
``top'', ``bottom'', ``left'', ``right'', and can be combined, as the example illustrates. 

Line thickness can be controlled by two kinds of commands: \\ 
\ci{linethickness}\verb|{|\emph{length}\verb|}|
on the one hand, \ci{thinlines} and \ci{thicklines} on the other. While \ci{linethickness}\verb|{|\emph{length}\verb|}|
applies only to horizontal and vertical lines (and quadratic B\'ezier curves), \ci{thinlines} and \ci{thicklines}
apply to oblique line segments as well as to circles and ovals. 


\subsection{Multiple Use of Predefined Picture Boxes}

\begin{example}
\setlength{\unitlength}{0.5mm}
\begin{picture}(120,168)
\newsavebox{\foldera}
\savebox{\foldera}
  (40,32)[bl]{% definition 
  \multiput(0,0)(0,28){2}
    {\line(1,0){40}}
  \multiput(0,0)(40,0){2}
    {\line(0,1){28}}
  \put(1,28){\oval(2,2)[tl]}
  \put(1,29){\line(1,0){5}}
  \put(9,29){\oval(6,6)[tl]}
  \put(9,32){\line(1,0){8}}
  \put(17,29){\oval(6,6)[tr]}
  \put(20,29){\line(1,0){19}}
  \put(39,28){\oval(2,2)[tr]}  
}
\newsavebox{\folderb}
\savebox{\folderb}
  (40,32)[l]{%         definition 
  \put(0,14){\line(1,0){8}}
  \put(8,0){\usebox{\foldera}}
}
\put(34,26){\line(0,1){102}} 
\put(14,128){\usebox{\foldera}}
\multiput(34,86)(0,-37){3}
  {\usebox{\folderb}} 
\end{picture}
\end{example}
A picture box can be \emph{declared} by the command
\begin{lscommand}
  \ci{newsavebox}\verb|{|\emph{name}\verb|}|
\end{lscommand}
\noindent then \emph{defined} by  
\begin{lscommand}
  \ci{savebox}\verb|{|\emph{name}\verb|}(|\emph{width,height}\verb|)[|\emph{position}\verb|]{|\emph{content}\verb|}|
\end{lscommand}
\noindent and finally arbitrarily often be \emph{drawn} by
\begin{lscommand}
  \ci{put}\verb|(|$x,y$\verb|)|\ci{usebox}\verb|{|\emph{name}\verb|}|
\end{lscommand}

The optional \emph{position} parameter has the effect of defining the
`anchor point' of the savebox. In the example it is set to \texttt{bl} which
puts the anchor point into the bottom left corner of the savebox. The other
position specifiers are \texttt{t}op and \texttt{r}ight.

The \emph{name} argument refers to a \LaTeX{} storage bin and therefore is
of a command nature (which accounts for the backslashes in the current
example). Boxed pictures can be nested: In this example, \ci{foldera} is
used within the definition of \ci{folderb}.

The \ci{oval} command had to be used as the \ci{line} command does not work if the segment length is less than 
about 3\,mm.

\subsection{Quadratic B\'ezier Curves}

\begin{example}
\setlength{\unitlength}{0.8cm}
\begin{picture}(6,4)
  \linethickness{0.075mm}
  \multiput(0,0)(1,0){7}
    {\line(0,1){4}}
  \multiput(0,0)(0,1){5}
    {\line(1,0){6}}
  \thicklines
  \put(0.5,0.5){\line(1,5){0.5}}    
  \put(1,3){\line(4,1){2}} 
  \qbezier(0.5,0.5)(1,3)(3,3.5)
  \thinlines   
  \put(2.5,2){\line(2,-1){3}}
  \put(5.5,0.5){\line(-1,5){0.5}}
  \linethickness{1mm}
  \qbezier(2.5,2)(5.5,0.5)(5,3)
  \thinlines
  \qbezier(4,2)(4,3)(3,3)
  \qbezier(3,3)(2,3)(2,2)
  \qbezier(2,2)(2,1)(3,1)
  \qbezier(3,1)(4,1)(4,2)
\end{picture}
\end{example}
As this example illustrates, splitting up a circle into 4 quadratic B\'ezier curves
is not satisfactory. At least 8 are needed. The figure again shows the effect of
the \ci{linethickness} command on horizontal or vertical lines, and of the 
\ci{thinlines} and the \ci{thicklines} commands on oblique line segments. It also 
shows that both kinds of commands affect quadratic B\'ezier curves, each command
overriding all previous ones.

Let $P_1=(x_1,\,y_1),\,P_2=(x_2,\,y_2)$ denote the end points, and $m_1,\,m_2$ the
respective slopes, of a quadratic B\'ezier curve. The intermediate control point 
$S=(x,\,y)$ is then given by the equations
\begin{equation} \label{zwischenpunkt}
  \left\{
    \begin{array}{rcl}
      x & = & \displaystyle \frac{m_2 x_2-m_1x_1-(y_2-y_1)}{m_2-m_1}, \\
      y & = & y_i+m_i(x-x_i)\qquad (i=1,\,2).
    \end{array}
  \right.
\end{equation}
\noindent See \graphicsinlatex\ for a Java program which generates
the necessary \ci{qbezier} command line.

\subsection{Catenary}

\begin{example}
\setlength{\unitlength}{1cm}
\begin{picture}(4.3,3.6)(-2.5,-0.25)
\put(-2,0){\vector(1,0){4.4}}
\put(2.45,-.05){$x$}
\put(0,0){\vector(0,1){3.2}}
\put(0,3.35){\makebox(0,0){$y$}}
\qbezier(0.0,0.0)(1.2384,0.0)
  (2.0,2.7622) 
\qbezier(0.0,0.0)(-1.2384,0.0)
  (-2.0,2.7622)
\linethickness{.075mm}
\multiput(-2,0)(1,0){5}
  {\line(0,1){3}}
\multiput(-2,0)(0,1){4}
  {\line(1,0){4}}
\linethickness{.2mm}
\put( .3,.12763){\line(1,0){.4}}
\put(.5,-.07237){\line(0,1){.4}}
\put(-.7,.12763){\line(1,0){.4}}
\put(-.5,-.07237){\line(0,1){.4}}
\put(.8,.54308){\line(1,0){.4}}
\put(1,.34308){\line(0,1){.4}}
\put(-1.2,.54308){\line(1,0){.4}}
\put(-1,.34308){\line(0,1){.4}}
\put(1.3,1.35241){\line(1,0){.4}}
\put(1.5,1.15241){\line(0,1){.4}}
\put(-1.7,1.35241){\line(1,0){.4}}
\put(-1.5,1.15241){\line(0,1){.4}}
\put(-2.5,-0.25){\circle*{0.2}}
\end{picture}
\end{example}

In this figure, each symmetric half of the catenary $y=\cosh x -1$ is approximated by a quadratic
B\'ezier curve. The right half of the curve ends in the point \((2,\,2.7622)\), the slope there having the value 
\(m=3.6269\). Using again equation (\ref{zwischenpunkt}), we can 
calculate the intermediate control points. They turn out to be $(1.2384,\,0)$ and $(-1.2384,\,0)$. 
The crosses indicate points of the \emph{real} catenary. The error is barely noticeable, being less 
than one percent.

This example points out the use of the optional argument of the \\
\verb|\begin{picture}| command.
The picture is defined in convenient ``mathematical'' coordinates, whereas by the command
\begin{lscommand} 
  \ci{begin}\verb|{picture}(4.3,3.6)(-2.5,-0.25)|
\end{lscommand}
\noindent its lower left corner (marked by the black disk) is assigned the coordinates $(-2.5,-0.25)$. 

\subsection{Rapidity in the Special Theory of Relativity}

\begin{example}
\setlength{\unitlength}{0.8cm}
\begin{picture}(6,4)(-3,-2)
  \put(-2.5,0){\vector(1,0){5}}
  \put(2.7,-0.1){$\chi$}
  \put(0,-1.5){\vector(0,1){3}}
  \multiput(-2.5,1)(0.4,0){13}
    {\line(1,0){0.2}}
  \multiput(-2.5,-1)(0.4,0){13}
    {\line(1,0){0.2}}
  \put(0.2,1.4)
    {$\beta=v/c=\tanh\chi$}
  \qbezier(0,0)(0.8853,0.8853)
    (2,0.9640)
  \qbezier(0,0)(-0.8853,-0.8853)
    (-2,-0.9640)
  \put(-3,-2){\circle*{0.2}}
\end{picture}
\end{example}
The control points of the two B\'ezier curves were calculated with formulas (\ref{zwischenpunkt}).
The positive branch is determined by $P_1=(0,\,0),\,m_1=1$ and $P_2=(2,\,\tanh 2),\,m_2=1/\cosh^2 2$.
Again, the picture is defined in mathematically convenient coordinates, and the lower left corner
is assigned the mathematical coordinates $(-3,-2)$ (black disk).


\section{\texorpdfstring{\Xy}{Xy}-pic}
\secby{Alberto Manuel Brand\~ao Sim\~oes}{albie@alfarrabio.di.uminho.pt}
\pai{xy} is a special package for drawing diagrams. To use it,
simply add the following line to the preamble of your document:
\begin{lscommand}
\verb|\usepackage[|\emph{options}\verb|]{xy}|
\end{lscommand}
\emph{options} is a list of functions from \Xy-pic you want to
load. These options are primarily useful when debugging the package.  I recommend
you pass the \verb!all! option, making \LaTeX{} load all the \Xy{} commands.

\Xy-pic diagrams are drawn over a matrix-oriented canvas, where
each diagram element is placed in a matrix slot:
\begin{example}
\begin{displaymath}
\xymatrix{A & B \\
          C & D }
\end{displaymath}
\end{example}
The \ci{xymatrix} command must be used in math mode. Here, we
specified two lines and two columns. To make this matrix a diagram we
just add directed arrows using the \ci{ar} command.
\begin{example}
\begin{displaymath}
\xymatrix{ A \ar[r] & B \ar[d] \\
           D \ar[u] & C \ar[l] }
\end{displaymath}
\end{example}
The arrow command is placed on the origin cell for the arrow. The
arguments are the direction the arrow should point to (\texttt{u}p,
\texttt{d}own, \texttt{r}ight and \texttt{l}eft).

\begin{example}
\begin{displaymath}
\xymatrix{
  A \ar[d] \ar[dr] \ar[r] & B \\
  D                       & C }
\end{displaymath}
\end{example}
To make diagonals, just use more than one direction. In
fact, you can repeat directions to make bigger arrows.
\begin{example}
\begin{displaymath}
\xymatrix{
  A \ar[d] \ar[dr] \ar[drr] & & \\
  B                      & C & D }
\end{displaymath}
\end{example}

We can draw even more interesting diagrams by adding
labels to the arrows. To do this, we use the common superscript and
subscript operators.
\begin{example}
\begin{displaymath}
\xymatrix{
  A \ar[r]^f \ar[d]_g &
             B \ar[d]^{g'} \\
  D \ar[r]_{f'}       & C }
\end{displaymath}
\end{example}

As shown, you use these operators as in math mode. The only
difference is that that superscript means ``on top of the arrow,''
and subscript means ``under the arrow.'' There is a third operator, the vertical bar: \verb+|+
It causes text to be placed \emph{in} the arrow.
\begin{example}
\begin{displaymath}
\xymatrix{
  A \ar[r]|f \ar[d]|g &
             B \ar[d]|{g'} \\
  D \ar[r]|{f'}       & C }
\end{displaymath}
\end{example}

To draw an arrow with a hole in it, use \verb!\ar[...]|\hole!.

In some situations, it is important to distinguish between different types of
arrows. This can be done by putting labels on them, or changing their appearance:

\begin{example}
\shorthandoff{"}
\begin{displaymath}
\xymatrix{
\bullet\ar@{->}[rr] && \bullet\\
\bullet\ar@{.<}[rr] && \bullet\\
\bullet\ar@{~)}[rr] && \bullet\\
\bullet\ar@{=(}[rr] && \bullet\\
\bullet\ar@{~/}[rr] && \bullet\\
\bullet\ar@{^{(}->}[rr] &&
                       \bullet\\
\bullet\ar@2{->}[rr] && \bullet\\
\bullet\ar@3{->}[rr] && \bullet\\
\bullet\ar@{=+}[rr]  && \bullet
}
\end{displaymath}
\shorthandon{"}
\end{example}

Notice the difference between the following two diagrams:

\begin{example}
\begin{displaymath}
\xymatrix{
 \bullet \ar[r] 
         \ar@{.>}[r] & 
 \bullet
}
\end{displaymath}
\end{example}

\begin{example}
\begin{displaymath}
\xymatrix{
 \bullet \ar@/^/[r] 
         \ar@/_/@{.>}[r] &
 \bullet
}
\end{displaymath}
\end{example}

The modifiers between the slashes define how the curves are drawn.
\Xy-pic offers many ways to influence the drawing of curves;
for more information, check \Xy-pic documentation.


% \begin{example}
% \begin{lscommand}
% \ci{dum}
% \end{lscommand}
% \end{example}


%%%%%%%%%%%%%%%%%%%%%%%%%%%%%%%%%%%%%%%%%%%%%%%%%%%%%%%%%%%%%%%%%
% Contents: Customising LaTeX output
% $Id: custom.tex 14 2002-05-26 03:44:42Z marcilr $
%%%%%%%%%%%%%%%%%%%%%%%%%%%%%%%%%%%%%%%%%%%%%%%%%%%%%%%%%%%%%%%%%
\chapter{Customising \LaTeX}

\begin{intro}
Documents produced by using the commands you have learned up to this
point will look acceptable to a large audience. While they are not
looking fancy, they obey all the established rules of good
typesetting, which will make them easy to read and pleasant to look at.

However there are situations where \LaTeX{} does not provide a
command or environment which matches your needs, or the output
produced by some existing command may not meet your requirements.

In this chapter, I will try to give some hints on
how to teach \LaTeX{} new tricks and how to make it produce output
which looks different than what is provided by default.
\end{intro}


\section{New Commands, Environments and Packages}

You may have noticed that all the commands I introduce in this
book are typeset in a box, and that they show up in the index at the end
of the book. Instead of directly using the necessary \LaTeX{} commands
to achieve this, I have created a \wi{package} in which I defined new
commands and environments for this purpose. Now I can simply write:

\begin{example}
\begin{lscommand}
\ci{dum}
\end{lscommand}
\end{example}

In this example, I am using both a new environment called
\ei{lscommand} which is responsible for drawing the box around the
command and a new command named \ci{ci} which typesets the command
name and also makes a corresponding entry in the index. You can check
this out by looking up the \ci{dum} command in the index at the back
of this book, where you'll find an entry for \ci{dum}, pointing to
every page where I mentioned the \ci{dum} command.

If I ever decide that I do not like the commands to be typeset in
a box any more, I can simply change the definition of the
\texttt{lscommand} environment to create a new look. This is much
easier than going through the whole document to hunt down all the
places where I have used some generic \LaTeX{} commands to draw a
box around some word. 


\subsection{New Commands}

To add your own commands, use the
\begin{lscommand}
\ci{newcommand}\verb|{|%
       \emph{name}\verb|}[|\emph{num}\verb|]{|\emph{definition}\verb|}|
\end{lscommand}
\noindent command. 
Basically, the command requires two arguments: the \emph{name} of the
command you want to create, and the \emph{definition} of the command.
The \emph{num} argument in square brackets is optional and specifies the number
of arguments the new command takes (up to 9 are possible).
If missing it defaults to 0, i.e. no argument allowed.

The following two examples should help you to get the idea.
The first example defines a new command called \ci{tnss}. This is
short for ``The Not So Short Introduction to \LaTeXe''. Such a command
could come in handy if you had to write the title of this book over 
and over again. 

\begin{example}
\newcommand{\tnss}{The not
    so Short Introduction to
    \LaTeXe}
This is ``\tnss'' \ldots{} 
``\tnss''
\end{example}

The next example illustrates how to define a new
command which takes one argument.
The \verb|#1| tag gets replaced by the argument you specify.
If you wanted to use more than one argument, use \verb|#2| and
so on.

\begin{example}
\newcommand{\txsit}[1]
 {This is the \emph{#1} Short 
      Introduction to \LaTeXe}
% in the document body: 
\begin{itemize}
\item \txsit{not so}
\item \txsit{very}
\end{itemize}
\end{example}

\LaTeX{} will not allow you to create a new command which would
overwrite an existing one. But there is a special command in case you
explicitly want this: \ci{renewcommand}.
It uses the same syntax as the \verb|\newcommand|
command.

In certain cases you might also want to use the \ci{providecommand}
command. It works like \ci{newcommand}, but if the command is
already defined, \LaTeXe{} will silently ignore it.

There are some points to note about whitespace following \LaTeX{} commands. See
page \pageref{whitespace} for more information.

\subsection{New Environments}
Similar to the \verb|\newcommand| command, there is also a command
to create your own environments. The \ci{newenvironment} command uses the
following syntax:

\begin{lscommand}
\ci{newenvironment}\verb|{|%
       \emph{name}\verb|}[|\emph{num}\verb|]{|%
       \emph{before}\verb|}{|\emph{after}\verb|}|
\end{lscommand}

Like the \verb|\newcommand| command, you can use \ci{newenvironment}
with an optional argument or without. The material specified
in the \emph{before} argument is processed before the text in the 
environment gets processed. The material in the \emph{after} argument gets
processed when the \verb|\end{|\emph{name}\verb|}| command is encountered.

The example below illustrates the usage of the \ci{newenvironment}
command. 
\begin{example}
\newenvironment{king}
 {\rule{1ex}{1ex}%
      \hspace{\stretch{1}}}
 {\hspace{\stretch{1}}%
      \rule{1ex}{1ex}}

\begin{king} 
My humble subjects \ldots
\end{king}
\end{example}

The \emph{num} argument is used the same way as in the
\verb|\newcommand| command. \LaTeX{} makes sure that you do not define
an environment which already exists. If you ever want to change an
existing command, you can use the \ci{renewenvironment} command. It
uses the same syntax as the \ci{newenvironment} command.

The commands used in this example will be explained later: For the
\ci{rule} command see page \pageref{sec:rule}, for \ci{stretch} go to
page \pageref{cmd:stretch}, and more information on \ci{hspace} can be
found on page \pageref{sec:hspace}.

\subsection{Your own Package}

If you define a lot of new environments and commands, the preamble of
your document will get quite long. In this situation, it is a good
idea to create a \LaTeX{} package containing all your command and
environment definitions. You can then use the \ci{usepackage}
command to make the package available in your document.

\begin{figure}[!htbp]
\begin{lined}{\textwidth}
\begin{verbatim}
% Demo Package by Tobias Oetiker
\ProvidesPackage{demopack}
\newcommand{\tnss}{The not so Short Introduction to \LaTeXe}
\newcommand{\txsit}[1]{The \emph{#1} Short 
                       Introduction to \LaTeXe}
\newenvironment{king}{\begin{quote}}{\end{quote}}
\end{verbatim}
\end{lined}
\caption{Example Package.} \label{package}
\end{figure}

Writing a package consists  basically in copying the contents of
your document preamble into a separate file with a name ending in
\texttt{.sty}. There is one special command,
\begin{lscommand}
\ci{ProvidesPackage}\verb|{|\emph{package name}\verb|}|
\end{lscommand}
\noindent for use at the very beginning of your package
file. \verb|\ProvidesPackage| tells \LaTeX{} the name of the package
and will allow it to issue a sensible error message when you try to
include a package twice. Figure~\ref{package} shows a small example
package which contains the commands defined in the examples above.

\section{Fonts and Sizes}

\subsection{Font changing Commands}
\index{font}\index{font size} \LaTeX{} chooses the appropriate font
and font size based on the logical structure of the document
(sections, footnotes, \ldots).  In some cases, one might like to change
fonts and sizes by hand. To do this, you can use the commands listed in
Tables~\ref{fonts} and~\ref{sizes}. The actual size of each font
is a design issue and depends on the document class and its options. 
Table~\ref{tab:pointsizes} shows the absolute point size for these
commands as implemented in the standard document classes.

\begin{example}
{\small The small and 
\textbf{bold} Romans ruled}
{\Large all of great big 
\textit{Italy}.}
\end{example}

One important feature of \LaTeXe{} is, that the font attributes are
independent. This means, that you can issue size or even font
changing commands and still keep the bold or slant attribute set
earlier.

In \emph{math mode} you can use the font changing \emph{commands} to
temporarily exit \emph{math mode} and enter some normal text. If you want to
switch to another font for math typesetting there exists another
special set of commands. Refer to Table~\ref{mathfonts}.

\begin{table}[!bp]
\caption{Fonts.} \label{fonts}
\begin{lined}{12cm}
%
% Alan suggested not to tell about the other form of the command
% eg \verb|\sffamily| or \verb|\bfseries|. This seems a good thing to me.
%
\begin{tabular}{@{}rl@{\qquad}rl@{}}
\ci{textrm}\verb|{...}|        &      \textrm{\wi{roman}}&
\ci{textsf}\verb|{...}|        &      \textsf{\wi{sans serif}}\\
\ci{texttt}\verb|{...}|        &      \texttt{typewriter}\\[6pt]
\ci{textmd}\verb|{...}|        &      \textmd{medium}&
\ci{textbf}\verb|{...}|        &      \textbf{\wi{bold face}}\\[6pt]
\ci{textup}\verb|{...}|        &       \textup{\wi{upright}}&
\ci{textit}\verb|{...}|        &       \textit{\wi{italic}}\\
\ci{textsl}\verb|{...}|        &       \textsl{\wi{slanted}}&
\ci{textsc}\verb|{...}|        &       \textsc{\wi{small caps}}\\[6pt]
\ci{emph}\verb|{...}|          &            \emph{emphasized} &
\ci{textnormal}\verb|{...}|    &    \textnormal{document} font
\end{tabular}

\bigskip
\end{lined}
\end{table}


\begin{table}[!bp]
\index{font size}
\caption{Font Sizes.} \label{sizes}
\begin{lined}{12cm}
\begin{tabular}{@{}ll}
\ci{tiny}      & \tiny        tiny font \\
\ci{scriptsize}   & \scriptsize  very small font\\
\ci{footnotesize} & \footnotesize  quite small font \\
\ci{small}        &  \small            small font \\
\ci{normalsize}   &  \normalsize  normal font \\
\ci{large}        &  \large       large font
\end{tabular}%
\qquad\begin{tabular}{ll@{}}
\ci{Large}        &  \Large       larger font \\[5pt]
\ci{LARGE}        &  \LARGE       very large font \\[5pt]
\ci{huge}         &  \huge        huge \\[5pt]
\ci{Huge}         &  \Huge        largest
\end{tabular}

\bigskip
\end{lined}
\end{table}

\begin{table}[!tbp]
\caption{Absolute Point Sizes in Standard Classes.}\label{tab:pointsizes}
\label{tab:sizes}
\begin{lined}{12cm}
\begin{tabular}{lrrr}
\multicolumn{1}{c}{size} &
\multicolumn{1}{c}{10pt (default) } &
           \multicolumn{1}{c}{11pt option}  &
           \multicolumn{1}{c}{12pt option}\\
\verb|\tiny|       & 5pt  & 6pt & 6pt\\
\verb|\scriptsize| & 7pt  & 8pt & 8pt\\
\verb|\footnotesize| & 8pt & 9pt & 10pt \\
\verb|\small|        & 9pt & 10pt & 11pt \\
\verb|\normalsize| & 10pt & 11pt & 12pt \\
\verb|\large|      & 12pt & 12pt & 14pt \\
\verb|\Large|      & 14pt & 14pt & 17pt \\
\verb|\LARGE|      & 17pt & 17pt & 20pt\\
\verb|\huge|       & 20pt & 20pt & 25pt\\
\verb|\Huge|       & 25pt & 25pt & 25pt\\
\end{tabular}

\bigskip
\end{lined}
\end{table}


\begin{table}[!bp]
\caption{Math Fonts.} \label{mathfonts}
\begin{lined}{\textwidth}
\begin{tabular}{@{}lll@{}}
\textit{Command}&\textit{Example}&    \textit{Output}\\[6pt]
\ci{mathcal}\verb|{...}|&    \verb|$\mathcal{B}=c$|&     $\mathcal{B}=c$\\
\ci{mathrm}\verb|{...}|&     \verb|$\mathrm{K}_2$|&      $\mathrm{K}_2$\\
\ci{mathbf}\verb|{...}|&     \verb|$\sum x=\mathbf{v}$|& $\sum x=\mathbf{v}$\\
\ci{mathsf}\verb|{...}|&     \verb|$\mathsf{G\times R}$|&        $\mathsf{G\times R}$\\
\ci{mathtt}\verb|{...}|&     \verb|$\mathtt{L}(b,c)$|&   $\mathtt{L}(b,c)$\\
\ci{mathnormal}\verb|{...}|& \verb|$\mathnormal{R_{19}}\neq R_{19}$|&
$\mathnormal{R_{19}}\neq R_{19}$\\
\ci{mathit}\verb|{...}|&     \verb|$\mathit{ffi}\neq ffi$|& $\mathit{ffi}\neq ffi$
\end{tabular}

\bigskip
\end{lined}
\end{table}

In connection with the font size commands, \wi{curly braces} play a
significant role. They are used to build \emph{groups}.  Groups
limit the scope of most \LaTeX{} commands.\index{grouping}

\begin{example}
He likes {\LARGE large and 
{\small small} letters}. 
\end{example}
 
The font size commands also change the line spacing, but only if the
paragraph ends within the scope of the font size command. The closing curly
brace \verb|}| should therefore not come too early.  Note the position of
the \ci{par} command in the next two examples. \footnote{\texttt{\bs{}par}
is equivalent to a blank line}


\begin{example}
{\Large Don't read this! It is not
true. You can believe me!\par}
\end{example}

\begin{example}
{\Large This is not true either.
But remember I am a liar.}\par
\end{example}

If you want to activate a size changing command for a whole paragraph
of text or even more, you might want to use the environment syntax for
font changing commands.

\begin{example}
\begin{Large} 
This is not true.
But then again, what is these
days \ldots
\end{Large}
\end{example}

\noindent This will save you from counting lots of curly braces.

\subsection{Danger, Will Robinson, Danger}

As noted at the beginning of this chapter, it is dangerous to clutter
your document with explicit commands like this, because they work in
opposition to the basic idea of \LaTeX{}, which is to separate the
logical and visual markup of your document.  This means that if you
use the same font changing command in several places in order to
typeset a special kind of information, you should use
\verb|\newcommand| to define a ``logical wrapper command'' for the font
changing command.

\begin{example}
\newcommand{\oops}[1]{\textbf{#1}}
Do not \oops{enter} this room,
it's occupied by a \oops{machine}
of unknown origin and purpose.
\end{example}

This approach has the advantage that you can decide at some later
stage whether you want to use some other visual representation of danger
than \verb|\textbf| without having to wade through your document,
identifying all the occurrences of \verb|\textbf| and then figuring out
for each one whether it was used for pointing out danger or for some other
reason.


\subsection{Advice}

To conclude this journey into the land of fonts and font sizes,
here is a little word of advice:\nopagebreak

\begin{quote}
  \underline{\textbf{Remember\Huge!}} \textit{The}
  \textsf{M\textbf{\LARGE O} \texttt{R}\textsl{E}} fonts \Huge you
  \tiny use \footnotesize \textbf{in} a \small \texttt{document},
  \large \textit{the} \normalsize more \textsc{readable} and
  \textsl{\textsf{beautiful} it bec\large o\Large m\LARGE e\huge s}.
\end{quote}

\section{Spacing}
 
\subsection{Line Spacing}

\index{line spacing} If you want to use larger inter-line spacing in a
document, you can change its value by putting the
\begin{lscommand}
\ci{linespread}\verb|{|\emph{factor}\verb|}|
\end{lscommand}
\noindent command into the preamble of your document.
Use \verb|\linespread{1.3}| for ``one and a half'' line
spacing, and \verb|\linespread{1.6}| for ``double'' line spacing.  Normally
the lines are not spread, therefore the default line spread factor
is~1.\index{double line spacing}

 
\subsection{Paragraph Formatting}\label{parsp}

In \LaTeX{}, there are two parameters influencing paragraph layout.
By placing a definition like
\begin{code}
\ci{setlength}\verb|{|\ci{parindent}\verb|}{0pt}| \\
\verb|\setlength{|\ci{parskip}\verb|}{1ex plus 0.5ex minus 0.2ex}|
\end{code}
in the preamble of the input file, you can change the layout of
paragraphs. These two commands increase the space between two paragraphs
while setting the paragraph indent to zero.  

The \texttt{plus} and \texttt{minus} parts of the length above tell
\TeX{}, that it can compress and expand the inter paragraph skip by the
amount specified if this is necessary to properly fit the paragraphs
onto the page.

In continental Europe,
paragraphs are often separated by some space and not indented. But
beware, this also has its effect on the table of contents. Its lines
get spaced more loosely now as well. To avoid this, you might want to
move the two commands from the preamble into your document to some
place after the \verb|\tableofcontents| or to not use them at all,
because you'll find that most professional books use indenting and not
spacing to separate paragraphs.


If you want to indent a paragraph which is not indented, you can use 
\begin{lscommand}
\ci{indent}
\end{lscommand}
\noindent at the beginning of the paragraph.\footnote{To indent the first paragraph after each section head, use
  the \pai{indentfirst} package in the `tools' bundle.} Obviously,
this will only have an effect when \verb|\parindent| is not set to
zero.

To create a non-indented paragraph, you can use 
\begin{lscommand}
\ci{noindent}
\end{lscommand}
\noindent as the first command of the paragraph. This might come in handy when
you start a document with body text and not with a sectioning command.

\subsection{Horizontal Space}

\label{sec:hspace}
\LaTeX{} determines the spaces between words and sentences
automatically. To add horizontal space, use: \index{horizontal!space}
\begin{lscommand}
\ci{hspace}\verb|{|\emph{length}\verb|}|
\end{lscommand}
If such a space should be kept even if it falls at the end or the
start of a line, use \verb|\hspace*| instead of \verb|\hspace|.  The
\emph{length} in the simplest case just is a number plus a unit.  The
most important units are listed in Table~\ref{units}. 
\index{units}\index{dimensions}

\begin{example}
This\hspace{1.5cm}is a space 
of 1.5 cm.
\end{example}
\suppressfloats
\begin{table}[tbp]
\caption{\TeX{} Units.} \label{units}\index{units}
\begin{lined}{9.5cm} 
\begin{tabular}{@{}ll@{}}
\texttt{mm} &  millimetre $\approx 1/25$~inch \quad \demowidth{1mm} \\
\texttt{cm} & centimetre = 10~mm  \quad \demowidth{1cm}                     \\
\texttt{in} & inch $=$ 25.4~mm \quad \demowidth{1in}                    \\
\texttt{pt} & point $\approx 1/72$~inch $\approx \frac{1}{3}$~mm  \quad\demowidth{1pt}\\
\texttt{em} & approx width of an `M' in the current font \quad \demowidth{1em}\\
\texttt{ex} & approx height of an `x' in the current font \quad \demowidth{1ex}
\end{tabular}

\bigskip
\end{lined}
\end{table}

\label{cmd:stretch} 
The command
\begin{lscommand}
\ci{stretch}\verb|{|\emph{n}\verb|}|
\end{lscommand} 
\noindent generates a special rubber space. It stretches until all the
remaining space on a line is filled up. If two
\verb|\hspace{\stretch{|\emph{n}\verb|}}| commands are issued on the
same line, they grow according to the stretch factor.

\begin{example}
x\hspace{\stretch{1}}
x\hspace{\stretch{3}}x
\end{example}

\subsection{Vertical Space}
The space between paragraphs, sections, subsections, \ldots\ is
determined automatically by \LaTeX. If necessary, additional vertical
space \emph{between two paragraphs} can be added with the command:
\begin{lscommand}
\ci{vspace}\verb|{|\emph{length}\verb|}|
\end{lscommand}

This command should normally be used between two empty lines.  If the
space should be preserved at the top or at the bottom of a page, use
the starred version of the command \verb|\vspace*| instead of \verb|\vspace|.
\index{vertical space}

The \verb|\stretch| command in connection with \verb|\pagebreak| can
be used to typeset text on the last line of a page, or to centre text
vertically on a page.
\begin{code}
\begin{verbatim}
Some text \ldots

\vspace{\stretch{1}}
This goes onto the last line of the page.\pagebreak
\end{verbatim}
\end{code}

Additional space between two lines of \emph{the same} paragraph or
within a table is specified with the
\begin{lscommand}
\ci{\bs}\verb|[|\emph{length}\verb|]|
\end{lscommand}
\noindent command. 

With \ci{bigskip} and \ci{smallskip} you can skip a predefined amount of
vertical space without having to worry about exact numbers.


\section{Page Layout}

\begin{figure}[!hp]
\begin{center}
\makeatletter\@layout\makeatother
\end{center}
\vspace*{1.8cm}
\caption{Page Layout Parameters.}
\label{fig:layout}
\end{figure}

\index{page layout}
\LaTeXe{} allows you to specify the \wi{paper size} in the
\verb|\documentclass| command. It then automatically picks the right 
text \wi{margins}. But sometimes you may not be happy with 
the predefined values. Naturally, you can change them. 
%no idea why this is needed here ...
\thispagestyle{fancyplain}
Figure~\ref{fig:layout} shows all the parameters which can be changed.
The figure was produced with the \pai{layout} package from the tools bundle%
\footnote{\texttt{CTAN:/tex-archive/macros/latex/required/tools}}.

\textbf{WAIT!} \ldots before you launch into a ``Let's make that
narrow page a bit wider'' frenzy, take a few seconds to think. As with
most things in \LaTeX, there is a good reason for the page layout to
be as it is.

Sure, compared to your off-the-shelf MS Word page, it looks awfully
narrow. But take a look at your favourite book\footnote{I mean a real
  printed book produced by a reputable publisher.} and count the number
of characters on a standard text line. You will find that there are no
more than about 66 characters on each line. Now do the same on your
\LaTeX{} page. You will find that there are also about 66 characters
per line.  Experience shows that the reading gets difficult as soon as
there are more characters on a single line. This is because it is
difficult for the eyes to move from the end of one line to the start of the next one.
This is also the reason why newspapers are typeset in multiple columns.

So if you increase the width of your body text, keep in mind that you
are making life difficult for the readers of your paper. But enough
of the cautioning, I promised to tell you how you do it \ldots
 
\LaTeX{} provides two commands to change these parameters. They are
usually used in the document preamble.

The first command assigns a fixed value to any of the parameters:
\begin{lscommand}
\ci{setlength}\verb|{|\emph{parameter}\verb|}{|\emph{length}\verb|}|
\end{lscommand}

The second command adds a length to any of the parameters. 
\begin{lscommand}
\ci{addtolength}\verb|{|\emph{parameter}\verb|}{|\emph{length}\verb|}|
\end{lscommand} 

This second command is actually more useful than the \ci{setlength}
command, because you can now work relative to the existing settings.
To add one centimetre to the overall text width, I put the
following commands into the document preamble:
\begin{code}
\verb|\addtolength{\hoffset}{-0.5cm}|\\
\verb|\addtolength{\textwidth}{1cm}|
\end{code}

In this context, you might want to look at the \pai{calc} package,
it allows you to use arithmetic operations in the argument of setlength
and other places where you can enter numeric values into function
arguments.

\section{More fun with lengths}

Whenever possible, I avoid using absolute lengths in
\LaTeX{} documents. I rather try to base things on the width or height
of other page elements. For the width of a figure this could
be \verb|\textwidth| in order to make it fill the page.

The following 3 commands allow you to determine the width, height and
depth of a text string.

\begin{lscommand}
\ci{settoheight}\verb|{|\emph{lscommand}\verb|}{|\emph{text}\verb|}|\\
\ci{settodepth}\verb|{|\emph{lscommand}\verb|}{|\emph{text}\verb|}|\\
\ci{settowidth}\verb|{|\emph{lscommand}\verb|}{|\emph{text}\verb|}|
\end{lscommand}

\noindent The example below shows a possible application of these commands.

\begin{example}
\flushleft
\newenvironment{vardesc}[1]{%
  \settowidth{\parindent}{#1:\ }
  \makebox[0pt][r]{#1:\ }}{}

\begin{displaymath}
a^2+b^2=c^2
\end{displaymath}

\begin{vardesc}{Where}$a$, 
$b$ -- are adjunct to the right 
angle of a right-angled triangle.  

$c$ -- is the hypotenuse of 
the triangle and feels lonely.

$d$ -- finally does not show up 
here at all. Isn't that puzzling?
\end{vardesc}
\end{example}

\section{Boxes}
\LaTeX{} builds up its pages by pushing around boxes. At first, each
letter is a little box, which is then glued to other letters to form
words. These are again glued to other words, but with special glue,
which is elastic so that a series of words can be squeezed or
stretched as to exactly fill a line on the page. 

I admit, this is a very simplistic version of what really happens, but
the point is that \TeX{} operates on glue and boxes. Not only a letter
can be a box. You can put virtually everything into a box including
other boxes. Each box will then be handled by \LaTeX{} as if it was a
single letter.

In the past chapters you have already encountered some boxes, although
I did not tell you. The \ei{tabular} environment and the
\ci{includegraphics}, for example, both produce a box. This means that you can
easily arrange two tables or images side by side. You just have to
make sure that their combined width is not larger than the textwidth.

You can also pack a paragraph of your choice into a box with either
the

\begin{lscommand}
\ci{parbox}\verb|[|\emph{pos}\verb|]{|\emph{width}\verb|}{|\emph{text}\verb|}|
\end{lscommand}

\noindent command or the

\begin{lscommand}
\verb|\begin{|\ei{minipage}\verb|}[|\emph{pos}\verb|]{|\emph{width}\verb|}| text
\verb|\end{|\ei{minipage}\verb|}|
\end{lscommand}

\noindent environment. The \texttt{pos} parameter can take one of the letters
\texttt{c, t} or \texttt{b} to control the vertical alignment of the box,
relative to the baseline of the surrounding text. \texttt{width} takes
a length argument specifying the width of the box. The main difference
between a \ei{minipage} and a \ei{parbox} is that you cannot use all commands
and environments inside a \ei{parbox} while almost anything is possible in
a \ei{minipage}.

While \ci{parbox} packs up a whole paragraph doing line breaking and
everything, there is also a class of boxing commands which operates
only on horizontally aligned material. We already know one of them.
It's called \ci{mbox}, it simply packs up a series of boxes into
another one, and can be used to prevent \LaTeX{} from breaking two
words. As you can put boxes inside boxes, these horizontal box packers
give you ultimate flexibility.

\begin{lscommand}
\ci{makebox}\verb|[|\emph{width}\verb|][|\emph{pos}\verb|]{|\emph{text}\verb|}|
\end{lscommand}

\noindent \texttt{width} defines the width of the resulting box as
seen from the outside.\footnote{This means it can be smaller than the
  material inside the box. You can even set the
  width to 0pt so that the text inside the box will be typeset without
  influencing the surrounding boxes.}  Apart from the length
  expressions you can also use \ci{width}, \ci{height}, \ci{depth} and
  \ci{totalheight} in the width parameter. They are set from values
  obtained by measuring the typeset \emph{text}. The \emph{pos} parameter takes
a one letter value: \textbf{c}enter, \textbf{l}eft flush,
\textbf{r}ight flush or \textbf{s} which spreads the text inside the
box to fill it.

The command \ci{framebox} works exactly the same as \ci{makebox}, but
it draws a box around the text.

The following example shows you some things you could do with
the \ci{makebox} and \ci{framebox} commands.

\begin{example}
\makebox[\textwidth]{%
    c e n t r a l}\par
\makebox[\textwidth][s]{%
    s p r e a d}\par
\framebox[1.1\width]{Guess I'm 
    framed now!} \par
\framebox[0.8\width][r]{Bummer, 
    I am too wide} \par
\framebox[1cm][l]{never 
    mind, so am I} 
Can you read this?
\end{example}

Now that we control the horizontal, the obvious next step is to go for
  the vertical.\footnote{Total control is only to be obtained by
  controlling both the horizontal and the vertical \ldots}
 No problem for \LaTeX{}. The


\begin{lscommand}
\ci{raisebox}\verb|{|\emph{lift}\verb|}[|\emph{depth}\verb|][|\emph{height}\verb|]{|\emph{text}\verb|}|
\end{lscommand}

\noindent command lets you define the vertical properties of a
box. You can use \ci{width}, \ci{height}, \ci{depth} and
  \ci{totalheight} in the first three parameters, in order to act
  upon the size of the box inside the \emph{text} argument.


\begin{example}
\raisebox{0pt}[0pt][0pt]{\Large%
\textbf{Aaaa\raisebox{-0.3ex}{a}%
\raisebox{-0.7ex}{aa}%
\raisebox{-1.2ex}{r}%
\raisebox{-2.2ex}{g}%
\raisebox{-4.5ex}{h}}}
he shouted but not even the next
one in line noticed that something
terrible had happened to him.
\end{example}

\section{Rules and Struts}
\label{sec:rule}

A few pages back you may have noticed the command

\begin{lscommand}
\ci{rule}\verb|[|\emph{lift}\verb|]{|\emph{width}\verb|}{|\emph{height}\verb|}|
\end{lscommand}

\noindent In normal use it produces a simple black box.
\newpage
\begin{example}
\rule{3mm}{.1pt}%
\rule[-1mm]{5mm}{1cm}%
\rule{3mm}{.1pt}%
\rule[1mm]{1cm}{5mm}%
\rule{3mm}{.1pt}
\end{example}

\noindent This is useful for drawing vertical and horizontal
lines. The line on the title page for example, has been created with a
\ci{rule} command.

A special case is a rule with no width but a certain height. In
professional typesetting, this is called a \wi{strut}. It is used to
guarantee that an element on a page has a certain minimal height. You
could use it in a \texttt{tabular} environment to make sure a row has
a certain minimum height.

\begin{example}
\begin{tabular}{|c|}
\hline
\rule{1pt}{4ex}Pitprop \ldots\\
\hline
\rule{0pt}{4ex}Strut\\
\hline
\end{tabular}
\end{example}

%%% Local Variables: 
%%% mode: latex
%%% TeX-master: "lshort2e"
%%% End: 


\backmatter
%%%%%%%%%%%%%%%%%%%%%%%%%%%%%%%%%%%%%%%%%%%%%%%%%%%%%%%%%%%%%%%%%
% Contents: The Bibliography
% File: biblio.tex (lshort2e.tex)
% $Id: biblio.tex 14 2002-05-26 03:44:42Z marcilr $
%%%%%%%%%%%%%%%%%%%%%%%%%%%%%%%%%%%%%%%%%%%%%%%%%%%%%%%%%%%%%%%%%
\begin{thebibliography}{99}
\addcontentsline{toc}{chapter}{\numberline{}\bibname} 
\bibitem{manual} Leslie Lamport.  \newblock \emph{{\LaTeX:} A Document
    Preparation System}.  \newblock Addison-Wesley, Reading,
  Massachusetts, second edition, 1994, ISBN~0-201-52983-1.
  
\bibitem{texbook} Donald~E. Knuth.  \newblock \textit{The \TeX{}book,}
  Volume~A of \textit{Computers and Typesetting}, Addison-Wesley,
  Reading, Massachusetts, second edition, 1984, ISBN~0-201-13448-9.

\bibitem{companion} Michel Goossens, Frank Mittelbach and Alexander
  Samarin.  \newblock \emph{The {\LaTeX} Companion}.  \newblock
  Addison-Wesley, Reading, Massachusetts, 1994, ISBN~0-201-54199-8.
 
\bibitem{local} Each \LaTeX{} installation should provide a so-called
  \emph{\LaTeX{} Local Guide} which explains the things which are
  special to the local system.  It should be contained in a file called
  \texttt{local.tex}. Unfortunately, some lazy sysops do not provide such a
  document. In this case, go and ask your local \LaTeX{} guru for help.
 
\bibitem{usrguide} \LaTeX3 Project Team.  \newblock \emph{\LaTeXe~for
    authors}.  \newblock Comes with the \LaTeXe{} distribution as
  \texttt{usrguide.tex}.

\bibitem{clsguide} \LaTeX3 Project Team.  \newblock \emph{\LaTeXe~for
    Class and Package writers}.  \newblock Comes with the \LaTeXe{}
  distribution as \texttt{clsguide.tex}.

\bibitem{fntguide} \LaTeX3 Project Team.  \newblock \emph{\LaTeXe~Font
    selection}.  \newblock Comes with the \LaTeXe{} distribution as
  \texttt{fntguide.tex}.

\bibitem{graphics} D.~P.~Carlisle.  \newblock \emph{Packages in the
    `graphics' bundle}.  \newblock Comes with the `graphics' bundle as
  \texttt{grfguide.tex}, available from the same source your \LaTeX{}
  distribution came from.

\bibitem{verbatim} Rainer~Sch\"opf, Bernd~Raichle, Chris~Rowley.  
\newblock \emph{A New Implementation of \LaTeX's verbatim
  Environments}.
 \newblock Comes with the `tools' bundle as
  \texttt{verbatim.dtx}, available from the same source your \LaTeX{}
  distribution came from. 

\bibitem{catalogue} Graham~Williams.  \newblock \emph{The TeX
    Catalogue} is a very complete listing of many \TeX{} and \LaTeX{}
    related packages.
  \newblock Available online from \texttt{CTAN:/tex-archive/help/Catalogue/catalogue.html}
  
\bibitem{eps} Keith~Reckdahl.  \newblock \emph{Using EPS Graphics in
    \LaTeXe{} Documents} which explains everything and much more than
  you ever wanted to know about EPS files and their use in \LaTeX{}
  documents.  \newblock Available online from
  \texttt{CTAN:/tex-archive/info/epslatex.ps}

\end{thebibliography}


%%% Local Variables: 
%%% mode: latex
%%% TeX-master: "lshort2e"
%%% End: 

\refstepcounter{chapter}
\addcontentsline{toc}{chapter}{Index} 
\printindex
\refstepcounter{chapter}
\label{verylast}
\mbox{}
\end{document}





%

% Local Variables:
% TeX-master: "lshort2e"
% mode: latex
% mode: flyspell
% End:
